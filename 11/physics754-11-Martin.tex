\input{../header.tex}

\newcommand\problemset{11}

\hypersetup{
    pdftitle={physics754 Problem Set \problemset}
}

\newenvironment{aside}{\itshape\small}{}

%\subject{}
\title{physics754 -- Problem Set \problemset}
%\subtitle{}
\author{
    Martin Ueding \\ \small{\href{mailto:mu@martin-ueding.de}{mu@martin-ueding.de}}
}
\publishers{Group 5 -- Olaf Baake}

\begin{document}

\maketitle

\section*{Questions}

\subsection*{Equation of state}

In the lecture notes, the equation of state is written as
\[
    p(t) = p\del{\epsilon(t)}.
\]
If that is an equation and $p$ an invertible function, then I could transform
this into
\[
    t = \epsilon(t),
\]
which is probably not what is really meant. If it was
\[
    p(t) := p\del{\epsilon(t)}
\]
where $p(t)$ is a different function than $p(\epsilon)$, which I find
cumbersome, then it would just say that the overloaded $p$ has the same value
that the previous $p$ had when it is evaluated at the value of $\epsilon(t)$.
That does not make more sense either.

As far as I recall from physik521, the equation of state was something like
like
\[
    f(p, \epsilon) = 0.
\]

The physik521 lecture notes read:
\begin{quote}
    The functional relation between all the (relevant) state variables in
    equilibrium is called \emph{equation of state}: $f(P, V, N, T) = 0$.
\end{quote}

So how does $p(t) = p(\epsilon(t))$ and the quote from the lecture notes fit
together?

\vspace*{10cm}

\subsection*{Cosmological constant and vacuum energy}

In the lecture, he has shown that if $\epsilon(t) = \lambda p(t)$ with $\lambda
= 1$ holds, this kind of matter does the same as a nonzero cosmological
constant would, mathematically. He also said that one could either take the
mathematical stance and say that the Hilbert theorem allows for such a
constant, or propose that some dark energy exists.

As far as I read in \parencite{penrose-road_to_reality}, the energy of the
vacuum does not vanish, but has fluctuations in it. Therefore, there is
something called vacuum energy.

The dark energy seems to have the property that the energy density stays the
same regardless of the size. So if you increase the volume, the total energy
increases. This sounds compatible with that vacuum energy, since the energy
density of the vacuum should not depend on the volume, I would think
intuitively.

Also, having negative pressure seems to be compatible with the vacuum version
of the Casimir effect. Two plates in vacuum attract each other because the
vacuum in between the plates is more empty than the surrounding vacuum. It is
more empty since the wave functions of the virtual particles are bounded by the
plates and therefore have less degrees of freedom. With ships on sea, there is
the same effect that ships get pushed together since there are less waves
pressing them apart in between the ships.

The catch seems to be that the difference in the cosmological constant and the
quantum mechanical vacuum energy is some \num{e120}. Does that mean that,
although the correspondence seems intuitive at first, it does not hold? Or does
that rather mean that this is a use case for a grand unified theory and one
cannot say anything about that as of now?

\vspace*{10cm}

\section*{P.12: The Robertson-Walker-metric for a universe filled with dust
particles}

\subsection*{(a)}

I have to show that
\[
    \frac{\epsilon}{\epsilon_0} = \sbr{\frac{R}{R_0}}^{-3}
\]
holds.

Since we are looking at dust with zero pressure, the equation (2) from the
problem set reduces to
\[
    \dod{}t \epsilon R^3 = 0,
\]
meaning that the energy contained in a sphere does not change. If $R$ is
assumed to be the radius of a spherical universe, this means that the total
energy within that universe does not change. This is because there is no
pressure to work against.

This differential equation can be solved to yield
\[
    \sbr{\epsilon R^3}(t) = c.
\]
With the initial condition that at $t = 0$ it has to be
\[
    \sbr{\epsilon R^3}(0) = \epsilon_0 R_0^3,
\]
Therefore,
\[
    \sbr{\epsilon R^3}(t) = \epsilon_0 R_0^3,
\]
It follows that
\[
    \frac{\epsilon(t)}{\epsilon_0} = \sbr{\frac{R(t)}{R_0}}^{-3}.
\]
Dropping the explicit functional evaluation, this yields equation (1) from the
problem set:
\[
    \frac{\epsilon}{\epsilon_0} = \sbr{\frac{R}{R_0}}^{-3}.
\]

\subsection*{(b)}

I have to show that
\[
    \sbr{\frac{\dot R}{R_0}}^2 = H_0^2 \sbr{1 - 2 q_0 + 2q_0 \frac{R_0}R}
\]
holds.

Since I do not see where I should start, I try to go backwards.
\begin{align*}
    \sbr{\frac{\dot R}{R_0}}^2 &= H_0^2 \sbr{1 - 2 q_0 + 2q_0 \frac{R_0}R} \\
    \iff \dot R^2 &= \dot R_0^2 \sbr{1 - 2 q_0 + 2q_0 \frac{R_0}R} \\
    \intertext{%
        Now I can insert
        \[
            q_0 := \frac{4\piup}3 \frac{G \epsilon_0}{H_0^2}
        \]
        into the equation.
    }
    \iff \dot R^2 &= \dot R_0^2 \sbr{1 - 2 \frac{4\piup}3 \frac{G \epsilon_0}{H_0^2}
+ 2\frac{4\piup}3 \frac{G \epsilon_0}{H_0^2} \frac{R_0}R} \\
\iff \dot R^2 &= R_0^2 \sbr{1 - \frac{8\piup}3 \frac{G \epsilon_0}{H_0^2}
    + \frac{8\piup}3 \frac{G \epsilon_0}{H_0^2} \frac{R_0}R} \\
    \intertext{%
        Since the definition of $H_0$ is given by
        \[
            H_0 := \frac{\dot R_0}{R_0},
        \]
        there are more of the $R_0$.
    }
    \iff \dot R^2 &= \dot R_0^2 \sbr{1 - \frac{8\piup}3 \frac{G \epsilon_0}{\dot R_0^2}
R_0^2 + \frac{8\piup}3 \frac{G \epsilon_0}{\dot R_0^2} \frac{R_0^3}R} \\
\intertext{%
    Cancel out $\dot R^2$.
}
\iff \dot R^2 &= \dot R_0^2 - \frac{8\piup}3 G \epsilon_0
R_0^2 + \frac{8\piup}3 G \epsilon_0 \frac{R_0^3}R \\
\intertext{%
    Then I can use equation (4) from the problem set in the last summand and
    then cancel out one $R$.
}
\iff \dot R^2 &= \dot R_0^2 - \frac{8\piup}3 G \epsilon_0
R_0^2 + \frac{8\piup}3 G \epsilon R^2 \\
\iff \dot R^2 - \dot R_0^2 &=
\frac{8\piup}3 G \epsilon R^2 - \frac{8\piup}3 G \epsilon_0 R_0^2
\end{align*}
After all, this is just equation (1) evaluated at $t$ minus the same equation
evaluated at 0. Now one could reverse this and have an actual derivation. The
difference also explains why $k$ does not show up in any of this.

\subsection*{(c)}

I have to show that
\[
    t = \frac{1}{H_0} \int_0^{R/R_0} \dif \rho \, \sbr{1 - 2q_0 +
    \frac{2q_0}{\rho}}^{-1/2}
\]
holds.

I will start with equation (5) from the problem set that was derived in the
previous problem:
\begin{align*}
    \sbr{\frac{\dot R}{R_0}}^2 &= H_0^2 \sbr{1 - 2 q_0 + 2q_0 \frac{R_0}R}. \\
    \intertext{%
        I put everything to one side.
    }
    \iff \frac{\dot R}{R_0} \frac1{H_0} \sbr{1 - 2 q_0 + 2q_0
    \frac{R_0}R}^{-1/2} &= 1 \\
        \intertext{%
            Writing down the derivative explicitly gives
        }
    \iff \dod Rt \frac{1}{R_0} \frac1{H_0} \sbr{1 - 2 q_0 + 2q_0
    \frac{R_0}R}^{-1/2} &= 1 \\
        \intertext{%
            Where the derivative can be split up to both sides of the equation.
            Look away if you faint from seeing blood.
        }
    \iff \dif R \frac{1}{R_0} \frac1{H_0} \sbr{1 - 2 q_0 + 2q_0
    \frac{R_0}R}^{-1/2} &= \dif t \\
        \intertext{%
            I define
            \[
                \rho := \frac{R}{R_0}
            \]
            with the total differential
            \[
                \dif \rho := \frac{\dif R}{R_0}.
            \]
            Inserting that into the equation gives
        }
    \iff \dif \rho \frac1{H_0} \sbr{1 - 2 q_0 + \frac{2q_0}\rho}^{-1/2} &= \dif
        t. \\
        \intertext{%
            Now I integrate on both sides. Before I do that, I will rename the
            integration variables from $X$ to $X'$ in order to distinguish
            them from the boundaries.
        }
        \iff \frac1{H_0} \int_0^{\rho} \dif \rho' \sbr{1 - 2 q_0 +
        \frac{2q_0}{\rho'}}^{-1/2} &= \int_0^t \dif t' \\
            \intertext{%
                Inserting $\rho$ into the boundary give me the chance to rename
                the integration variables back. However, the $\rho$ in the
                integration is a “free” integration variable and not the
                $R/R_0$ that is supplied from “outside” of that integral. I
                just do not like overloading the meaning of symbols with
                different things. So after renaming and integration I am left
                with
            }
        \iff \frac1{H_0} \int_0^{R/R_0} \dif \rho \sbr{1 - 2 q_0 +
        \frac{2q_0}\rho}^{-1/2} &= t.
\end{align*}

The age of the universe is obtained when we let $R \to R_0$.

\subsection*{(d)}

\subsection*{(e)}

\subsection*{(f)}

\section*{H.14: The Robertson-Walker-metric for a universe dominated by
radiation}

\IfFileExists{\bibliographyfile}{
    \printbibliography
}{}

\end{document}

% vim: spell spelllang=en tw=79

\input{../header.tex}

\newcommand\problemset{11}

\hypersetup{
    pdftitle={physics754 Problem Set \problemset}
}

\newenvironment{aside}{\itshape\small}{}

%\subject{}
\title{physics754 -- Problem Set \problemset}
%\subtitle{}
\author{
    Martin Ueding \\ \small{\href{mailto:mu@martin-ueding.de}{mu@martin-ueding.de}}
}
\publishers{Group 5 -- Olaf Baake}

\begin{document}

\maketitle

\section*{P.12: The Robertson-Walker-metric for a universe filled with dust
particles}

\subsection*{(a): Energy density}

I have to show that
\[
    \frac{\epsilon}{\epsilon_0} = \sbr{\frac{R}{R_0}}^{-3}
\]
holds.

Since we are looking at dust with zero pressure, the equation (2) from the
problem set reduces to
\[
    \dod{}t \epsilon R^3 = 0,
\]
meaning that the energy contained in a sphere does not change. If $R$ is
assumed to be the radius of a spherical universe, this means that the total
energy within that universe does not change. This is because there is no
pressure to work against.

This differential equation can be solved to yield
\[
    \sbr{\epsilon R^3}(t) = c.
\]
With the initial condition that at the point $t = 0$ it has to be
\[
    \sbr{\epsilon R^3}(0) = \epsilon_0 R_0^3,
\]
the constant is fixed to
\[
    \sbr{\epsilon R^3}(t) = \epsilon_0 R_0^3.
\]
It follows that
\[
    \frac{\epsilon(t)}{\epsilon_0} = \sbr{\frac{R(t)}{R_0}}^{-3}.
\]
Dropping the explicit functional evaluation, this yields equation (1) from the
problem set:
\[
    \frac{\epsilon}{\epsilon_0} = \sbr{\frac{R}{R_0}}^{-3}.
\]

\subsection*{(b): Radius evolution}

I have to show that
\[
    \sbr{\frac{\dot R}{R_0}}^2 = H_0^2 \sbr{1 - 2 q_0 + 2q_0 \frac{R_0}R}
\]
holds.

Since I do not see where I should start, I try to go backwards.
\begin{align*}
    \sbr{\frac{\dot R}{R_0}}^2 &= H_0^2 \sbr{1 - 2 q_0 + 2q_0 \frac{R_0}R} \\
    \iff \dot R^2 &= \dot R_0^2 \sbr{1 - 2 q_0 + 2q_0 \frac{R_0}R} \\
    \intertext{%
        Now I can insert
        \[
            q_0 := \frac{4\piup}3 \frac{G \epsilon_0}{H_0^2}
        \]
        into the equation.
    }
    \iff \dot R^2 &= \dot R_0^2 \sbr{1 - 2 \frac{4\piup}3 \frac{G \epsilon_0}{H_0^2}
+ 2\frac{4\piup}3 \frac{G \epsilon_0}{H_0^2} \frac{R_0}R} \\
\iff \dot R^2 &= R_0^2 \sbr{1 - \frac{8\piup}3 \frac{G \epsilon_0}{H_0^2}
    + \frac{8\piup}3 \frac{G \epsilon_0}{H_0^2} \frac{R_0}R} \\
    \intertext{%
        Since the definition of $H_0$ is given by
        \[
            H_0 := \frac{\dot R_0}{R_0},
        \]
        there are more of the $R_0$.
    }
    \iff \dot R^2 &= \dot R_0^2 \sbr{1 - \frac{8\piup}3 \frac{G \epsilon_0}{\dot R_0^2}
R_0^2 + \frac{8\piup}3 \frac{G \epsilon_0}{\dot R_0^2} \frac{R_0^3}R} \\
\intertext{%
    Cancel out $\dot R^2$.
}
\iff \dot R^2 &= \dot R_0^2 - \frac{8\piup}3 G \epsilon_0
R_0^2 + \frac{8\piup}3 G \epsilon_0 \frac{R_0^3}R \\
\intertext{%
    Then I can use equation (4) from the problem set in the last summand and
    then cancel out one $R$.
}
\iff \dot R^2 &= \dot R_0^2 - \frac{8\piup}3 G \epsilon_0
R_0^2 + \frac{8\piup}3 G \epsilon R^2 \\
\iff \dot R^2 - \dot R_0^2 &=
\frac{8\piup}3 G \epsilon R^2 - \frac{8\piup}3 G \epsilon_0 R_0^2
\end{align*}
After all, this is just equation (1) evaluated at $t$ minus the same equation
evaluated at 0. Now one could reverse this and have an actual derivation. The
difference also explains why $k$ does not show up in any of this.

\subsection*{(c): Age of the universe}

I have to show that
\[
    t = \frac{1}{H_0} \int_0^{R/R_0} \dif \rho \, \sbr{1 - 2q_0 +
    \frac{2q_0}{\rho}}^{-1/2}
\]
holds.

I will start with equation (5) from the problem set that was derived in the
previous problem:
\begin{align*}
    \sbr{\frac{\dot R}{R_0}}^2 &= H_0^2 \sbr{1 - 2 q_0 + 2q_0 \frac{R_0}R}. \\
    \intertext{%
        I put everything to one side.
    }
    \iff \frac{\dot R}{R_0} \frac1{H_0} \sbr{1 - 2 q_0 + 2q_0
    \frac{R_0}R}^{-1/2} &= 1 \\
        \intertext{%
            Writing down the derivative explicitly gives
        }
    \iff \dod Rt \frac{1}{R_0} \frac1{H_0} \sbr{1 - 2 q_0 + 2q_0
    \frac{R_0}R}^{-1/2} &= 1 \\
        \intertext{%
            Where the derivative can be split up to both sides of the equation.
            Look away if you faint from seeing derivatives used as fractions.
            Rated “P” for “Physicist”.
        }
    \iff \dif R \frac{1}{R_0} \frac1{H_0} \sbr{1 - 2 q_0 + 2q_0
    \frac{R_0}R}^{-1/2} &= \dif t \\
        \intertext{%
            I define
            \[
                \rho := \frac{R}{R_0}
            \]
            with the total differential
            \[
                \dif \rho := \frac{\dif R}{R_0}.
            \]
            Inserting that into the equation gives
        }
    \iff \dif \rho \frac1{H_0} \sbr{1 - 2 q_0 + \frac{2q_0}\rho}^{-1/2} &= \dif
        t. \\
        \intertext{%
            Now I integrate on both sides. Before I do that, I will rename the
            integration variables from $X$ to $X'$ in order to distinguish
            them from the boundaries.
        }
        \iff \frac1{H_0} \int_0^{\rho} \dif \rho' \sbr{1 - 2 q_0 +
        \frac{2q_0}{\rho'}}^{-1/2} &= \int_0^t \dif t' \\
            \intertext{%
                Inserting $\rho$ into the boundary give me the chance to rename
                the integration variables back. However, the $\rho$ in the
                integration is a “free” integration variable and not the
                $R/R_0$ that is supplied from “outside” of that integral. I
                just do not like overloading the meaning of symbols with
                different things. So after renaming and integration I am left
                with
            }
        \iff \frac1{H_0} \int_0^{R/R_0} \dif \rho \sbr{1 - 2 q_0 +
        \frac{2q_0}\rho}^{-1/2} &= t.
\end{align*}

The age of the universe is obtained when we let $R \to R_0$. The $R_0$ is
defined to be the size parameter at the current time, so the upper bound has to
be 1.

\subsection*{(d): Large deceleration}

I tried, but I did not succeed in showing that. I will wait for the exercise.

\subsection*{(e): Medium deceleration}

For the case that $q_0 = 1/2$ it should be shown that
\[
    \frac{R}{R_0} = \sbr{\frac 32 H_0 t}^{2/3}
\]
holds.

I have tried, but I came up with a different result. Let me show. With the
given value of $q_0$ the relation (7) on the problem set reduces to
\[
    1 - \cos(\theta) = 0.
\]
Therefore, the formula (6) on the set reduces to
\[
    t = \frac{1}{H_0} \int_0^{R/R_0} \dif \rho \, \rho^{-1/2}.
\]
That is, note the factor 2 at the beginning,
\[
    t = \frac{2}{H_0} \int_0^{R/R_0} \dif \rho \, \dod{}\rho \sqrt\rho.
\]

\begin{aside}
    I would like to see whether I understand this correctly. So let me attempt
    to use fancy words.
\end{aside}

In the above, I cancel the $\dif \rho$ and get
\[
    t = \frac{2}{H_0} \int_0^{R/R_0} \dif \sqrt\rho.
\]

The fundamental theorem of exterior calculus says for a p-form $\alpha$ and a
set $\Omega \subseteq M$ the following holds:
\[
    \intop_\Omega \dif \alpha = \intop_{\partial \Omega} \alpha.
\]
The boundary of a compact interval is just the set of the boundary points, one
counting negative and the other one positive.

In this case, $\sqrt p$ is a 0-form, i.\,e. a simple $M \mapsto \R$ function. So
that makes $\dif\sqrt p$ a 1-form, which is a total differential.

So I can write the above integral as
\[
    t = \frac{2}{H_0} \intop_{[0, {R}/{R_0}]} \dif \sqrt\rho
    = \frac{2}{H_0} \intop_{\partial [0, {R}/{R_0}]} \sqrt\rho
    = \frac{2}{H_0} \sqrt{\frac{R}{R_0}}.
\]

When I solve this for $R/R_0$, I get something different, though:
\[
    \frac{R}{R_0} = \sbr{\frac 12 H_0t}^2.
\]
The expected result from the problem set looks like it was integrated yet
another time, but I do not see where that would come into.

\subsection*{(f): Small deceleration}

I tried, but I did not succeed in showing that. I will wait for the exercise.

\section*{H.14: The Robertson-Walker-metric for a universe dominated by
radiation}

\subsection*{(a): Energy density}

Now there is $p = \epsilon/3$. Therefore:
\begin{align*}
    \dod{}t \epsilon R^3 &= - p \dod{}t R^3 \\
    \iff \dod{}t \epsilon R^3 &= - \frac13 \epsilon \dod{}t R^3 \\
    \iff \dot \epsilon R^3 + 3 \epsilon R^2 \dot R &= - \epsilon R^2 \dot R \\
    \iff \dot \epsilon R^3 &= - 4 \epsilon R^2 \dot R \\
    \iff \frac{\dot\epsilon}{\epsilon} &= -4 \frac{\dot R}{R} \\
    \intertext{%
        I now cancel a $\dif t$ on both sides. Again, rated “P” for
        “Physicist”.
    }
    \iff \frac{\dif\epsilon}{\epsilon} &= -4 \frac{\dif R}{R} \\
    \intertext{%
        These are the differentials of a function composed with the logarithm.
        Therefore, the solution is
    }
    \iff \ln\del{\epsilon(t)} &= -4 \ln\del{R(t)} + c_1 \\
    \iff \epsilon(t) &= c_2 R^{-4}. \\
    \intertext{%
        With the initial condition that for $t = 0$ the value is $\epsilon_0$,
        this gives
    }
    \iff \epsilon(t) &= \epsilon_0 R_0^2 R(t)^{-4} \\
    \iff \frac{\epsilon}{\epsilon_0} &= \sbr{\frac{R}{R_0}}^{-4}.
\end{align*}

\subsection*{(b): Radius evolution}

I go the right way compared to the derivation that I did in the presence
problem above. I subtract equation (1) from itself evaluated at $t=0$. That
gives me
\begin{align*}
    \dot R^2 - \dot R_0^2 &= \frac{8\piup}3 G \sbr{\epsilon R^2 - \epsilon_0 R_0^2} \\
    \intertext{%
        With the equation derived above, the last summand can be written at time
        $t_0$:
    }
    \iff \dot R^2 - \dot R_0^2 &= \frac{8\piup}3 G \sbr{\epsilon_0 R_0^4
\frac{1}{R^2} - \epsilon_0 R_0^2} \\
    \intertext{%
        Putting $\dot R_0^2$ to the other side …
    }
    \iff \dot R^2 &= \dot R_0^2 + \frac{8\piup}3 G \sbr{\epsilon_0 R_0^4
\frac{1}{R^2} - \epsilon_0 R_0^2} \\
    \intertext{%
        Expanding …
    }
    \iff \dot R^2 &= \dot R_0^2 + \frac{8\piup}3 G \epsilon_0 R_0^4
    \frac{1}{R^2} - \frac{8\piup}3 G \epsilon_0 R_0^2 \\
    \intertext{%
        Factoring in a $\dot R_0^2$ …
    }
    \iff \dot R^2 &= \dot R_0^2 \sbr{1 + \frac{8\piup}3 \frac{G \epsilon_0
    R_0^2}{\dot R_0^2} \frac{R_0^2}{R^2} - \frac{8\piup}3 \frac{G \epsilon_0
    R_0^2}{\dot R_0^2}} \\
    \intertext{%
        Inserting $H_0$ …
    }
    \iff \dot R^2 &= \dot R_0^2 \sbr{1 + \frac{8\piup}3 \frac{G \epsilon_0}{H_0^2} \frac{R_0^2}{R^2} - \frac{8\piup}3 \frac{G \epsilon_0}{H_0^2}} \\
    \intertext{%
        Inserting $q_0$ …
    }
    \iff \dot R^2 &= \dot R_0^2 \sbr{1 + 2q_0 \frac{R_0^2}{R^2} - 2q_0} \\
    \intertext{%
        Bringing it into the same form …
    }
    \iff \dot R^2 &= \dot R_0^2 \sbr{1 - 2q_0 + 2q_0 \frac{R_0^2}{R^2}}
\end{align*}

So this is the same as the dust universe, except for the power of the last
$R_0/R$ term.

\subsection*{(c): Age of the universe}

The derivation is the same as for the dust universe, except that the power of
$\rho$ in the denominator is two.

\section*{Questions}

\subsection*{Equation of state}

In the lecture notes, the equation of state is written as
\[
    p(t) = p\del{\epsilon(t)}.
\]
If that is an equation and $p$ an invertible function, then I could transform
this into
\[
    t = \epsilon(t),
\]
which is probably not what is really meant. If it was
\[
    p(t) := p\del{\epsilon(t)}
\]
where $p(t)$ is a different function than $p(\epsilon)$, which I find
cumbersome, then it would just say that the overloaded $p$ has the same value
that the previous $p$ had when it is evaluated at the value of $\epsilon(t)$.
That does not make more sense either.

As far as I recall from physik521, the equation of state was something like
like
\[
    f(p, \epsilon) = 0.
\]

The physik521 lecture notes read:
\begin{quote}
    The functional relation between all the (relevant) state variables in
    equilibrium is called \emph{equation of state}: $f(P, V, N, T) = 0$.
\end{quote}

So how does $p(t) = p(\epsilon(t))$ and the quote from the lecture notes fit
together?

\vfill
\vspace*{10cm}

\subsection*{Cosmological constant and vacuum energy}

In the lecture, he has shown that if $\epsilon(t) = \lambda p(t)$ with $\lambda
= 1$ holds, this kind of matter does the same as a nonzero cosmological
constant would, mathematically. He also said that one could either take the
mathematical stance and say that the Hilbert theorem allows for such a
constant, or propose that some dark energy exists.

As far as I read in \parencite{penrose-road_to_reality}, the energy of the
vacuum does not vanish, but has fluctuations in it. Therefore, there is
something called vacuum energy.

The dark energy seems to have the property that the energy density stays the
same regardless of the size. So if you increase the volume, the total energy
increases. This sounds compatible with that vacuum energy, since the energy
density of the vacuum should not depend on the volume, I would think
intuitively.

Also, having negative pressure seems to be compatible with the vacuum version
of the Casimir effect. Two plates in vacuum attract each other because the
vacuum in between the plates is more empty than the surrounding vacuum. It is
more empty since the wave functions of the virtual particles are bounded by the
plates and therefore have less degrees of freedom. With ships on sea, there is
the same effect that ships get pushed together since there are less waves
pressing them apart in between the ships.

The catch seems to be that the difference in the cosmological constant and the
quantum mechanical vacuum energy is some \num{e120}. Does that mean that,
although the correspondence seems intuitive at first, it does not hold? Or does
that rather mean that this is a use case for a grand unified theory and one
cannot say anything about that as of now?

\vfill
\vspace*{10cm}

\subsection{Energy density of gravitational waves}

In the energy momentum tensor $\tens T$, the electromagnetic waves are a form
of energy density. With $\tens G = 8 \piup G \tens T$, this means that photons
create curvature.

The gravitational waves carry energy. Are they also a sort of energy density?
If so, do they need to be included in the energy momentum tensor or are they
taken care of by the Einstein tensor itself?

\vfill
\vspace*{10cm}

\IfFileExists{\bibliographyfile}{
    \printbibliography
}{}

\end{document}

% vim: spell spelllang=en tw=79

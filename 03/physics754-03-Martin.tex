\input{../header.tex}

\newcommand\problemset{3}

\hypersetup{
    pdftitle={physics754 Problem Set \problemset}
}

\newenvironment{aside}{\itshape\small}{}

%\subject{}
\title{physics754 -- Problem Set \problemset}
%\subtitle{}
\author{
    Martin Ueding \\ \small{\href{mailto:mu@martin-ueding.de}{mu@martin-ueding.de}}
}
\publishers{Group 5 -- Olaf Baake}

\begin{document}

\maketitle

\section{Bianchi Identities}

\subsection{Show the differential Bianchi identity}

I have to show that:
\[
    R_{\alpha\beta\mu\nu;\lambda}
    + R_{\alpha\beta\lambda\mu;\nu}
    + R_{\alpha\beta\nu\lambda;\mu}
    = 0.
\]
Since this tensor is antisymmetric in its last two indices, I think it is the
same as $R_{\alpha\beta[\mu\nu;\lambda]} = 0$. The hint says that I only have
to show
\[
    R_{\alpha\beta\mu\nu,\lambda}
    + R_{\alpha\beta\lambda\mu,\nu}
    + R_{\alpha\beta\nu\lambda,\mu}
    = 0.
\]

Since $\vec\Gamma(\vec x) = \tens 0$ and only $\vec\partial \vec\Gamma(\vec x)
\neq \tens 0$, this individual summands reduce to:
\[
    R^\alpha_{\beta\mu\nu,\lambda} = \Gamma^\alpha_{\beta\nu,\mu,\lambda}
    - \Gamma^\alpha_{\beta\mu,\nu,\lambda}
    \eqnsep
    R^\alpha_{\beta\lambda\mu,\nu} = \Gamma^\alpha_{\beta\mu,\lambda,\nu}
    - \Gamma^\alpha_{\beta\lambda,\mu,\nu}
    \eqnsep
    R^\alpha_{\beta\nu\lambda,\mu} = \Gamma^\alpha_{\beta\lambda,\nu,\mu}
    - \Gamma^\alpha_{\beta\nu,\lambda,\mu}.
\]

Now I just add them all together and get
\[
    + \Gamma^\alpha_{\beta\nu,\mu,\lambda}
    - \Gamma^\alpha_{\beta\mu,\nu,\lambda}
    + \Gamma^\alpha_{\beta\mu,\lambda,\nu}
    - \Gamma^\alpha_{\beta\lambda,\mu,\nu}
    + \Gamma^\alpha_{\beta\lambda,\nu,\mu}
    - \Gamma^\alpha_{\beta\nu,\lambda,\mu},
\]
which can be rearranged (last summand into second place) to
\[
    + \Gamma^\alpha_{\beta\nu,\mu,\lambda}
    - \Gamma^\alpha_{\beta\nu,\lambda,\mu}
    - \Gamma^\alpha_{\beta\mu,\nu,\lambda}
    + \Gamma^\alpha_{\beta\mu,\lambda,\nu}
    - \Gamma^\alpha_{\beta\lambda,\mu,\nu}
    + \Gamma^\alpha_{\beta\lambda,\nu,\mu}
\]
which is 0 given the commutative property of the partial derivatives.

\subsection{Argument for equation (7)}

It should be shown that $G^{\mu\nu}_{;\mu} = 0$ holds. Given the definition of
$\tens G$, this is:
\begin{align*}
    \nabla_\mu G^{\mu\nu}
    &= \nabla_\mu g^{\mu\alpha} g^{\nu\beta} R_{\alpha\beta} - \frac12
    \nabla_\mu g^{\mu\nu} \mathcal R \\
    \intertext{%
        All the $\vnabla \tens g$ terms are 0.
    }
    &= g^{\mu\alpha} g^{\nu\beta} \nabla_\mu R_{\alpha\beta} - \frac12
    g^{\mu\nu} \nabla_\mu \mathcal R \\
    &= g^{\mu\alpha} g^{\nu\beta} \nabla_\mu R^\lambda_{\alpha\lambda\beta} + \nabla_\mu
    g^{\beta\nu} R_{\beta\nu} \\
    &= g^{\mu\alpha} g^{\nu\beta} \nabla_\mu R^\lambda_{\alpha\lambda\beta} + g^{\beta\nu}
    \nabla_\mu R_{\beta\nu} \\
    &= g^{\mu\alpha} g^{\nu\beta} \nabla_\mu R^\lambda_{\alpha\lambda\beta} + g^{\beta\nu}
    \nabla_\mu R^\lambda_{\beta\lambda\nu} \\
    \intertext{%
        I found this easier without all the explicit occurrences of $\tens g$.
        That and $\vnabla$ commute, so I can make this implicit by raising and
        lowering the indices, keeping track of their order.
    }
    &= \nabla_\mu R^{\lambda\mu}{}_\lambda{}^\nu - \frac12 \nabla^\nu
    R^{\lambda\rho}{}_{\lambda\rho} \\
    &= R^{\lambda\mu}{}_\lambda{}^\nu{}_{;\mu} + \frac12
    (R^{\lambda\rho\nu}{}_{\lambda;\rho} +
    R^{\lambda\rho}{}_\rho{}^\nu{}_{;\lambda}) \\
    &= R^{\lambda\mu}{}_\lambda{}^\nu{}_{;\mu} + \frac12
    (R^{\lambda\rho\nu}{}_{\lambda;\rho} +
    R^{\rho\lambda}{}_\lambda{}^\nu{}_{;\rho}) \\
    &= R^{\lambda\mu}{}_\lambda{}^\nu{}_{;\mu} + \frac12
    (R^{\lambda\rho\nu}{}_{\lambda;\rho} +
    R^{\lambda\rho\nu}{}_{\lambda;\rho}) \\
    &= R^{\lambda\mu}{}_\lambda{}^\nu{}_{;\mu} +
    R^{\lambda\rho\nu}{}_{\lambda;\rho} \\
    &= R^{\lambda\rho}{}_\lambda{}^\nu{}_{;\rho} -
    R^{\lambda\rho}{}_\lambda{}^\nu{}_{;\rho} \\
    &= 0
\end{align*}

\section{Normal Coordinates}

\subsection{Conclusion}

If all the geodesics are straight lines, that means that the covariant and
partial derivative are equal. Since
\[
    \nabla_\mu \ev_i = \partial_\mu \ev_i + \Gamma^k_{\mu i} \ev_k,
\]
the Christoffel-symbols have to be zero in that case. If that is the case, the
relation holds.

I can construct the symbol from equation (16) like so:
\[
    (\nabla_\alpha y^\beta) y^\alpha = (\partial_\alpha y^\beta) y^\alpha
    + \Gamma^\beta_{\alpha\lambda} y^\alpha y^\nabla.
\]
If I require that $\nabla \sim \partial$ holds in this case, the term with
$\Gamma$ has to be zero. I can require that in this coordinate system, since
geodesics are straight lines. I think that also means that we are in a sort of
“inertial reference frame”, where the covariant derivatives are just the
partial derivatives.

\subsection{Implication}

The Christoffel-symbol is given as
\[
    \Gamma^\gamma_{\alpha\beta}
    = \frac12 g^{\gamma\delta} \del{
        g_{\beta\delta,\alpha} + g_{\alpha\delta,\beta} +
        g_{\alpha\beta,\delta}
    }.
\]

The property that has to be shown is that there is no “slope” in $\tens g$ at
the origin. I could not come up with a sound reason why this has to be the case
if the previous relation holds.


\IfFileExists{\bibliographyfile}{
    \printbibliography
}{}

\end{document}

% vim: spell spelllang=en tw=79

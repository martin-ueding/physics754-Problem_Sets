\input{../header.tex}

\newcommand\problemset{6}

\hypersetup{
    pdftitle={physics754 Problem Set \problemset}
}

\newenvironment{aside}{\itshape\small}{}

\let\Call\del
\let\Group\sbr

\newcommand\call[1]{(#1)}
\newcommand\group[1]{(#1)}

%\subject{}
\title{physics754 -- Problem Set \problemset}
%\subtitle{}
\author{
    Martin Ueding \\ \small{\href{mailto:mu@martin-ueding.de}{mu@martin-ueding.de}}
}
\publishers{Group 5 -- Olaf Baake}

\begin{document}

\maketitle

Sorry for the landscape printing. The equations got pretty long and I did not
want to break them all into several lines since they have so many parentheses.

\section*{H.9: Schwarzschild trajectories given by the Jacobi method}

I will omit the subscript $q$ since there are no other velocities. Vectors in
$\R^3$ are typeset in bold serif italics.

\subsection*{(a) Canonical momenta}

The canonical momenta are:
\[
    p_i = \dpd L{{v^i}} = - m \frac{1}{2f(\vec q, \vec v, q^0)} \Group{
    2 g_{0i}(q^0, \vec q) + 2 g_{ij}(q^0, \vec q) v^j}.
\]

I will assemble the hamiltonian from that. Since the function arguments to do
change within the following calculations, I will just omit them. $f$ and $g$
are still functions.
\begin{align*}
    H(\vec q, \vec p)
    &= \Group{p_i v^i - L}_{v = v(\vec q, \vec p)} \\
    %
    &= - \Group{\frac{m}{2f} \sbr{ 2 g_{0i} +
    2 g_{ij} v^j} v^i - L}_{\vec v = \vec v(\vec q, \vec p)} \\
    %
    &= - \Group{\frac{m}{f} \sbr{ g_{0i} +
    g_{ij} v^j} v^i - L}_{\vec v = \vec v(\vec q, \vec p)} \\
    %
    &= - \Group{\frac{m}{f} \Group{ g_{0i}
    v^i + g_{ij} v^i v^j - g_{00} - 2 g_{0i} v^i - g_{ik} v^i
    v^k}}_{\vec v = \vec v(\vec q, \vec p)} \\
    %
    &= - \Group{\frac{m}{f} \Group{ - g_{0i}
    v^i + g_{ij} v^i v^j - g_{00} }}_{\vec v = \vec v(\vec q, \vec
    p)} \\
    %
    &= \Group{\frac{m}{f} \Group{ g_{00} + g_{0i} v^i}}_{\vec v = \vec v(\vec q, \vec p)}
\end{align*}

And that is the result from the problem set.

\subsection*{(b) Differential equation for $W$}

\begin{align*}
    m^2
    &\overset != 
    \bracket{\tens \partial W, \tens \partial W} \\
    %
    &= g^{00} [\partial_0 W] [\partial_0 W]
    + 2 g^{0i} [\partial_0 W] [\partial_i W]
    + 2 g^{ij} [\partial_i W] [\partial_j W]
    \\
    &= g^{00} H^2 - 2 g^{0i} H p_i + g^{ij} p_i p_j \\
    \intertext{%
        I will now insert all the $H$ and $\vec p$.
    }
    &= \frac{m^2}{f^2} \left\{ g^{00} \Group{g_{00} + g_{0i} v^i}^2 + 2 g^{0j}
    \Group{g_{00} + g_{0i} v_i}\Group{g_{0j} + g_{ij} v^i}
+ g^{ij} \Group{g_{0i} +
g_{ki} v^k} \Group{g_{0j} + g_{kj} v^k} \right\} \\
\intertext{%
    The terms in the big group have to equal $f^2$ so that the result really
    becomes $m^2$. I will expand everything and see what cancels.
}
&= \frac{m^2}{f^2} \left\{ g^{00} \Group{ [g_{00}]^2 + 2 g_{00} g_{0i} v^i + \Group{
g_{0i} v^i }^2 } \right.
+ 2 g^{0j} \Group{ g_{00} g_{0j} +
g_{00} g_{ij} v^i + g_{0j} g_{0i} v^i + g_{0i} v^i g_{kj} v^k }
\\ &\quad
\left. + g^{ij} \Group{ g_{0i} g_{0j} + g_{0i} g_{kj} v^k + g_{0j}
g_{ki} v^k + g_{ki} v^k g_{lj} v^l} \right\} \\
%
\intertext{%
    Indices can now be raised, and some combinations of $\tens g$ become $\tens
    \delta$.
}
&= \frac{m^2}{f^2} \Group{ g_{00} + 2 g_{0i} v^i + g^{00} \Group{ g_{0i} v^i }^2 
+ 2 g_{00} + g_{00} v^0 + 2 g_{0i} v^i + 2 g_{0i} v^i v^0
+ g_{00} + g_{k0} v^k + g_{k0} v^k + g_{kl} v^k v^l } \\
%
\intertext{%
    Using that $v^0 = 1$ and renaming the dummy indices, a lot of this can be
    written more compact:
}
&= \frac{m^2}{f^2} \Group{ 5 g_{00} + 8 g_{0i} v^i + g^{00} \Group{ g_{0i} v^i
}^2 + g_{kl} v^k v^l }
\end{align*}

That is getting kind of into the desired direction, but not close enough.

\subsection*{(c) Differential equation for $S(r)$}

$\tens g$ is now diagonal. That makes the previous differential equation much
easier.
\begin{align*}
    m^2
    &= \bracket{\tens \partial W, \tens \partial W} \\
    &= g^{00} \Group{ \dpd Wt }^2
    + g^{11} \Group{ \dpd Wr }^2
    + g^{22} \Group{ \dpd W\theta }^2
    + g^{33} \Group{ \dpd W\phi }^2 \\
    \intertext{%
        The separation ansatz \[ W(t, r, \theta, \phi) = - Et + S(r) + l
        \arccos\Call{y\call{\theta, \phi}} \] with \[ y(\theta, \phi) =
        \cos(\alpha) \cos(\theta) + \sin(\alpha) \sin(\theta) \sin(\phi) \] and
        $\alpha$ being a constant will give me chain rule contributions for the
        last two summands. I factor those out.
    }
    &= g^{00} E^2
    + g^{11} \Group{ \dpd Sr }^2
    + \Group{ \dpd Wy }^2 \Group { g^{22} \Group{ \dpd y\theta }^2 + g^{33} \Group{
\dpd y\phi }^2 } \\
\intertext{%
    The derivative of $\arccos$ is given as:
    \[
        \arccos'(x) = - \frac{1}{\sqrt{1-x^2}}.
    \]
    With that:
}
    m^2 &= g^{00} E^2
    + g^{11} \Group{ \dpd Sr }^2
    + \frac{l}{1-y^2} \Group { g^{22} \Group{ \dpd y\theta }^2 + g^{33} \Group{
\dpd y\phi }^2 } \\
m^2 &= \frac{1}{1-\frac{r_\text S}{r}} E^2 - \Group{ 1 - \frac{r_\text S}r }
\Group{ \dpd Sr }^2 + \frac{1}{r^2} \frac{l}{1-y^2} \Group { \Group{ \dpd
y\theta }^2 + \csc(\theta)^2 \Group{ \dpd y\phi }^2 } \\
m^2 r^2 &= \frac{r^2}{1-\frac{r_\text S}{r}} E^2 - r^2 \Group{ 1 - \frac{r_\text S}r }
\Group{ \dpd Sr }^2 - \frac{l}{1-y^2} \Group { \Group{ \dpd
y\theta }^2 + \csc(\theta)^2 \Group{ \dpd y\phi }^2 }
\end{align*}

Now I separate the radial and angular parts.
\[
  \frac{r^2}{1-\frac{r_\text S}{r}} E^2 - r^2 \Group{ 1 - \frac{r_\text S}r }
  \Group{ \dpd Sr }^2 - m^2 r^2 = \frac{l}{1-y^2} \Group { \Group{ \dpd y\theta
  }^2 + \csc(\theta)^2 \Group{ \dpd y\phi }^2 }
\]

Since both sides depend on different variables, they both have to equal a
constant, that I will name $c_1$.
\begin{align*}
  \frac{r^2}{1-\frac{r_\text S}{r}} E^2 - r^2 \Group{ 1 - \frac{r_\text S}r }
  \Group{ \dpd Sr }^2 - m^2 r^2 &= c_1 \\
  \intertext{%
      Now I can isolate $S$:
  }
  \Group{ \dpd Sr }^2 &= \frac{E^2}{\Group{1 - \frac{r_\text S}r}^2} -
  \frac{c_1}{r^2 \Group{1 - \frac{r_\text S}r}} - \frac{m^2}{1 - \frac{r_\text
  S}{r}}
  \intertext{%
      Taking the square root, integrating ...
  }
  S(r') &= \int_0^r \dif{r'} \, \sqrt{ \frac{E^2}{\Group{1 - \frac{r_\text S}r}^2} -
  \frac{c_1}{r^2 \Group{1 - \frac{r_\text S}r}} - \frac{m^2}{1 - \frac{r_\text
  S}{r}}}
\end{align*}

So the whole $W$ now looks like this:
\[
    W = - Et + \int_0^r \dif{r'} \, \sqrt{ \frac{E^2}{\Group{1 - \frac{r_\text S}r}^2} -
        \frac{c_1}{r^2 \Group{1 - \frac{r_\text S}r}} - \frac{m^2}{1 - \frac{r_\text
    S}{r}}} - l \arccos\del{y(\theta, \phi)}.
\]

\subsection*{(d) Trajectories}

I am supposed to calculate $P_2$. I do not even know which of my parameters is
$Q_2$. So my parameters are $E$, $c_1$ and I think that the $l$ is yet another
one of those parameters. But it could also be $\alpha$. One of those parameters
is meaningless, and I should only get 3. That is what I read in
\parencite{Kuypers/Mechanik}.

So I will just start with $l$. That might be $Q_3$ or so.
\begin{align*}
    \dpd Wl = \arccos(y) &= -P_3 \\
    y &= \cos(-P_3) \\
    y &= \cos(P_3) \\
    \cos(\alpha) \cos(\theta) + \sin(\alpha) \sin(\theta) \sin(\phi) &= \cos(P_3) \\
    \intertext{%
        I can now separate $\theta$ and $\phi$ to either side of the equation.
    }
    \sin(\phi) &= \cos(P_3) \sin(\alpha) \sin(\theta) - \cot(\alpha)
    \cot(\theta)
\end{align*}

That gives me a relationship between $\theta$ and $\phi$, but nothing with
respect to $r$.

Then I tried to use $c_1$ as the $Q_2$. The problem is that $c_1$ does not
occur in the angular term. So this will give me only a relationship from $r$
with $t$ at best.
\begin{align*}
    \dpd W{{c_1}} &= -P_2 \\
    - \int \dif r\, \frac{1}{\sqrt{ \frac{E^2}{\Group{1 - \frac{r_\text S}r}^2} -
        \frac{c_1}{r^2 \Group{1 - \frac{r_\text S}r}} - \frac{m^2}{1 - \frac{r_\text
    S}{r}}}} \frac{1}{r^2 \sbr{1 - \frac{r_\text S}{r}}} &= -P_2
\end{align*}

Even if I could solve this with respect to $r$, there would not be any angular
part.

\IfFileExists{\bibliographyfile}{
    \printbibliography
}{}

\end{document}

% vim: spell spelllang=en tw=79

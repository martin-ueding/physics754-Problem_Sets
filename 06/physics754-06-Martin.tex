\input{../header.tex}

\newcommand\problemset{6}

\hypersetup{
    pdftitle={physics754 Problem Set \problemset}
}

\newenvironment{aside}{\itshape\small}{}

\let\Call\del
\let\Group\sbr

\newcommand\call[1]{(#1)}
\newcommand\group[1]{(#1)}

%\subject{}
\title{physics754 -- Problem Set \problemset}
%\subtitle{}
\author{
    Martin Ueding \\ \small{\href{mailto:mu@martin-ueding.de}{mu@martin-ueding.de}}
}
\publishers{Group 5 -- Olaf Baake}

\begin{document}

\maketitle

\section*{H.9: Schwarzschild trajectories given by the Jacobi method}

I will omit the subscript $q$ since there are no other velocities. Vectors in
$\R^3$ are typeset in bold serif italics.

\subsection*{(a) Canonical momenta}

The canonical momenta are:
\[
    p_i = \dpd L{{v^i}} = - m \frac{1}{2f(\vec q, \vec v, q^0)} \Group{
    2 g_{0i}(q^0, \vec q) + 2 g_{ij}(q^0, \vec q) v^j}.
\]

I will assemble the hamiltonian from that. Since the function arguments to do
change within the following calculations, I will just omit them. $f$ and $g$
are still functions.
\begin{align*}
    H(\vec q, \vec p)
    &= \Group{p_i v^i - L}_{v = v(\vec q, \vec p)} \\
    %
    &= - \Group{\frac{m}{2f} \sbr{ 2 g_{0i} +
    2 g_{ij} v^j} v^i - L}_{\vec v = \vec v(\vec q, \vec p)} \\
    %
    &= - \Group{\frac{m}{f} \sbr{ g_{0i} +
    g_{ij} v^j} v^i - L}_{\vec v = \vec v(\vec q, \vec p)} \\
    %
    &= - \Group{\frac{m}{f} \Group{ g_{0i}
    v^i + g_{ij} v^i v^j - g_{00} - 2 g_{0i} v^i - g_{ik} v^i
    v^k}}_{\vec v = \vec v(\vec q, \vec p)} \\
    %
    &= - \Group{\frac{m}{f} \Group{ - g_{0i}
    v^i + g_{ij} v^i v^j - g_{00} }}_{\vec v = \vec v(\vec q, \vec
    p)} \\
    %
    &= \Group{\frac{m}{f} \Group{ g_{00} + g_{0i} v^i}}_{\vec v = \vec v(\vec q, \vec p)}
\end{align*}

And that is the result from the problem set.

\subsection*{(b) Differential equation for $W$}

\begin{align*}
    m^2
    &\overset != 
    \bracket{\tens \partial W, \tens \partial W} \\
    %
    &= g^{00} [\partial_0 W] [\partial_0 W]
    + 2 g^{0i} [\partial_0 W] [\partial_i W]
    + 2 g^{ij} [\partial_i W] [\partial_j W]
    \\
    &= g^{00} H^2 - 2 g^{0i} H p_i + g^{ij} p_i p_j \\
    \intertext{%
        I will now insert all the $H$ and $\vec p$.
    }
    &= \frac{m^2}{f^2} \left\{ g^{00} \Group{g_{00} + g_{0i} v^i}^2 + 2 g^{0j}
    \Group{g_{00} + g_{0i} v_i}\Group{g_{0j} + g_{ij} v^i}
+ g^{ij} \Group{g_{0i} +
g_{ki} v^k} \Group{g_{0j} + g_{kj} v^k} \right\} \\
\intertext{%
    The terms in the big group have to equal $f^2$ so that the result really
    becomes $m^2$. I will expand everything and see what cancels.
}
&= \frac{m^2}{f^2} \left\{ g^{00} \Group{ [g_{00}]^2 + 2 g_{00} g_{0i} v^i + \Group{
g_{0i} v^i }^2 } \right.
+ 2 g^{0j} \Group{ g_{00} g_{0j} +
g_{00} g_{ij} v^i + g_{0j} g_{0i} v^i + g_{0i} v^i g_{kj} v^k }
\\ &\quad
\left. + g^{ij} \Group{ g_{0i} g_{0j} + g_{0i} g_{kj} v^k + g_{0j}
g_{ki} v^k + g_{ki} v^k g_{lj} v^l} \right\} \\
%
\intertext{%
    Indices can now be raised, and some combinations of $\tens g$ become $\tens
    \delta$.
}
&= \frac{m^2}{f^2} \Group{ g_{00} + 2 g_{0i} v^i + g^{00} \Group{ g_{0i} v^i }^2 
+ 2 g_{00} + g_{00} v^0 + 2 g_{0i} v^i + 2 g_{0i} v^i v^0
+ g_{00} + g_{k0} v^k + g_{k0} v^k + g_{kl} v^k v^l } \\
%
\intertext{%
    Using that $v^0 = 1$ and renaming the dummy indices, a lot of this can be
    written more compact:
}
&= \frac{m^2}{f^2} \Group{ 5 g_{00} + 8 g_{0i} v^i + g^{00} \Group{ g_{0i} v^i
}^2 + g_{kl} v^k v^l }
\end{align*}

That is getting kind of into the desired direction, but not close enough.

\subsection*{(c) Differential equation for $S(r)$}

$\tens g$ is now diagonal. That makes the previous differential equation much
easier.
\begin{align*}
    m^2
    &= \bracket{\tens \partial W, \tens \partial W} \\
    &= g^{00} \Group{ \dpd Wt }^2
    + g^{11} \Group{ \dpd Wr }^2
    + g^{22} \Group{ \dpd W\theta }^2
    + g^{33} \Group{ \dpd W\phi }^2 \\
    \intertext{%
        The separation ansatz \[ W(t, r, \theta, \phi) = - Et + S(r) + l
        \arccos\Call{y\call{\theta, \phi}} \] with \[ y(\theta, \phi) =
        \cos(\alpha) \cos(\theta) + \sin(\alpha) \sin(\theta) \sin(\phi) \] and
        $\alpha$ being a constant will give me chain rule contributions for the
        last two summands. I factor those out.
    }
    &= g^{00} E^2
    + g^{11} \Group{ \dpd Sr }^2
    + \Group{ \dpd Wy }^2 \Group { g^{22} \Group{ \dpd y\theta }^2 + g^{33} \Group{
\dpd y\phi }^2 } \\
\intertext{%
    The derivative of $\arccos$ is given as:
    \[
        \arccos'(x) = - \frac{1}{\sqrt{1-x^2}}.
    \]
    With that:
}
    &= g^{00} E^2
    + g^{11} \Group{ \dpd Sr }^2
    + \frac{l}{1-y^2} \Group { g^{22} \Group{ \dpd y\theta }^2 + g^{33} \Group{
\dpd y\phi }^2 } \\
\intertext{%
    The problems asks for a differential equation for $S(r)$. This can be
    obtained by isolating $S$:
}
    \Group{ \dpd Sr }^2
    &= g_{11} \Group{ m^2 - g^{00} E^2
    - \frac{l}{1-y^2} \Group { g^{22} \Group{ \dpd y\theta }^2 + g^{33} \Group{
\dpd y\phi }^2 } } \\
\intertext{%
    The last step would be to take the square root and perform $\int_0^r \dif
    r'$. However, the current representation still looks like it could be
    simplified. I will now insert most of the known terms, which will make the
    formula explode, again.
}
&= - \frac{1}{1 - \frac{r_\text S}{r}} \left\{ m^2 - \Group{1 - \frac{r_S}{r}} E^2
- \frac{l}{1-y^2} \Group { -r^2 \Group{ \dpd y\theta }^2 -r^2 \sin(\theta)^2 \Group{
\dpd y\phi }^2 } \right\} \\
\intertext{Factor out $r^2$.}
&= - \frac{1}{1 - \frac{r_\text S}{r}} \left\{ m^2 - \Group{1 - \frac{r_S}{r}} E^2
+ \frac{r^2}{1-y^2} \Group { \Group{ \dpd y\theta }^2 + \sin(\theta)^2 \Group{
\dpd y\phi }^2 } \right\} \\
%
\intertext{Expand the derivatives.}
&= - \frac{1}{1 - \frac{r_\text S}{r}} \left\{ m^2 - \Group{1 - \frac{r_S}{r}}
E^2 + \frac{r^2}{1-y^2} \bigg[ \right.
    \\&\quad
    \left. \Group{ -\cos(\alpha) \sin(\theta) + \sin(\alpha) \cos(\theta)
    \sin(\phi) }^2 + \sin(\theta)^2
    \Group{ \sin(\alpha) \sin(\theta) \sin(\phi) }^2
\bigg] \right\} \\
\end{align*}

\IfFileExists{\bibliographyfile}{
    \printbibliography
}{}

\end{document}

% vim: spell spelllang=en tw=79

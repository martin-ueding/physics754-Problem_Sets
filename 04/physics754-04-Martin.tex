\input{../header.tex}

\newcommand\problemset{4}

\hypersetup{
    pdftitle={physics754 Problem Set \problemset}
}

\newenvironment{aside}{\itshape\small}{}

%\subject{}
\title{physics754 -- Problem Set \problemset}
%\subtitle{}
\author{
    Martin Ueding \\ \small{\href{mailto:mu@martin-ueding.de}{mu@martin-ueding.de}}
}
\publishers{Group 5 -- Olaf Baake}

\begin{document}

\maketitle

\section{Diffeomorphisms and vector fields}

The various uses of $y$ are not defined in the problem set, so I will do that
now to avoid confusion:

\begin{itemize}
    \item
        $X \colon \R^4 \mapsto \R^4$ is a vector field.
    \item
        $y \colon \R \mapsto \R^4$ is a trajectory.
    \item
        $f_\tau \colon \R^4 \mapsto \R^4$ is a diffeomorphism.
    \item
        $f \colon \R^4 \times \R \mapsto \R^4$ is a one parameter family of
        diffeomorphisms.
    \item
        $\tilde y \colon \R^4 \times \R \mapsto \R^4$ is identical to
        $f_\tau$. This is the use of some of the $y$ in the problem set.
    \item
        $\hat y \in \R^4$ is a vector in the image of $\tilde y$.
    \item
        $x \in \R^4$ is a vector in the domain of $\tilde y$.
    \item
        $\tau \in \R$ is a scalar and parameter for $y$.
\end{itemize}

The ambiguity of round parentheses arising because they are used for both
scoping and function application makes the equation a bit hard to read. I will
use a Mathematica inspired notation that does not contradict the usual
notation: Round parentheses are used only for function application, square
brackets are used for scoping and grouping.

I can now safely omit using different fonts for scalars, vectors and tensors
of higher rank since every symbol is defined.

\subsection{}

With all that set, the first problem nearly does itself:
\begin{align*}
    X(\hat y)
    &= \sbr{\dod{}\tau f_\tau} \del{f_\tau^{-1}(\hat y)} \\
    &= \sbr{\dod{}\tau \tilde y} \del{\tilde y^{-1}(\hat y)} \\
    \intertext{%
        $y$ is a solution of (12) from the problem set. $\tilde y$ additinally
        solves (13) as well. So $\tilde y$ will solve (12). So
        \[
            \sbr{\dod{}\tau \tilde y} (x, \tau) = [X \circ \tilde y](x, \tau)
        \]
        and therefore
    }
    &= [X \circ \tilde y] \del{\tilde y^{-1}(\hat y)} \\
    &= [X \circ \tilde y \circ \tilde y^{-1}] (\hat y) \\
    &= X(\hat y) \\
\end{align*}


\IfFileExists{\bibliographyfile}{
    \printbibliography
}{}

\end{document}

% vim: spell spelllang=en tw=79

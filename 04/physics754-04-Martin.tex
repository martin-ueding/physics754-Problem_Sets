\input{../header.tex}

\newcommand\problemset{4}

\hypersetup{
    pdftitle={physics754 Problem Set \problemset}
}

\newenvironment{aside}{\itshape\small}{}

%\subject{}
\title{physics754 -- Problem Set \problemset}
%\subtitle{}
\author{
    Martin Ueding \\ \small{\href{mailto:mu@martin-ueding.de}{mu@martin-ueding.de}}
}
\publishers{Group 5 -- Olaf Baake}

\begin{document}

\maketitle

\section{Diffeomorphisms and vector fields}

The various uses of $y$ are not defined in the problem set, so I will do that
now to avoid confusion:

\begin{itemize}
    \item
        $X \colon \R^4 \mapsto \R^4$ is a vector field.
    \item
        $y \colon \R \mapsto \R^4$ is a trajectory.
    \item
        $f_\tau \colon \R^4 \mapsto \R^4$ is a diffeomorphism.
    \item
        $f \colon \R^4 \times \R \mapsto \R^4$ is a $\R$-parameter family of
        diffeomorphisms.
    \item
        $\tilde y \colon \R \times \R^4 \mapsto \R^4$ is a $\R^4$-parameter
        family of trajectories starting at $x$. This is used by $f$ to
        create a $\R$-parameter family of diffeomorphisms.
    \item
        $\hat y \in \R^4$ is a vector in the image of $\tilde y$.
    \item
        $x \in \R^4$ is a vector in the domain of $\tilde y$, which is the
        same as the image, since $y(0) = x$.
    \item
        $\tau \in \R$ is a scalar and parameter for $y$.
\end{itemize}

The ambiguity of round parentheses arising because they are used for both
scoping and function application makes the equation a bit hard to read. I will
use a Mathematica inspired notation that does not contradict the usual
notation: Round parentheses are used only for function application, square
brackets are used for scoping and grouping.

I can now safely omit using different fonts for scalars, vectors and tensors
of higher rank since every symbol is defined.

\subsection{}

With all that set, the first problem nearly does itself:
\begin{align*}
    X(\hat y)
    &= \sbr{\dod{}\tau f_\tau} \del{f_\tau^{-1}(\hat y)} \\
    &= \sbr{\dod{}\tau \tilde y} \del{\tilde y^{-1}(\hat y)} \\
    \intertext{%
        $y$ is a solution of (12) from the problem set. $\tilde y$ additionally
        solves (13) as well. So $\tilde y$ will solve (12). So
        \[
            \sbr{\dod{}\tau \tilde y} (x, \tau) = [X \circ \tilde y](x, \tau)
        \]
        and therefore
    }
    &= [X \circ \tilde y] \del{\tilde y^{-1}(\hat y)} \\
    &= [X \circ \tilde y \circ \tilde y^{-1}] (\hat y) \\
    &= X(\hat y)
\end{align*}

$f_0(x)$ is just $\tilde y(x, 0)$. Since $y(0) = x$, $\tilde y(x, 0)$ reduces
to $x$. Therefore, $f_0$ is indeed the identity transformation. That shows
$f_0 = \mathbb 1$.

\subsection{}

Show that for any $g$:
\[
    \eval{\dod{}\tau f_\tau^* g_{\mu\nu}}_{\tau=0} = 2 X_{(\mu;\nu)},
\]
where I have used the symmetrization parentheses.

The right hand side is given by:
\[
    2 X_{(\mu;\nu)} = 2 X_{(\mu,\nu)} - 2 \Gamma^\lambda_{(\mu\nu)} X_\lambda
\]

Now I will turn to the left hand side, where I will evaluate it at $x$.
\begin{align*}
    \text{RHS}
    &= \eval{\dod{}\tau f_\tau^* g_{\mu\nu}}_{\tau=0} (x) \\
    \intertext{%
        Now I will apply the transformation to $g$.
    }
    &= \eval{\dod{}\tau
    g_{\alpha\beta}\del{f_\tau(x)} \; f_\tau^\alpha{}_{,\mu}(x) \;
    f_\tau^\beta{}_{,\nu}(x)
    }_{\tau=0} \\
    \intertext{%
        The lower index $\tau$ is a little messy, so I will expand $f$ in terms
        of $\tilde y$, introducing even more explicit parameters.
    }
    &= \eval{\dod{}\tau
    g_{\alpha\beta}\del{\tilde y(x, \tau)} \; \tilde y^\alpha_{,\mu}(x, \tau) \;
    \tilde y^\beta_{,\nu}(x, \tau)
    }_{\tau=0} \\
    \intertext{%
        Total differentiation will act on all three factors, since all of them
        depend on $\tau$. This gives three terms:
    }
    &= 
    \sbr{
        \dod{}\tau
        g_{\alpha\beta}\del{\tilde y(x, \tau)}
    }
    \; \tilde y^\alpha_{,\mu}(x, \tau)
    \; \tilde y^\beta_{,\nu}(x, \tau)
    +
    g_{\alpha\beta}\del{\tilde y(x, \tau)}
    \;
    \sbr{
        \dod{}\tau
        \tilde y^\alpha_{,\mu}(x, \tau)
    }
    \; \tilde y^\beta_{,\nu}(x, \tau)
    \\ &\quad
    +
    g_{\alpha\beta}\del{\tilde y(x, \tau)}
    \; \tilde y^\alpha_{,\mu}(x, \tau)
    \;
    \sbr{
        \dod{}\tau
        \tilde y^\beta_{,\nu}(x, \tau)
    }
    \Bigg|_{\tau=0} \\
    \intertext{%
        To simplify, I will now apply the $\tau = 0$ to all the terms that make
        it possible at this stage. $\tilde y(x, 0)$ is just $y(0)$, which was
        defined to be $x$. This simplification will be in the $g(\ldots)$. The
        other thing I can do right now is to look at $\tilde y^\alpha_{,\mu}(x,
        \tau)$: When $\tau = 0$, $\tilde y$ reduces to $x$, so that will be a
        $\delta^\alpha_\mu$.
    }
    &=
    \sbr{
        \dod{}\tau
        g_{\alpha\beta}\del{\tilde y(x, \tau)}
    } \Bigg|_{\tau=0}
    \; \delta^\alpha_\mu
    \; \delta^\beta_\nu
    +
    g_{\alpha\beta}(x)
    \;
    \sbr{
        \dod{}\tau
        \tilde y^\alpha_{,\mu}(x, \tau)
    } \Bigg|_{\tau=0}
    \; \delta^\beta_\nu
    \\ &\quad
    +
    g_{\alpha\beta}(x)
    \; \delta^\alpha_\mu
    \;
    \sbr{
        \dod{}\tau
        \tilde y^\beta_{,\nu}(x, \tau)
    } \Bigg|_{\tau=0}
    \\
\end{align*}

\IfFileExists{\bibliographyfile}{
    \printbibliography
}{}

\end{document}

% vim: spell spelllang=en tw=79

\input{../header.tex}

\newcommand\problemset{8}

\hypersetup{
    pdftitle={physics754 Problem Set \problemset}
}

\newenvironment{aside}{\itshape\small}{}

%\subject{}
\title{physics754 -- Problem Set \problemset}
%\subtitle{}
\author{
    Martin Ueding \\ \small{\href{mailto:mu@martin-ueding.de}{mu@martin-ueding.de}}
}
\publishers{Group 5 -- Olaf Baake}

\begin{document}

\maketitle

\section*{H.11: Matchine the dust cloud to the exterior Schwarzschild metric}

\subsection*{(b) Equal at certain point}

The metric $\tilde{\tens g}$ is defined by:
\[
    \tilde g_{00} = B
    \eqnsep
    \tilde g_{11} = - A
    \eqnsep
    \tilde g_{22} = - r^2 f(t)^2
    \eqnsep
    \tilde g_{33} = - r^2 f(t)^2 \sin(\theta)^2
\]

\subsubsection*{Computing the jacobian}

The derivatives of the zeroth coordinate are the most complicated. In order to
take the derivative of the integral, I will use the following formula. Let
\[
    f(t) := \int_{g_1(t)}^{g_2(t)} \dif \tau \, h(t, \tau).
\]
Then the derivative is given by
\[
    f'(t) = g_2'(t) h\del{t, g_2(t)} - g_1'(t) h\del{t, g_1(t)} +
    \int_{g_1(t)}^{g_2(t)} \dif \tau \, \dpd h t(t, \tau).
\]

\newcommand\WW{\sqrt{\frac{1-ar^2}{1-ar_0^2}}}

Since I will need them later on, I will first write down the derivatives of
$S$:
\[
    \dpd St = \WW f'(t)
    \eqnsep
    \dpd Sr = - \frac{ar}\WW \sbr{1-f(t)}
\]

For the derivatives of the zeroth coordinate, I came up with:
\[
    \dpd{{y^0}}{{x^0}} = \sqrt{\frac{1-ar_0^2}a} \WW f'(t)
    \frac{\sqrt{\frac{S}{1-S}}}{1-\frac{ar_0^2}S}
\]
and
\[
    \dpd{{y^0}}{{x^1}} = \sqrt{\frac{1-ar_0^2}a} \frac{ar}\WW \sbr{1-f(t)}
    \frac{\sqrt{\frac{S}{1-S}}}{1-\frac{ar_0^2}S},
\]
where $S$ is just a sloppy shorthand for $S(t, r)$.


The non-vanishing derivatives of the first coordinate are given as:
\[
    \dpd{{y^1}}{{x^1}} = f(t)
    \eqnsep
    \dpd{{y^1}}{{x^0}} = r f'(t)
\]

The last two elements of the jacobian are equal since the coordinates did not
change:
\[
    \dpd{{y^2}}{{x^2}} = 1
    \eqnsep
    \dpd{{y^3}}{{x^3}} = 1.
\]

All other components are zero.

\subsubsection*{Computing the transformed metric}

Since $\tilde{\tens g}$ is diagonal, and most components of the jacobian are
zero, the transformations only consist of two summands at most.

\begin{align*}
    g_{00}
    &= \tilde g_{\mu\nu} \dpd{{y^\mu}}{{x^0}} \dpd{{y^\nu}}{{x^0}}
    \intertext{%
        Since the metric $\tilde{\tens g}$ is diagonal, only $\mu = \nu$ equals
        non-vanishing components. Therefore, I write it like this, still having
        a single summation over $\mu$ implicitly.
    }
    &= \tilde g_{\mu\mu} \sbr{\dpd{{y^\mu}}{{x^0}}}^2
    \intertext{%
        Non-vanishing contributions are given by $\mu = 0$ and $\mu = 1$. I
        write those down explicitly:
    }
    &= - B \sbr{\sqrt{\frac{1-ar^2}a} f'(t)
    \frac{\sqrt{\frac{S}{1-S}}}{1-\frac{ar_0^2}S}}^2 - A \sbr{r f'(t)}^2 \\
    &= - B \frac{1-ar^2}a f'(t)^2
    \frac{\frac{S}{1-S}}{\sbr{1-\frac{ar_0^2}S}^2} - A r^2 f'(t)^2 \\
\end{align*}

Due to the identity of $x^2$ and $x^3$ and the diagonal metric, those entries
transform identically:
\[
    g_{22} = \tilde g_{22}
    \eqnsep
    g_{33} = \tilde g_{33}.
\]

% TODO What happens to g_{01} for instance?

\IfFileExists{\bibliographyfile}{
    \printbibliography
}{}

\end{document}

% vim: spell spelllang=en tw=79

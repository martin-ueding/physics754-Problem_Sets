\input{../header.tex}

\newcommand\problemset{8}

\hypersetup{
    pdftitle={physics754 Problem Set \problemset}
}

\newenvironment{aside}{\itshape\small}{}

\newcommand\tS{t_\text S}

%\subject{}
\title{physics754 -- Problem Set \problemset}
%\subtitle{}
\author{
    Martin Ueding \\ \small{\href{mailto:mu@martin-ueding.de}{mu@martin-ueding.de}}
}
\publishers{Group 5 -- Olaf Baake}

\begin{document}

\maketitle

\section*{H.11: Matchine the dust cloud to the exterior Schwarzschild metric}

\subsection*{(a) Equal at certain point}

At $r_0$, $S$ is simply
\[
    S(t, r_0) = f(t).
\]
Then $A$ is
\[
    A(t, r_0) = \sbr{1 - \frac{a r_0^2}{f(t)}}^{-1}.
\]
In $B$, the first two term become 1, which leaves the last term. There, the
numerator is just the square of the denominator, leaving
\[
    B(t, r_0) = 1 - \frac{ar_0^2}{f(t)}
\]
as a final result.

From that, $f$ is expressed with $\bar r$:
\[
    f(t) = \frac{\bar r(t, r_0)}{r_0}.
\]
I insert this into $A$ and $B$:
\[
    A(t, r_0) = \sbr{1 - \frac{a r_0^3}{\bar r(t, r_0)}}^{-1}
    \eqnsep
    B(t, r_0) = 1 - \frac{ar_0^3}{\bar r(t, r_0)}.
\]

Since $r_g = a r_0^3$ is a bijective relation, the following really does imply
an “iff” like required in the problem. So iff $r_g = a r_0^3$ holds, then
\[
    A(t, r_0) = \sbr{1 - \frac{r_g}{\bar r(t, r_0)}}^{-1}
    \eqnsep
    B(t, r_0) = 1 - \frac{r_g}{\bar r(t, r_0)}.
\]

Using all the definitions, I can get
\[
    M = \frac 43 \piup \rho(0) r_0^3.
\]
The mass $M$ is the mass of a sphere with radius $r_0$ and average density
$\rho(0)$. This mass that generates the gravitational field at the $t = 0$ is
distributed equally within that sphere.

\subsection*{(b) Transformation}

The metric $\tilde{\tens g}$ is defined by:
\[
    \tilde g_{00} = B
    \eqnsep
    \tilde g_{11} = - A
    \eqnsep
    \tilde g_{22} = - r^2 f(t)^2
    \eqnsep
    \tilde g_{33} = - r^2 f(t)^2 \sin(\theta)^2
\]

\subsubsection*{Computing the jacobian}

The derivatives of the zeroth coordinate are the most complicated. In order to
take the derivative of the integral, I will use the following formula. Let
\[
    f(t) := \int_{g_1(t)}^{g_2(t)} \dif \tau \, h(t, \tau).
\]
Then the derivative is given by
\[
    f'(t) = g_2'(t) h\del{t, g_2(t)} - g_1'(t) h\del{t, g_1(t)} +
    \int_{g_1(t)}^{g_2(t)} \dif \tau \, \dpd h t(t, \tau).
\]

\newcommand\WW{\sqrt{\frac{1-ar^2}{1-ar_0^2}}}

Since I will need them later on, I will first write down the derivatives of
$S$:
\[
    \dpd St = \WW f'(t)
    \eqnsep
    \dpd Sr = - \frac{ar}\WW \sbr{1-f(t)}
\]

For the derivatives of the zeroth coordinate, I came up with:
\[
    \dpd{{y^0}}{{x^0}} = \sqrt{\frac{1-ar_0^2}a} \WW f'(t)
    \frac{\sqrt{\frac{S}{1-S}}}{1-\frac{ar_0^2}S}
\]
and
\[
    \dpd{{y^0}}{{x^1}} = \sqrt{\frac{1-ar_0^2}a} \frac{ar}\WW \sbr{1-f(t)}
    \frac{\sqrt{\frac{S}{1-S}}}{1-\frac{ar_0^2}S},
\]
where $S$ is just a sloppy shorthand for $S(t, r)$.


The non-vanishing derivatives of the first coordinate are given as:
\[
    \dpd{{y^1}}{{x^1}} = f(t)
    \eqnsep
    \dpd{{y^1}}{{x^0}} = r f'(t)
\]

The last two elements of the jacobian are equal since the coordinates did not
change:
\[
    \dpd{{y^2}}{{x^2}} = 1
    \eqnsep
    \dpd{{y^3}}{{x^3}} = 1.
\]

All other components are zero.

\subsubsection*{Computing the transformed metric}

Since $\tilde{\tens g}$ is diagonal, and most components of the jacobian are
zero, the transformations only consist of two summands at most.

\begin{align*}
    g_{00}
    &= \tilde g_{\mu\nu} \dpd{{y^\mu}}{{x^0}} \dpd{{y^\nu}}{{x^0}}
    \intertext{%
        Since the metric $\tilde{\tens g}$ is diagonal, only $\mu = \nu$ equals
        non-vanishing components. Therefore, I write it like this, still having
        a single summation over $\mu$ implicitly.
    }
    &= \tilde g_{\mu\mu} \sbr{\dpd{{y^\mu}}{{x^0}}}^2
    \intertext{%
        Non-vanishing contributions are given by $\mu = 0$ and $\mu = 1$. I
        write those down explicitly:
    }
    &= B \sbr{\sqrt{\frac{1-ar^2}a} f'(t)
    \frac{\sqrt{\frac{S}{1-S}}}{1-\frac{ar_0^2}S}}^2 - A \sbr{r f'(t)}^2 \\
    \intertext{%
        Simplifying …
    }
    &= B \frac{1-ar^2}a f'(t)^2
    \frac{\frac{S}{1-S}}{\sbr{1-\frac{ar_0^2}S}^2} - A r^2 f'(t)^2 \\
    \intertext{%
        Now I use the derivative of $f$ that is given by $f'(t)^2 = a
        [f(t)^{-1} - 1]$.
    }
    &= B \sbr{1-ar^2} \sbr{f(t)^{-1}-1}
    \frac{\frac{S}{1-S}}{\sbr{1-\frac{ar_0^2}S}^2} - A r^2 f'(t)^2 \\
    \intertext{%
        I now insert $B$.
    }
    &= \frac{f(t)}{S} \WW \sbr{1 - \frac{ar_0^2}S}^2 A \sbr{1-ar^2}
    \sbr{f(t)^{-1}-1} \frac{\frac{S}{1-S}}{\sbr{1-\frac{ar_0^2}S}^2} - A r^2
    f'(t)^2 \\
    \intertext{%
        I cancel $S$ and the very last denominator in the first summand.
    }
    &= f(t) \WW A \sbr{1-ar^2} \sbr{f(t)^{-1}-1} \frac{1}{1-S} - A r^2 f'(t)^2
    \\
    &= \WW A \sbr{1-ar^2} \sbr{1 - f(t)} \frac{1}{1-S} - A r^2 f'(t)^2
    \\
    \intertext{%
        The $1 - S$ is just
        \[
            \WW \sbr{1 - f(t)}.
        \]
        This can be cancelled with the other factors.
    }
    &= A \sbr{1-ar^2} - A r^2 f'(t)^2 \\
    &= A - A ar^2 - A r^2 a \sbr{f(t)^{-1} - 1} \\
    &= A - A ar^2 - A r^2 a f(t)^{-1} + A r^2 a \\
    &= A - A r^2 a f(t)^{-1} \\
    &= A \sbr{1 - r^2 a f(t)^{-1}} \\
    \intertext{%
        Using the definition of $A$, this is just
    }
    &= 1.
\end{align*}
This is the desired result.

Now I will turn to the first diagonal entry:
\begin{align*}
    g_{11}
    &= \tilde g_{\mu\mu} \sbr{\dpd{{y^\mu}}{{x^1}}}^2 \\
    \intertext{%
        Here again, the interesting components are $\mu = 0$ and $\mu = 1$.
        Expanding all those directly gives:
    }
    &= B \frac{\sbr{1 - ar_0^2}^2}{1 - ar^2} ar^2 \sbr{1 - f(t)}^2 - A f(t)^2
    \\
    \intertext{%
        Expanding $B$ again.
    }
    &= \frac{f(t)^2}S \WW \sbr{1 - \frac{ar_0^2}S}^2 A \frac{\sbr{1 -
    ar_0^2}^2}{1 - ar^2} ar^2 \sbr{1 - f(t)}^2 - A f(t)^2 \\
    \intertext{%
        I can factor out $f(t)^2$ and $A$.
    }
    &= f(t)^2 A \sbr{ \frac{1}S \WW \sbr{1 - \frac{ar_0^2}S}^2 \frac{\sbr{1 -
    ar_0^2}^2}{1 - ar^2} ar^2 \sbr{1 - f(t)}^2 - 1} \\
    \intertext{%
        I can factor out a $1 - ar^2$ since I will need it for the desired
        result.
    }
    &= \frac{f(t)^2}{1 - ar^2} A \sbr{ \frac{1}S \WW \sbr{1 - \frac{ar_0^2}S}^2
    \sbr{1 - ar_0^2}^2 ar^2 \sbr{1 - f(t)}^2 - \sbr{1 - ar^2}} \\
    \intertext{%
        $A$ times the bracket has to give $-1$ for the whole thing to come out
        right.
    }
\end{align*}

Due to the identity of $x^2$ and $x^3$ and the diagonal metric, those entries
transform identically:
\[
    g_{22} = \tilde g_{22}
    \eqnsep
    g_{33} = \tilde g_{33}.
\]

% TODO What happens to g_{01} for instance?

\subsection*{(c) Limit and domain}

\begin{align*}
    \bar t(r_0, \tS)
    &= \sqrt{\frac{1-ar_0^2}a} \int_{S(r_0, \tS)}^1 \dif \rho \,
    \frac{\sqrt{\frac\rho{1-\rho}}}{1-\frac{ar_0^2}\rho} \\
    \intertext{%
        At $r_0$, $S$ is just $f(t)$.
    }
    &= \sqrt{\frac{1-ar_0^2}a} \int_{f(t)}^1 \dif \rho \,
    \frac{\sqrt{\frac\rho{1-\rho}}}{1-\frac{ar_0^2}\rho} \\
    \intertext{%
        From $r_0 f(\tS) = r_g$ follows that $f(\tS) = ar_0^2$. The integrand
        is singular at the lower bound of integration. I can expand the
        denominator into a Laurent series
        \[
            \frac{1}{1 - \frac cz} = \sum_{k = -\infty}^0 c^{-k} z^k
        \]
        using the geometric series. Since the negative part of the series
        extends until negative infinity, this means that this is a singularity
        of the most series kind. The singularity at $\rho = 1$ does not seem to
        be as bad as the one as $\rho = ar_0^2$, so I think that this one does
        not have an impact on the convergence of the integral. The first one
        has, so this integral does not converge. Therefore, as a physicist, I
        can write
    }
    &= \infty.
\end{align*}

If $\rho > f(\tS)$, this singularity does not occur. If $t < \tS$, this should
be given. The restrictions on the map are therefore
\[
    t < \tS
\]
so far.

Also $f(t)$ must not be greater than $1$ so that the integral does not return
negative values. Therefore $f(t) \leq 1$. Applying the inverse $f$ which is
bijective and strictly monotonically decreasing on the domain of interest, this
gives $t \geq 0$. The restrictions are now:
\[
    0 \leq t < \tS
\]


\IfFileExists{\bibliographyfile}{
    \printbibliography
}{}

\end{document}

% vim: spell spelllang=en tw=79

\input{../header.tex}

\newcommand\problemset{2}

\hypersetup{
    pdftitle={physics754 Problem Set \problemset}
}

\newenvironment{aside}{\itshape\small}{}

%\subject{}
\title{physics754 -- Problem Set \problemset}
%\subtitle{}
\author{
    Martin Ueding \\ \small{\href{mailto:mu@martin-ueding.de}{mu@martin-ueding.de}}
}
\publishers{Group 5}

\begin{document}

\maketitle

\begin{aside}
    When writing down tensor components, I have seen
    $R_{\alpha\beta\gamma}^\delta$ as well as $R_{\alpha\beta\gamma}{}^\delta$
    and maybe $R^\delta{}_{\alpha\beta\gamma}$.
    Does that make a difference? If you raise and lower indices and want to
    keep their position fixed, this “splitting up” the lower and upper indices
    seems to clarify. But do we need this really?
\end{aside}

\section{Transformation of the Christoffel-Symbols}

\subsection{Compute $\Gamma^\alpha_{\beta\gamma}(\varphi^*g)$}

I assume that this means that I should compose $\Gamma$ with the transformed
$g$.

The transformed $g$ is given by:
\[
    (\varphi^* g)_{\mu\nu}(x)
    = g_{\alpha\beta}(\varphi(x)) \dpd{{\varphi^\alpha}}{{x^\mu}}
    \dpd{{\varphi^\beta}}{{x^\nu}}
    = g_{\alpha\beta}(\varphi(x)) \varphi^\alpha_{,\mu}(x)
    \varphi^\beta_{,\nu}(x).
\]

The transformed dual\footnote{Is “dual” correct?} $g$ is similarly obtained:
\[
    (\varphi^* g)^{\mu\nu} = g^{\mu\nu}(\varphi(x))
    \dpd{{x^\gamma}}{{\varphi^\mu}} \dpd{{x^\delta}}{{\varphi^\nu}}.
\]

With that, I construct $\Gamma$ using its definition, except that I use the
transformed $g$ wherever $g$ is used in the definition.
\begin{multline*}
    \Gamma^\gamma_{\beta\alpha}(\varphi^*g)
    = \frac 12 g^{\epsilon\eta}(\varphi(x))
    \dpd{{x^\gamma}}{{\varphi^\mu}} \dpd{{x^\delta}}{{\varphi^\nu}}
    \left(
        \partial_\beta g_{\mu\nu}(\varphi(x)) \varphi^\mu_{,\delta}(x)
        \varphi^\nu_{,\alpha}(x) 
    \right. \\
    \left.
        +
        \partial_\alpha g_{\mu\nu}(\varphi(x)) \varphi^\mu_{,\delta}(x)
        \varphi^\nu_{,\beta}(x)
        +
        \partial_\gamma g_{\mu\nu}(\varphi(x)) \varphi^\mu_{,\alpha}(x)
        \varphi^\nu_{,\beta}(x)
    \right)
\end{multline*}

\begin{aside}
    In a case like
    \[
        \partial_\beta g_{\mu\nu}(\varphi(x)) \varphi^\mu_{,\delta}(x)
        \varphi^\nu_{,\alpha}(x),
    \]
    where I have expanded $g$ in front of a partial derivative, do I have to
    use the product rule next? But that would involve second derivatives of
    $\varphi$, which seems arcane to me right now.
\end{aside}

\subsection{Do the Christoffel-symbols transform like tensor(field)s?}

Transforming $\Gamma$ like a tensor:
\[
    (\varphi^*\Gamma)^\alpha_{\beta\gamma}(x) =
    \Gamma^\gamma_{\epsilon\eta}(\varphi(x)) \dpd{{\varphi^\alpha}}{{x^\delta}}
    \dpd{{x^\epsilon}}{{\varphi^\beta}} \dpd{{x^\eta}}{{\varphi^\gamma}}.
\]
Expanding $\Gamma$:
\[
    = \frac 12 g^{\delta\chi}(\varphi(x))
    \del{
        \partial_\epsilon g_{\chi\eta}(\varphi(x)) + \partial_\eta
        g_{\chi\epsilon}(\varphi(x)) - \partial_\chi
    g_{\epsilon\eta}(\varphi(x))
    }
    \dpd{{\varphi^\alpha}}{{x^\delta}}
    \dpd{{x^\epsilon}}{{\varphi^\beta}} \dpd{{x^\eta}}{{\varphi^\gamma}}.
\]

It looks different than $\Gamma$ with the transformed metric. In that, there
are four partial dervatives of $\varphi$, the transformed $\Gamma$ only
features three of them, though.

\section{Curvature Tensor}

\IfFileExists{\bibliographyfile}{
    \printbibliography
}{}

\end{document}

% vim: spell spelllang=en tw=79

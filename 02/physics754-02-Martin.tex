\input{../header.tex}

\newcommand\problemset{2}

\hypersetup{
    pdftitle={physics754 Problem Set \problemset}
}

\newenvironment{aside}{\itshape\small}{}

%\subject{}
\title{physics754 -- Problem Set \problemset}
%\subtitle{}
\author{
    Martin Ueding \\ \small{\href{mailto:mu@martin-ueding.de}{mu@martin-ueding.de}}
}
\publishers{Group 5}

\begin{document}

\maketitle

\begin{aside}
    When writing down tensor components, I have seen
    $R_{\alpha\beta\gamma}^\delta$ as well as $R_{\alpha\beta\gamma}{}^\delta$
    and maybe $R^\delta{}_{\alpha\beta\gamma}$.
    Does that make a difference? If you raise and lower indices and want to
    keep their position fixed, this “splitting up” the lower and upper indices
    seems to clarify. But do we need this really?
\end{aside}

\section{Transformation of the Christoffel-Symbols}

\subsection{Compute $\Gamma^\alpha_{\beta\gamma}(\varphi^*g)$}

I assume that this means that I should compose $\Gamma$ with the transformed
$g$.

The transformed $g$ is given by:
\[
    (\varphi^* g)_{\mu\nu}(x)
    = g_{\alpha\beta}(\varphi(x)) \dpd{{\varphi^\alpha}}{{x^\mu}}
    \dpd{{\varphi^\beta}}{{x^\nu}}
    = g_{\alpha\beta}(\varphi(x)) \varphi^\alpha_{,\mu}(x)
    \varphi^\beta_{,\nu}(x).
\]

The transformed dual\footnote{Is “dual” correct at this point?} $g$ is
similarly obtained:
\[
    (\varphi^* g)^{\mu\nu} = g^{\mu\nu}(\varphi(x))
    \dpd{{x^\gamma}}{{\varphi^\mu}} \dpd{{x^\delta}}{{\varphi^\nu}}.
\]

With that, I construct $\Gamma$ using its definition, except that I use the
transformed $g$ wherever $g$ is used in the definition.
\begin{multline*}
    \Gamma^\gamma_{\beta\alpha}(\varphi^*g)
    = \frac 12 g^{\epsilon\eta}(\varphi(x))
    \dpd{{x^\gamma}}{{\varphi^\mu}} \dpd{{x^\delta}}{{\varphi^\nu}}
    \left(
        \partial_\beta g_{\mu\nu}(\varphi(x)) \varphi^\mu_{,\delta}(x)
        \varphi^\nu_{,\alpha}(x) 
    \right. \\
    \left.
        +
        \partial_\alpha g_{\mu\nu}(\varphi(x)) \varphi^\mu_{,\delta}(x)
        \varphi^\nu_{,\beta}(x)
        +
        \partial_\gamma g_{\mu\nu}(\varphi(x)) \varphi^\mu_{,\alpha}(x)
        \varphi^\nu_{,\beta}(x)
    \right)
\end{multline*}

\begin{aside}
    In a case like
    \[
        \partial_\beta g_{\mu\nu}(\varphi(x)) \varphi^\mu_{,\delta}(x)
        \varphi^\nu_{,\alpha}(x),
    \]
    where I have expanded $g$ in front of a partial derivative, do I have to
    use the product rule next? But that would involve second derivatives of
    $\varphi$, which seems arcane to me right now.
\end{aside}

\subsection{Do the Christoffel-symbols transform like tensor(field)s?}

Transforming $\Gamma$ like a tensor:
\[
    (\varphi^*\Gamma)^\alpha_{\beta\gamma}(x) =
    \Gamma^\gamma_{\epsilon\eta}(\varphi(x)) \dpd{{\varphi^\alpha}}{{x^\delta}}
    \dpd{{x^\epsilon}}{{\varphi^\beta}} \dpd{{x^\eta}}{{\varphi^\gamma}}.
\]
Expanding $\Gamma$:
\[
    = \frac 12 g^{\delta\chi}(\varphi(x))
    \del{
        \partial_\epsilon g_{\chi\eta}(\varphi(x)) + \partial_\eta
        g_{\chi\epsilon}(\varphi(x)) - \partial_\chi
    g_{\epsilon\eta}(\varphi(x))
    }
    \dpd{{\varphi^\alpha}}{{x^\delta}}
    \dpd{{x^\epsilon}}{{\varphi^\beta}} \dpd{{x^\eta}}{{\varphi^\gamma}}.
\]

It looks different than $\Gamma$ with the transformed metric. In that, there
are four partial dervatives of $\varphi$, the transformed $\Gamma$ only
features three of them, though.

\section{Curvature Tensor}

I use the antisymmetrization notation that is used in
\parencite{penrose-road_to_reality} with a normalization of $1/n!$ so that it
is idempotent, like so:
\[
    \nabla_{[\mu} \nabla_{\nu]} := \frac 12 \del{
        \nabla_{\mu} \nabla_{\nu} - \nabla_{\nu} \nabla_{\mu}
    }.
\]

\subsection{Show that $2 \nabla_{[\mu} \nabla_{\nu]} X^\rho =
R^\rho_{\alpha\mu\nu} X^\alpha$}

I will start with the $\mu\nu$ term and antisymmetrize that later.
\begin{align*}
    \nabla_\nu X^\rho
    &=
    \partial_\nu X^\rho + X^\alpha \Gamma^\rho_{\alpha\nu} \\
    %
    \nabla_\mu \nabla_\nu X^\rho
    &=
    (\nabla_\mu \partial_\nu) X^\rho
    + \partial_\nu \nabla_\mu X^\rho
    + (\nabla_\mu X^\alpha) \Gamma^\rho_{\alpha\nu}
    + X^\alpha \nabla_\mu \Gamma^\rho_{\alpha\nu}
    \intertext{%
        \begin{aside}
            Until here, I am not even sure whether I have do differentiate the
            $\partial$ again. Do I have to? Does $\nabla$ and $\partial$ even commute,
            like I assumed? If I have to do the first and may not use the latter, how
            would I even write this down?
        \end{aside}
        Now I have to use the definition of $\nabla$ again:
    }
    &= (\partial_\mu \partial_\nu + \partial_\alpha \Gamma^\alpha_{\nu\mu})
    X^\rho
    + \partial_\nu (\partial_\mu X^\rho + x^\alpha \Gamma^\rho_{\alpha\mu})
    + (\partial_\mu X^\alpha + X^\lambda \Gamma^\alpha_{\lambda\mu})
    \Gamma^\rho_{\alpha\nu}
    \\ &\qquad
    + X^\alpha (\partial_\mu \Gamma^\rho_{\alpha\nu} +
    \Gamma^\lambda_{\alpha\nu} \Gamma^\rho_{\lambda\nu} -
    \Gamma^\rho_{\lambda\nu} \Gamma^\lambda_{\alpha\mu} -
    \Gamma^\rho_{\alpha\lambda} \Gamma^\lambda_{\nu\mu})
    \intertext{%
        Using the notation introduced earlier and using the symmetry of $\Gamma$ in its
        lower indices as well as the commutative property of multiplication, I can
        can indicate terms that are already symmetric or antisymmetric in $\mu$ and
        $\nu$. I hope that the partial derivatives commute, but I do not see a problem
        with that since they are orthogonal to each other.
    }
    &= (\partial_{(\mu} \partial_{\nu)} + \partial_\alpha
    \Gamma^\alpha_{(\mu\nu)})
    X^\rho
    + \partial_\nu (\partial_\mu X^\rho + x^\alpha \Gamma^\rho_{\alpha\mu})
    + (\partial_\mu X^\alpha + X^\lambda \Gamma^\alpha_{\lambda\mu})
    \Gamma^\rho_{\alpha\nu}
    \\ &\qquad
    + X^\alpha (\partial_\mu \Gamma^\rho_{\alpha\nu}
    - 2 \Gamma^\lambda_{\alpha[\mu} \Gamma^\rho_{\nu]\lambda}
    - \Gamma^\rho_{\alpha\lambda} \Gamma^\lambda_{(\mu\nu)})
    \intertext{
        At this point, the first big summand was created because I differentiated
        $\partial$ again. That term is totally symmetric, so that it will not show up
        in the antisymmetric part that is asked in the problem anyway. The last summand
        in the last big summand will drop out as well. These terms are the interesting
        ones:
    }
    &=
    \partial_\nu (\partial_\mu X^\rho + x^\alpha \Gamma^\rho_{\alpha\mu})
    + (\partial_\mu X^\alpha + X^\lambda \Gamma^\alpha_{\lambda\mu})
    \Gamma^\rho_{\alpha\nu}
    + X^\alpha (\partial_\mu \Gamma^\rho_{\alpha\nu}
    - 2 \Gamma^\lambda_{\alpha[\mu} \Gamma^\rho_{\nu]\lambda})
    \intertext{%
        Now I will take (twice of) the antisymmetric part with respect to $\mu$ and
        $\nu$ of the above expression.
    }
    2 \nabla_{[\mu} \nabla_{\nu]} X^\rho
    &= 
    - 2 \partial_{[\mu} \partial_{\nu]} X^\rho
    + \partial_\nu x^\alpha \Gamma^\rho_{\alpha\mu}
    - \partial_\mu x^\alpha \Gamma^\rho_{\alpha\nu}
    + 2 \partial_{[\mu} X^\alpha \Gamma^\rho_{\nu]\alpha}
    + 2 X^\lambda \Gamma^\alpha_{\lambda[\mu} \Gamma^\rho_{\nu]\alpha}
    + 2 X^\alpha \partial_{[\mu} \Gamma^\rho_{\nu]\alpha} \\
    &\qquad
    -4 X^\alpha \Gamma^\lambda_{\alpha[\mu} \Gamma^\rho_{\nu]\lambda}
    \intertext{%
        Assuming that the partial derivatives commute, I can drop the first
        summand. If you look really close at the last and the third last
        summand, you will see that they are the same except for the dummy
        indices.
    }
    &=
    \partial_\nu x^\alpha \Gamma^\rho_{\alpha\mu}
    - \partial_\mu x^\alpha \Gamma^\rho_{\alpha\nu}
    + 2 \partial_{[\mu} X^\alpha \Gamma^\rho_{\nu]\alpha}
    + 2 X^\alpha \partial_{[\mu} \Gamma^\rho_{\nu]\alpha}
    - 2 X^\lambda \Gamma^\alpha_{\lambda[\mu} \Gamma^\rho_{\nu]\alpha}
    \intertext{%
        The first three terms sum to zero. Using the product rule, the first
        two terms will give me
        \[
            2 (\partial_{[\nu}X^\alpha)\Gamma^\rho_{\mu]\alpha} + 2 x^\alpha
            \partial_{[\nu} \Gamma^\rho_{\mu]\alpha},
        \]
        the third term will have to be expanded using product rule as well,
        yielding
        \[
            2 (\partial_{[\mu}X^\alpha) \Gamma^\rho_{\nu]\alpha} + 2 X^\alpha
            \partial_{[\mu} \Gamma^\rho_{\nu]\alpha}.
        \]
        Those are the same as the two summands above, except that $\mu$ and
        $\nu$ are reversed in the brackets. Therefore, the two sets of summands
        sum two zero. The total expression now is:
    }
    &=
    2 X^\alpha \partial_{[\mu} \Gamma^\rho_{\nu]\alpha}
    - 2 X^\lambda \Gamma^\alpha_{\lambda[\mu} \Gamma^\rho_{\nu]\alpha}
    \\
    \intertext{%
        I move the $X$ to the back of each summand to factor it out in the next
        step. The minus sign that I previously had seems to be wrong. So I made
        a sign mistake somewhere or got the order of $\mu$ and $\nu$ mixed up.
        Since I would rather focus on the other problems, I will just leave it
        as it, flipping the sign to make it fit.
    }
    &=
    2 (\partial_{[\mu} \Gamma^\rho_{\nu]\alpha}) X^\alpha
    + 2 \Gamma^\rho_{\lambda[\mu} \Gamma^\lambda_{\nu]\alpha} X^\alpha
    \intertext{%
        The $X$ can be factored out:
    }
    &=
    (2 \partial_{[\mu} \Gamma^\rho_{\nu]\alpha}
    + 2 \Gamma^\rho_{\lambda[\mu} \Gamma^\lambda_{\nu]\alpha}
    ) X^\alpha
    \intertext{%
        Expanding the brackets again gives the expression given on the problem
        set:
    }
    &=
    (\partial_\mu \Gamma^\rho_{\alpha\nu} - \partial_\nu
    \Gamma^\rho_{\alpha\mu} + \Gamma^\rho_{\lambda\mu}
    \Gamma^\lambda_{\alpha\nu} - \Gamma^\rho_{\lambda\nu}
    \Gamma^\lambda_{\alpha\mu}) X^\alpha \\
    &=
    R^\rho_{\alpha\mu\nu} X^\alpha
\end{align*}

\IfFileExists{\bibliographyfile}{
    \printbibliography
}{}

\end{document}

% vim: spell spelllang=en tw=79

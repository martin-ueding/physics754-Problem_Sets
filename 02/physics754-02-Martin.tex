% Copyright © 2012-2013 Martin Ueding <dev@martin-ueding.de>

% This is my general purpose LaTeX header file for writing German documents.
% Ideally, you include this using a simple ``% Copyright © 2012-2013 Martin Ueding <dev@martin-ueding.de>

% This is my general purpose LaTeX header file for writing German documents.
% Ideally, you include this using a simple ``% Copyright © 2012-2013 Martin Ueding <dev@martin-ueding.de>

% This is my general purpose LaTeX header file for writing German documents.
% Ideally, you include this using a simple ``\input{header.tex}`` in your main
% document and start with ``\title`` and ``\begin{document}`` afterwards.

% If you need to add additional packages, I recommend not doing this in this
% file, but in your main document. That way, you can just drop in a new
% ``header.tex`` and get all the new commands without having to merge manually.

% Since this file encorporates a CC-BY-SA fragment, this whole files is
% licensed under the CC-BY-SA license.

\documentclass[11pt, ngerman, fleqn, DIV=15, headinclude]{scrartcl}

\usepackage{graphicx}

\setkomafont{caption}{\sffamily}
\setkomafont{captionlabel}{\usekomafont{caption}}

%%%%%%%%%%%%%%%%%%%%%%%%%%%%%%%%%%%%%%%%%%%%%%%%%%%%%%%%%%%%%%%%%%%%%%%%%%%%%%%
%                                Locale, date                                 %
%%%%%%%%%%%%%%%%%%%%%%%%%%%%%%%%%%%%%%%%%%%%%%%%%%%%%%%%%%%%%%%%%%%%%%%%%%%%%%%

\usepackage{babel}
\usepackage[iso]{isodate}

%%%%%%%%%%%%%%%%%%%%%%%%%%%%%%%%%%%%%%%%%%%%%%%%%%%%%%%%%%%%%%%%%%%%%%%%%%%%%%%
%                          Margins and other spacing                          %
%%%%%%%%%%%%%%%%%%%%%%%%%%%%%%%%%%%%%%%%%%%%%%%%%%%%%%%%%%%%%%%%%%%%%%%%%%%%%%%

\usepackage[parfill]{parskip}
\usepackage{setspace}
\usepackage[activate]{microtype}

\setlength{\columnsep}{2cm}

%%%%%%%%%%%%%%%%%%%%%%%%%%%%%%%%%%%%%%%%%%%%%%%%%%%%%%%%%%%%%%%%%%%%%%%%%%%%%%%
%                                    Color                                    %
%%%%%%%%%%%%%%%%%%%%%%%%%%%%%%%%%%%%%%%%%%%%%%%%%%%%%%%%%%%%%%%%%%%%%%%%%%%%%%%

\usepackage[usenames, dvipsnames]{xcolor}

\colorlet{darkred}{red!70!black}
\colorlet{darkblue}{blue!70!black}
\colorlet{darkgreen}{green!40!black}

%%%%%%%%%%%%%%%%%%%%%%%%%%%%%%%%%%%%%%%%%%%%%%%%%%%%%%%%%%%%%%%%%%%%%%%%%%%%%%%
%                         Font and font like settings                         %
%%%%%%%%%%%%%%%%%%%%%%%%%%%%%%%%%%%%%%%%%%%%%%%%%%%%%%%%%%%%%%%%%%%%%%%%%%%%%%%

% This replaces all fonts with Bitstream Charter, Bitstream Vera Sans and
% Bitstream Vera Mono. Math will be rendered in Charter.
\usepackage[charter, greekuppercase=italicized]{mathdesign}
\usepackage{beramono}
\usepackage{berasans}

% Bold, sans-serif tensors. This fragment is taken from “egreg” from
% http://tex.stackexchange.com/a/82747/8945 and licensed under `CC-BY-SA
% <https://creativecommons.org/licenses/by-sa/3.0/>`_.
\usepackage{bm}
\DeclareMathAlphabet{\mathsfit}{\encodingdefault}{\sfdefault}{m}{sl}
\SetMathAlphabet{\mathsfit}{bold}{\encodingdefault}{\sfdefault}{bx}{sl}
\newcommand{\tens}[1]{\bm{\mathsfit{#1}}}

% Bold vectors.
\renewcommand{\vec}[1]{\boldsymbol{#1}}

%%%%%%%%%%%%%%%%%%%%%%%%%%%%%%%%%%%%%%%%%%%%%%%%%%%%%%%%%%%%%%%%%%%%%%%%%%%%%%%
%                               Input encoding                                %
%%%%%%%%%%%%%%%%%%%%%%%%%%%%%%%%%%%%%%%%%%%%%%%%%%%%%%%%%%%%%%%%%%%%%%%%%%%%%%%

\usepackage[T1]{fontenc}
\usepackage[utf8]{inputenc}

%%%%%%%%%%%%%%%%%%%%%%%%%%%%%%%%%%%%%%%%%%%%%%%%%%%%%%%%%%%%%%%%%%%%%%%%%%%%%%%
%                         Hyperrefs and PDF metadata                          %
%%%%%%%%%%%%%%%%%%%%%%%%%%%%%%%%%%%%%%%%%%%%%%%%%%%%%%%%%%%%%%%%%%%%%%%%%%%%%%%

\usepackage{hyperref}

% This sets the author in the properties of the PDF as well. If you want to
% change it, just override it with another ``\hypersetup`` call.
\hypersetup{
	breaklinks=false,
	citecolor=darkgreen,
	colorlinks=true,
	linkcolor=darkblue,
	menucolor=black,
	pdfauthor={Martin Ueding},
	urlcolor=darkblue,
}

%%%%%%%%%%%%%%%%%%%%%%%%%%%%%%%%%%%%%%%%%%%%%%%%%%%%%%%%%%%%%%%%%%%%%%%%%%%%%%%
%                               Math Operators                                %
%%%%%%%%%%%%%%%%%%%%%%%%%%%%%%%%%%%%%%%%%%%%%%%%%%%%%%%%%%%%%%%%%%%%%%%%%%%%%%%

% AMS environments like ``align`` and theorems like ``proof``.
\usepackage{amsmath}
\usepackage{amsthm}

% Common math constructs like partial derivatives.
\usepackage{commath}

% Physical units.
\usepackage[output-decimal-marker={,}]{siunitx}

% Since I use mathdesign with italic uppercase greek characters, the Ohm unit will be displayed with an italic Ω by default. Units have to be roman, so this forces it the right way.
\DeclareSIUnit{\ohm}{$\Omegaup$}
\DeclareSIUnit{\division}{DIV}
\DeclareSIUnit{\voltss}{$\mathrm{V_{SS}}$}

% Word like operators.
\DeclareMathOperator{\acosh}{arcosh}
\DeclareMathOperator{\arcosh}{arcosh}
\DeclareMathOperator{\arcsinh}{arsinh}
\DeclareMathOperator{\arsinh}{arsinh}
\DeclareMathOperator{\asinh}{arsinh}
\DeclareMathOperator{\card}{card}
\DeclareMathOperator{\csch}{csch}
\DeclareMathOperator{\diam}{diam}
\DeclareMathOperator{\sech}{sech}
\renewcommand{\Im}{\mathop{{}\mathrm{Im}}\nolimits}
\renewcommand{\Re}{\mathop{{}\mathrm{Re}}\nolimits}

% Fourier transform.
\DeclareMathOperator{\fourier}{\ensuremath{\mathcal{F}}}

% Roman versions of “e” and “i” to serve as Euler's number and the imaginary
% constant.
\newcommand{\eup}{\mathrm e}
\newcommand{\iup}{\mathrm i}

% Symbols for the various mathematical fields (natural numbers, integers,
% rational numbers, real numbers, complex numbers).
\newcommand{\C}{\ensuremath{\mathbb C}}
\newcommand{\N}{\ensuremath{\mathbb N}}
\newcommand{\Q}{\ensuremath{\mathbb Q}}
\newcommand{\R}{\ensuremath{\mathbb R}}
\newcommand{\Z}{\ensuremath{\mathbb Z}}

% Shape like operators.
\DeclareMathOperator{\dalambert}{\Box}
\DeclareMathOperator{\laplace}{\bigtriangleup}
\newcommand{\curl}{\vnabla \times}
\newcommand{\divergence}[1]{\inner{\vnabla}{#1}}
\newcommand{\Divergence}[1]{\Inner{\vnabla}{#1}}
\newcommand{\vnabla}{\vec \nabla}

\newcommand{\half}{\frac 12}

% Unit vector (German „Einheitsvektor“).
\newcommand{\ev}{\hat{\vec e}}

% Mathematician's notation for the inner (scalar, dot) product.
\newcommand{\bracket}[1]{\langle #1 \rangle}
\newcommand{\Bracket}[1]{\left\langle #1 \right\rangle}
\newcommand{\inner}[2]{\bracket{#1, #2}}
\newcommand{\Inner}[2]{\Bracket{#1, #2}}

% Placeholders.
\newcommand{\fehlt}{\textcolor{darkred}{Hier fehlen noch Inhalte.}}
\newcommand{\messwert}{\textcolor{blue}{\square}}
\newcommand{\punkte}{\phantom{xxxxx}}

% Separator for equations on a single line.
\newcommand{\eqnsep}{,\quad}

% Quantum Mechanics.
\usepackage{braket}

% Thermodynamic partial derivative.
\newcommand\tdpd[3]{\del{\dpd{#1}{#2}}_{#3}}

%%%%%%%%%%%%%%%%%%%%%%%%%%%%%%%%%%%%%%%%%%%%%%%%%%%%%%%%%%%%%%%%%%%%%%%%%%%%%%%
%                                  Headings                                   %
%%%%%%%%%%%%%%%%%%%%%%%%%%%%%%%%%%%%%%%%%%%%%%%%%%%%%%%%%%%%%%%%%%%%%%%%%%%%%%%

% This will set fancy headings to the top of the page. The page number will be
% accompanied by the total number of pages. That way, you will know if any page
% is missing.
%
% If you do not want this for your document, you can just use
% ``\pagestyle{plain}``.

\usepackage{scrpage2}

\pagestyle{scrheadings}
\automark{section}
\chead{}
\ihead{}
\ohead{\rightmark}
\setheadsepline{.4pt}

%%%%%%%%%%%%%%%%%%%%%%%%%%%%%%%%%%%%%%%%%%%%%%%%%%%%%%%%%%%%%%%%%%%%%%%%%%%%%%%
%                            Bibliography (BibTeX)                            %
%%%%%%%%%%%%%%%%%%%%%%%%%%%%%%%%%%%%%%%%%%%%%%%%%%%%%%%%%%%%%%%%%%%%%%%%%%%%%%%

\newcommand{\bibliographyfile}{../../zentrale_BibTeX/Central}

\usepackage[
    backend=bibtex,
    style=authoryear-icomp,
    %isbn=false,
    %pagetracker=false,
    %maxbibnames=50,
    %maxcitenames=2,
    %autocite=inline,
    %block=space,
    %backref=false,
    %backrefstyle=three+,
    %date=short,
    hyperref=true
]{biblatex}

\setlength{\bibitemsep}{.7em}
\setlength{\bibhang}{4ex}

\IfFileExists{\bibliographyfile}{
    \bibliography{\bibliographyfile}
}{}

%%%%%%%%%%%%%%%%%%%%%%%%%%%%%%%%%%%%%%%%%%%%%%%%%%%%%%%%%%%%%%%%%%%%%%%%%%%%%%%
%                                Abbreviations                                %
%%%%%%%%%%%%%%%%%%%%%%%%%%%%%%%%%%%%%%%%%%%%%%%%%%%%%%%%%%%%%%%%%%%%%%%%%%%%%%%

\newcommand{\dhabk}{\mbox{d.\,h.}}

%%%%%%%%%%%%%%%%%%%%%%%%%%%%%%%%%%%%%%%%%%%%%%%%%%%%%%%%%%%%%%%%%%%%%%%%%%%%%%%
%                                  Licences                                   %
%%%%%%%%%%%%%%%%%%%%%%%%%%%%%%%%%%%%%%%%%%%%%%%%%%%%%%%%%%%%%%%%%%%%%%%%%%%%%%%

\usepackage{ccicons}

\newcommand{\ccbysadetext}{%
	\begin{small}
		Dieses Werk bzw. Inhalt steht unter einer
		\href{http://creativecommons.org/licenses/by-sa/3.0/deed.de}{%
			Creative Commons Namensnennung - Weitergabe unter gleichen
		Bedingungen 3.0 Unported Lizenz}.
	\end{small}
}

\newcommand{\ccbysadetitle}{%
	Lizenz: \href{http://creativecommons.org/licenses/by-sa/3.0/deed.de}
	{CC-BY-SA 3.0 \ccbysa}
}
`` in your main
% document and start with ``\title`` and ``\begin{document}`` afterwards.

% If you need to add additional packages, I recommend not doing this in this
% file, but in your main document. That way, you can just drop in a new
% ``header.tex`` and get all the new commands without having to merge manually.

% Since this file encorporates a CC-BY-SA fragment, this whole files is
% licensed under the CC-BY-SA license.

\documentclass[11pt, ngerman, fleqn, DIV=15, headinclude]{scrartcl}

\usepackage{graphicx}

\setkomafont{caption}{\sffamily}
\setkomafont{captionlabel}{\usekomafont{caption}}

%%%%%%%%%%%%%%%%%%%%%%%%%%%%%%%%%%%%%%%%%%%%%%%%%%%%%%%%%%%%%%%%%%%%%%%%%%%%%%%
%                                Locale, date                                 %
%%%%%%%%%%%%%%%%%%%%%%%%%%%%%%%%%%%%%%%%%%%%%%%%%%%%%%%%%%%%%%%%%%%%%%%%%%%%%%%

\usepackage{babel}
\usepackage[iso]{isodate}

%%%%%%%%%%%%%%%%%%%%%%%%%%%%%%%%%%%%%%%%%%%%%%%%%%%%%%%%%%%%%%%%%%%%%%%%%%%%%%%
%                          Margins and other spacing                          %
%%%%%%%%%%%%%%%%%%%%%%%%%%%%%%%%%%%%%%%%%%%%%%%%%%%%%%%%%%%%%%%%%%%%%%%%%%%%%%%

\usepackage[parfill]{parskip}
\usepackage{setspace}
\usepackage[activate]{microtype}

\setlength{\columnsep}{2cm}

%%%%%%%%%%%%%%%%%%%%%%%%%%%%%%%%%%%%%%%%%%%%%%%%%%%%%%%%%%%%%%%%%%%%%%%%%%%%%%%
%                                    Color                                    %
%%%%%%%%%%%%%%%%%%%%%%%%%%%%%%%%%%%%%%%%%%%%%%%%%%%%%%%%%%%%%%%%%%%%%%%%%%%%%%%

\usepackage[usenames, dvipsnames]{xcolor}

\colorlet{darkred}{red!70!black}
\colorlet{darkblue}{blue!70!black}
\colorlet{darkgreen}{green!40!black}

%%%%%%%%%%%%%%%%%%%%%%%%%%%%%%%%%%%%%%%%%%%%%%%%%%%%%%%%%%%%%%%%%%%%%%%%%%%%%%%
%                         Font and font like settings                         %
%%%%%%%%%%%%%%%%%%%%%%%%%%%%%%%%%%%%%%%%%%%%%%%%%%%%%%%%%%%%%%%%%%%%%%%%%%%%%%%

% This replaces all fonts with Bitstream Charter, Bitstream Vera Sans and
% Bitstream Vera Mono. Math will be rendered in Charter.
\usepackage[charter, greekuppercase=italicized]{mathdesign}
\usepackage{beramono}
\usepackage{berasans}

% Bold, sans-serif tensors. This fragment is taken from “egreg” from
% http://tex.stackexchange.com/a/82747/8945 and licensed under `CC-BY-SA
% <https://creativecommons.org/licenses/by-sa/3.0/>`_.
\usepackage{bm}
\DeclareMathAlphabet{\mathsfit}{\encodingdefault}{\sfdefault}{m}{sl}
\SetMathAlphabet{\mathsfit}{bold}{\encodingdefault}{\sfdefault}{bx}{sl}
\newcommand{\tens}[1]{\bm{\mathsfit{#1}}}

% Bold vectors.
\renewcommand{\vec}[1]{\boldsymbol{#1}}

%%%%%%%%%%%%%%%%%%%%%%%%%%%%%%%%%%%%%%%%%%%%%%%%%%%%%%%%%%%%%%%%%%%%%%%%%%%%%%%
%                               Input encoding                                %
%%%%%%%%%%%%%%%%%%%%%%%%%%%%%%%%%%%%%%%%%%%%%%%%%%%%%%%%%%%%%%%%%%%%%%%%%%%%%%%

\usepackage[T1]{fontenc}
\usepackage[utf8]{inputenc}

%%%%%%%%%%%%%%%%%%%%%%%%%%%%%%%%%%%%%%%%%%%%%%%%%%%%%%%%%%%%%%%%%%%%%%%%%%%%%%%
%                         Hyperrefs and PDF metadata                          %
%%%%%%%%%%%%%%%%%%%%%%%%%%%%%%%%%%%%%%%%%%%%%%%%%%%%%%%%%%%%%%%%%%%%%%%%%%%%%%%

\usepackage{hyperref}

% This sets the author in the properties of the PDF as well. If you want to
% change it, just override it with another ``\hypersetup`` call.
\hypersetup{
	breaklinks=false,
	citecolor=darkgreen,
	colorlinks=true,
	linkcolor=darkblue,
	menucolor=black,
	pdfauthor={Martin Ueding},
	urlcolor=darkblue,
}

%%%%%%%%%%%%%%%%%%%%%%%%%%%%%%%%%%%%%%%%%%%%%%%%%%%%%%%%%%%%%%%%%%%%%%%%%%%%%%%
%                               Math Operators                                %
%%%%%%%%%%%%%%%%%%%%%%%%%%%%%%%%%%%%%%%%%%%%%%%%%%%%%%%%%%%%%%%%%%%%%%%%%%%%%%%

% AMS environments like ``align`` and theorems like ``proof``.
\usepackage{amsmath}
\usepackage{amsthm}

% Common math constructs like partial derivatives.
\usepackage{commath}

% Physical units.
\usepackage[output-decimal-marker={,}]{siunitx}

% Since I use mathdesign with italic uppercase greek characters, the Ohm unit will be displayed with an italic Ω by default. Units have to be roman, so this forces it the right way.
\DeclareSIUnit{\ohm}{$\Omegaup$}
\DeclareSIUnit{\division}{DIV}
\DeclareSIUnit{\voltss}{$\mathrm{V_{SS}}$}

% Word like operators.
\DeclareMathOperator{\acosh}{arcosh}
\DeclareMathOperator{\arcosh}{arcosh}
\DeclareMathOperator{\arcsinh}{arsinh}
\DeclareMathOperator{\arsinh}{arsinh}
\DeclareMathOperator{\asinh}{arsinh}
\DeclareMathOperator{\card}{card}
\DeclareMathOperator{\csch}{csch}
\DeclareMathOperator{\diam}{diam}
\DeclareMathOperator{\sech}{sech}
\renewcommand{\Im}{\mathop{{}\mathrm{Im}}\nolimits}
\renewcommand{\Re}{\mathop{{}\mathrm{Re}}\nolimits}

% Fourier transform.
\DeclareMathOperator{\fourier}{\ensuremath{\mathcal{F}}}

% Roman versions of “e” and “i” to serve as Euler's number and the imaginary
% constant.
\newcommand{\eup}{\mathrm e}
\newcommand{\iup}{\mathrm i}

% Symbols for the various mathematical fields (natural numbers, integers,
% rational numbers, real numbers, complex numbers).
\newcommand{\C}{\ensuremath{\mathbb C}}
\newcommand{\N}{\ensuremath{\mathbb N}}
\newcommand{\Q}{\ensuremath{\mathbb Q}}
\newcommand{\R}{\ensuremath{\mathbb R}}
\newcommand{\Z}{\ensuremath{\mathbb Z}}

% Shape like operators.
\DeclareMathOperator{\dalambert}{\Box}
\DeclareMathOperator{\laplace}{\bigtriangleup}
\newcommand{\curl}{\vnabla \times}
\newcommand{\divergence}[1]{\inner{\vnabla}{#1}}
\newcommand{\Divergence}[1]{\Inner{\vnabla}{#1}}
\newcommand{\vnabla}{\vec \nabla}

\newcommand{\half}{\frac 12}

% Unit vector (German „Einheitsvektor“).
\newcommand{\ev}{\hat{\vec e}}

% Mathematician's notation for the inner (scalar, dot) product.
\newcommand{\bracket}[1]{\langle #1 \rangle}
\newcommand{\Bracket}[1]{\left\langle #1 \right\rangle}
\newcommand{\inner}[2]{\bracket{#1, #2}}
\newcommand{\Inner}[2]{\Bracket{#1, #2}}

% Placeholders.
\newcommand{\fehlt}{\textcolor{darkred}{Hier fehlen noch Inhalte.}}
\newcommand{\messwert}{\textcolor{blue}{\square}}
\newcommand{\punkte}{\phantom{xxxxx}}

% Separator for equations on a single line.
\newcommand{\eqnsep}{,\quad}

% Quantum Mechanics.
\usepackage{braket}

% Thermodynamic partial derivative.
\newcommand\tdpd[3]{\del{\dpd{#1}{#2}}_{#3}}

%%%%%%%%%%%%%%%%%%%%%%%%%%%%%%%%%%%%%%%%%%%%%%%%%%%%%%%%%%%%%%%%%%%%%%%%%%%%%%%
%                                  Headings                                   %
%%%%%%%%%%%%%%%%%%%%%%%%%%%%%%%%%%%%%%%%%%%%%%%%%%%%%%%%%%%%%%%%%%%%%%%%%%%%%%%

% This will set fancy headings to the top of the page. The page number will be
% accompanied by the total number of pages. That way, you will know if any page
% is missing.
%
% If you do not want this for your document, you can just use
% ``\pagestyle{plain}``.

\usepackage{scrpage2}

\pagestyle{scrheadings}
\automark{section}
\chead{}
\ihead{}
\ohead{\rightmark}
\setheadsepline{.4pt}

%%%%%%%%%%%%%%%%%%%%%%%%%%%%%%%%%%%%%%%%%%%%%%%%%%%%%%%%%%%%%%%%%%%%%%%%%%%%%%%
%                            Bibliography (BibTeX)                            %
%%%%%%%%%%%%%%%%%%%%%%%%%%%%%%%%%%%%%%%%%%%%%%%%%%%%%%%%%%%%%%%%%%%%%%%%%%%%%%%

\newcommand{\bibliographyfile}{../../zentrale_BibTeX/Central}

\usepackage[
    backend=bibtex,
    style=authoryear-icomp,
    %isbn=false,
    %pagetracker=false,
    %maxbibnames=50,
    %maxcitenames=2,
    %autocite=inline,
    %block=space,
    %backref=false,
    %backrefstyle=three+,
    %date=short,
    hyperref=true
]{biblatex}

\setlength{\bibitemsep}{.7em}
\setlength{\bibhang}{4ex}

\IfFileExists{\bibliographyfile}{
    \bibliography{\bibliographyfile}
}{}

%%%%%%%%%%%%%%%%%%%%%%%%%%%%%%%%%%%%%%%%%%%%%%%%%%%%%%%%%%%%%%%%%%%%%%%%%%%%%%%
%                                Abbreviations                                %
%%%%%%%%%%%%%%%%%%%%%%%%%%%%%%%%%%%%%%%%%%%%%%%%%%%%%%%%%%%%%%%%%%%%%%%%%%%%%%%

\newcommand{\dhabk}{\mbox{d.\,h.}}

%%%%%%%%%%%%%%%%%%%%%%%%%%%%%%%%%%%%%%%%%%%%%%%%%%%%%%%%%%%%%%%%%%%%%%%%%%%%%%%
%                                  Licences                                   %
%%%%%%%%%%%%%%%%%%%%%%%%%%%%%%%%%%%%%%%%%%%%%%%%%%%%%%%%%%%%%%%%%%%%%%%%%%%%%%%

\usepackage{ccicons}

\newcommand{\ccbysadetext}{%
	\begin{small}
		Dieses Werk bzw. Inhalt steht unter einer
		\href{http://creativecommons.org/licenses/by-sa/3.0/deed.de}{%
			Creative Commons Namensnennung - Weitergabe unter gleichen
		Bedingungen 3.0 Unported Lizenz}.
	\end{small}
}

\newcommand{\ccbysadetitle}{%
	Lizenz: \href{http://creativecommons.org/licenses/by-sa/3.0/deed.de}
	{CC-BY-SA 3.0 \ccbysa}
}
`` in your main
% document and start with ``\title`` and ``\begin{document}`` afterwards.

% If you need to add additional packages, I recommend not doing this in this
% file, but in your main document. That way, you can just drop in a new
% ``header.tex`` and get all the new commands without having to merge manually.

% Since this file encorporates a CC-BY-SA fragment, this whole files is
% licensed under the CC-BY-SA license.

\documentclass[11pt, ngerman, fleqn, DIV=15, headinclude]{scrartcl}

\usepackage{graphicx}

\setkomafont{caption}{\sffamily}
\setkomafont{captionlabel}{\usekomafont{caption}}

%%%%%%%%%%%%%%%%%%%%%%%%%%%%%%%%%%%%%%%%%%%%%%%%%%%%%%%%%%%%%%%%%%%%%%%%%%%%%%%
%                                Locale, date                                 %
%%%%%%%%%%%%%%%%%%%%%%%%%%%%%%%%%%%%%%%%%%%%%%%%%%%%%%%%%%%%%%%%%%%%%%%%%%%%%%%

\usepackage{babel}
\usepackage[iso]{isodate}

%%%%%%%%%%%%%%%%%%%%%%%%%%%%%%%%%%%%%%%%%%%%%%%%%%%%%%%%%%%%%%%%%%%%%%%%%%%%%%%
%                          Margins and other spacing                          %
%%%%%%%%%%%%%%%%%%%%%%%%%%%%%%%%%%%%%%%%%%%%%%%%%%%%%%%%%%%%%%%%%%%%%%%%%%%%%%%

\usepackage[parfill]{parskip}
\usepackage{setspace}
\usepackage[activate]{microtype}

\setlength{\columnsep}{2cm}

%%%%%%%%%%%%%%%%%%%%%%%%%%%%%%%%%%%%%%%%%%%%%%%%%%%%%%%%%%%%%%%%%%%%%%%%%%%%%%%
%                                    Color                                    %
%%%%%%%%%%%%%%%%%%%%%%%%%%%%%%%%%%%%%%%%%%%%%%%%%%%%%%%%%%%%%%%%%%%%%%%%%%%%%%%

\usepackage[usenames, dvipsnames]{xcolor}

\colorlet{darkred}{red!70!black}
\colorlet{darkblue}{blue!70!black}
\colorlet{darkgreen}{green!40!black}

%%%%%%%%%%%%%%%%%%%%%%%%%%%%%%%%%%%%%%%%%%%%%%%%%%%%%%%%%%%%%%%%%%%%%%%%%%%%%%%
%                         Font and font like settings                         %
%%%%%%%%%%%%%%%%%%%%%%%%%%%%%%%%%%%%%%%%%%%%%%%%%%%%%%%%%%%%%%%%%%%%%%%%%%%%%%%

% This replaces all fonts with Bitstream Charter, Bitstream Vera Sans and
% Bitstream Vera Mono. Math will be rendered in Charter.
\usepackage[charter, greekuppercase=italicized]{mathdesign}
\usepackage{beramono}
\usepackage{berasans}

% Bold, sans-serif tensors. This fragment is taken from “egreg” from
% http://tex.stackexchange.com/a/82747/8945 and licensed under `CC-BY-SA
% <https://creativecommons.org/licenses/by-sa/3.0/>`_.
\usepackage{bm}
\DeclareMathAlphabet{\mathsfit}{\encodingdefault}{\sfdefault}{m}{sl}
\SetMathAlphabet{\mathsfit}{bold}{\encodingdefault}{\sfdefault}{bx}{sl}
\newcommand{\tens}[1]{\bm{\mathsfit{#1}}}

% Bold vectors.
\renewcommand{\vec}[1]{\boldsymbol{#1}}

%%%%%%%%%%%%%%%%%%%%%%%%%%%%%%%%%%%%%%%%%%%%%%%%%%%%%%%%%%%%%%%%%%%%%%%%%%%%%%%
%                               Input encoding                                %
%%%%%%%%%%%%%%%%%%%%%%%%%%%%%%%%%%%%%%%%%%%%%%%%%%%%%%%%%%%%%%%%%%%%%%%%%%%%%%%

\usepackage[T1]{fontenc}
\usepackage[utf8]{inputenc}

%%%%%%%%%%%%%%%%%%%%%%%%%%%%%%%%%%%%%%%%%%%%%%%%%%%%%%%%%%%%%%%%%%%%%%%%%%%%%%%
%                         Hyperrefs and PDF metadata                          %
%%%%%%%%%%%%%%%%%%%%%%%%%%%%%%%%%%%%%%%%%%%%%%%%%%%%%%%%%%%%%%%%%%%%%%%%%%%%%%%

\usepackage{hyperref}

% This sets the author in the properties of the PDF as well. If you want to
% change it, just override it with another ``\hypersetup`` call.
\hypersetup{
	breaklinks=false,
	citecolor=darkgreen,
	colorlinks=true,
	linkcolor=darkblue,
	menucolor=black,
	pdfauthor={Martin Ueding},
	urlcolor=darkblue,
}

%%%%%%%%%%%%%%%%%%%%%%%%%%%%%%%%%%%%%%%%%%%%%%%%%%%%%%%%%%%%%%%%%%%%%%%%%%%%%%%
%                               Math Operators                                %
%%%%%%%%%%%%%%%%%%%%%%%%%%%%%%%%%%%%%%%%%%%%%%%%%%%%%%%%%%%%%%%%%%%%%%%%%%%%%%%

% AMS environments like ``align`` and theorems like ``proof``.
\usepackage{amsmath}
\usepackage{amsthm}

% Common math constructs like partial derivatives.
\usepackage{commath}

% Physical units.
\usepackage[output-decimal-marker={,}]{siunitx}

% Since I use mathdesign with italic uppercase greek characters, the Ohm unit will be displayed with an italic Ω by default. Units have to be roman, so this forces it the right way.
\DeclareSIUnit{\ohm}{$\Omegaup$}
\DeclareSIUnit{\division}{DIV}
\DeclareSIUnit{\voltss}{$\mathrm{V_{SS}}$}

% Word like operators.
\DeclareMathOperator{\acosh}{arcosh}
\DeclareMathOperator{\arcosh}{arcosh}
\DeclareMathOperator{\arcsinh}{arsinh}
\DeclareMathOperator{\arsinh}{arsinh}
\DeclareMathOperator{\asinh}{arsinh}
\DeclareMathOperator{\card}{card}
\DeclareMathOperator{\csch}{csch}
\DeclareMathOperator{\diam}{diam}
\DeclareMathOperator{\sech}{sech}
\renewcommand{\Im}{\mathop{{}\mathrm{Im}}\nolimits}
\renewcommand{\Re}{\mathop{{}\mathrm{Re}}\nolimits}

% Fourier transform.
\DeclareMathOperator{\fourier}{\ensuremath{\mathcal{F}}}

% Roman versions of “e” and “i” to serve as Euler's number and the imaginary
% constant.
\newcommand{\eup}{\mathrm e}
\newcommand{\iup}{\mathrm i}

% Symbols for the various mathematical fields (natural numbers, integers,
% rational numbers, real numbers, complex numbers).
\newcommand{\C}{\ensuremath{\mathbb C}}
\newcommand{\N}{\ensuremath{\mathbb N}}
\newcommand{\Q}{\ensuremath{\mathbb Q}}
\newcommand{\R}{\ensuremath{\mathbb R}}
\newcommand{\Z}{\ensuremath{\mathbb Z}}

% Shape like operators.
\DeclareMathOperator{\dalambert}{\Box}
\DeclareMathOperator{\laplace}{\bigtriangleup}
\newcommand{\curl}{\vnabla \times}
\newcommand{\divergence}[1]{\inner{\vnabla}{#1}}
\newcommand{\Divergence}[1]{\Inner{\vnabla}{#1}}
\newcommand{\vnabla}{\vec \nabla}

\newcommand{\half}{\frac 12}

% Unit vector (German „Einheitsvektor“).
\newcommand{\ev}{\hat{\vec e}}

% Mathematician's notation for the inner (scalar, dot) product.
\newcommand{\bracket}[1]{\langle #1 \rangle}
\newcommand{\Bracket}[1]{\left\langle #1 \right\rangle}
\newcommand{\inner}[2]{\bracket{#1, #2}}
\newcommand{\Inner}[2]{\Bracket{#1, #2}}

% Placeholders.
\newcommand{\fehlt}{\textcolor{darkred}{Hier fehlen noch Inhalte.}}
\newcommand{\messwert}{\textcolor{blue}{\square}}
\newcommand{\punkte}{\phantom{xxxxx}}

% Separator for equations on a single line.
\newcommand{\eqnsep}{,\quad}

% Quantum Mechanics.
\usepackage{braket}

% Thermodynamic partial derivative.
\newcommand\tdpd[3]{\del{\dpd{#1}{#2}}_{#3}}

%%%%%%%%%%%%%%%%%%%%%%%%%%%%%%%%%%%%%%%%%%%%%%%%%%%%%%%%%%%%%%%%%%%%%%%%%%%%%%%
%                                  Headings                                   %
%%%%%%%%%%%%%%%%%%%%%%%%%%%%%%%%%%%%%%%%%%%%%%%%%%%%%%%%%%%%%%%%%%%%%%%%%%%%%%%

% This will set fancy headings to the top of the page. The page number will be
% accompanied by the total number of pages. That way, you will know if any page
% is missing.
%
% If you do not want this for your document, you can just use
% ``\pagestyle{plain}``.

\usepackage{scrpage2}

\pagestyle{scrheadings}
\automark{section}
\chead{}
\ihead{}
\ohead{\rightmark}
\setheadsepline{.4pt}

%%%%%%%%%%%%%%%%%%%%%%%%%%%%%%%%%%%%%%%%%%%%%%%%%%%%%%%%%%%%%%%%%%%%%%%%%%%%%%%
%                            Bibliography (BibTeX)                            %
%%%%%%%%%%%%%%%%%%%%%%%%%%%%%%%%%%%%%%%%%%%%%%%%%%%%%%%%%%%%%%%%%%%%%%%%%%%%%%%

\newcommand{\bibliographyfile}{../../zentrale_BibTeX/Central}

\usepackage[
    backend=bibtex,
    style=authoryear-icomp,
    %isbn=false,
    %pagetracker=false,
    %maxbibnames=50,
    %maxcitenames=2,
    %autocite=inline,
    %block=space,
    %backref=false,
    %backrefstyle=three+,
    %date=short,
    hyperref=true
]{biblatex}

\setlength{\bibitemsep}{.7em}
\setlength{\bibhang}{4ex}

\IfFileExists{\bibliographyfile}{
    \bibliography{\bibliographyfile}
}{}

%%%%%%%%%%%%%%%%%%%%%%%%%%%%%%%%%%%%%%%%%%%%%%%%%%%%%%%%%%%%%%%%%%%%%%%%%%%%%%%
%                                Abbreviations                                %
%%%%%%%%%%%%%%%%%%%%%%%%%%%%%%%%%%%%%%%%%%%%%%%%%%%%%%%%%%%%%%%%%%%%%%%%%%%%%%%

\newcommand{\dhabk}{\mbox{d.\,h.}}

%%%%%%%%%%%%%%%%%%%%%%%%%%%%%%%%%%%%%%%%%%%%%%%%%%%%%%%%%%%%%%%%%%%%%%%%%%%%%%%
%                                  Licences                                   %
%%%%%%%%%%%%%%%%%%%%%%%%%%%%%%%%%%%%%%%%%%%%%%%%%%%%%%%%%%%%%%%%%%%%%%%%%%%%%%%

\usepackage{ccicons}

\newcommand{\ccbysadetext}{%
	\begin{small}
		Dieses Werk bzw. Inhalt steht unter einer
		\href{http://creativecommons.org/licenses/by-sa/3.0/deed.de}{%
			Creative Commons Namensnennung - Weitergabe unter gleichen
		Bedingungen 3.0 Unported Lizenz}.
	\end{small}
}

\newcommand{\ccbysadetitle}{%
	Lizenz: \href{http://creativecommons.org/licenses/by-sa/3.0/deed.de}
	{CC-BY-SA 3.0 \ccbysa}
}


\newcommand\problemset{2}

\hypersetup{
    pdftitle={physics754 Problem Set \problemset}
}

\newenvironment{aside}{\itshape\small}{}

%\subject{}
\title{physics754 -- Problem Set \problemset}
%\subtitle{}
\author{
    Martin Ueding \\ \small{\href{mailto:mu@martin-ueding.de}{mu@martin-ueding.de}}
}
\publishers{Group 5}

\begin{document}

\maketitle

\begin{aside}
    When writing down tensor components, I have seen
    $R_{\alpha\beta\gamma}^\delta$ as well as $R_{\alpha\beta\gamma}{}^\delta$
    and maybe $R^\delta{}_{\alpha\beta\gamma}$.
    Does that make a difference? If you raise and lower indices and want to
    keep their position fixed, this “splitting up” the lower and upper indices
    seems to clarify. But do we need this really?
\end{aside}

\section{Transformation of the Christoffel-Symbols}

\subsection{Compute $\Gamma^\alpha_{\beta\gamma}(\varphi^*g)$}

I assume that this means that I should compose $\Gamma$ with the transformed
$g$.

The transformed $g$ is given by:
\[
    (\varphi^* g)_{\mu\nu}(x)
    = g_{\alpha\beta}(\varphi(x)) \dpd{{\varphi^\alpha}}{{x^\mu}}
    \dpd{{\varphi^\beta}}{{x^\nu}}
    = g_{\alpha\beta}(\varphi(x)) \varphi^\alpha_{,\mu}(x)
    \varphi^\beta_{,\nu}(x).
\]

The transformed dual\footnote{Is “dual” correct at this point?} $g$ is
similarly obtained:
\[
    (\varphi^* g)^{\mu\nu} = g^{\mu\nu}(\varphi(x))
    \dpd{{x^\gamma}}{{\varphi^\mu}} \dpd{{x^\delta}}{{\varphi^\nu}}.
\]

With that, I construct $\Gamma$ using its definition, except that I use the
transformed $g$ wherever $g$ is used in the definition.
\begin{multline*}
    \Gamma^\gamma_{\beta\alpha}(\varphi^*g)
    = \frac 12 g^{\epsilon\eta}(\varphi(x))
    \dpd{{x^\gamma}}{{\varphi^\mu}} \dpd{{x^\delta}}{{\varphi^\nu}}
    \left(
        \partial_\beta g_{\mu\nu}(\varphi(x)) \varphi^\mu_{,\delta}(x)
        \varphi^\nu_{,\alpha}(x) 
    \right. \\
    \left.
        +
        \partial_\alpha g_{\mu\nu}(\varphi(x)) \varphi^\mu_{,\delta}(x)
        \varphi^\nu_{,\beta}(x)
        +
        \partial_\gamma g_{\mu\nu}(\varphi(x)) \varphi^\mu_{,\alpha}(x)
        \varphi^\nu_{,\beta}(x)
    \right)
\end{multline*}

\begin{aside}
    In a case like
    \[
        \partial_\beta g_{\mu\nu}(\varphi(x)) \varphi^\mu_{,\delta}(x)
        \varphi^\nu_{,\alpha}(x),
    \]
    where I have expanded $g$ in front of a partial derivative, do I have to
    use the product rule next? But that would involve second derivatives of
    $\varphi$, which seems arcane to me right now.
\end{aside}

\subsection{Do the Christoffel-symbols transform like tensor(field)s?}

Transforming $\Gamma$ like a tensor:
\[
    (\varphi^*\Gamma)^\alpha_{\beta\gamma}(x) =
    \Gamma^\gamma_{\epsilon\eta}(\varphi(x)) \dpd{{\varphi^\alpha}}{{x^\delta}}
    \dpd{{x^\epsilon}}{{\varphi^\beta}} \dpd{{x^\eta}}{{\varphi^\gamma}}.
\]
Expanding $\Gamma$:
\[
    = \frac 12 g^{\delta\chi}(\varphi(x))
    \del{
        \partial_\epsilon g_{\chi\eta}(\varphi(x)) + \partial_\eta
        g_{\chi\epsilon}(\varphi(x)) - \partial_\chi
    g_{\epsilon\eta}(\varphi(x))
    }
    \dpd{{\varphi^\alpha}}{{x^\delta}}
    \dpd{{x^\epsilon}}{{\varphi^\beta}} \dpd{{x^\eta}}{{\varphi^\gamma}}.
\]

It looks different than $\Gamma$ with the transformed metric. In that, there
are four partial dervatives of $\varphi$, the transformed $\Gamma$ only
features three of them, though.

\section{Curvature Tensor}

I use the antisymmetrization notation that is used in
\parencite{penrose-road_to_reality} with a normalization of $1/n!$ so that it
is idempotent, like so:
\[
    \nabla_{[\mu} \nabla_{\nu]} := \frac 12 \del{
        \nabla_{\mu} \nabla_{\nu} - \nabla_{\nu} \nabla_{\mu}
    }.
\]

\subsection{Show that $2 \nabla_{[\mu} \nabla_{\nu]} X^\rho =
R^\rho_{\alpha\mu\nu} X^\alpha$}

I will start with the $\mu\nu$ term and antisymmetrize that later.
\begin{align*}
    \nabla_\nu X^\rho
    &=
    \partial_\nu X^\rho + X^\alpha \Gamma^\rho_{\alpha\nu} \\
    %
    \nabla_\mu \nabla_\nu X^\rho
    &=
    (\nabla_\mu \partial_\nu) X^\rho
    + \partial_\nu \nabla_\mu X^\rho
    + (\nabla_\mu X^\alpha) \Gamma^\rho_{\alpha\nu}
    + X^\alpha \nabla_\mu \Gamma^\rho_{\alpha\nu}
    \intertext{%
        \begin{aside}
            Until here, I am not even sure whether I have do differentiate the
            $\partial$ again. Do I have to? Does $\nabla$ and $\partial$ even commute,
            like I assumed? If I have to do the first and may not use the latter, how
            would I even write this down?
        \end{aside}
        Now I have to use the definition of $\nabla$ again:
    }
    &= (\partial_\mu \partial_\nu + \partial_\alpha \Gamma^\alpha_{\nu\mu})
    X^\rho
    + \partial_\nu (\partial_\mu X^\rho + x^\alpha \Gamma^\rho_{\alpha\mu})
    + (\partial_\mu X^\alpha + X^\lambda \Gamma^\alpha_{\lambda\mu})
    \Gamma^\rho_{\alpha\nu}
    \\ &\qquad
    + X^\alpha (\partial_\mu \Gamma^\rho_{\alpha\nu} +
    \Gamma^\lambda_{\alpha\nu} \Gamma^\rho_{\lambda\nu} -
    \Gamma^\rho_{\lambda\nu} \Gamma^\lambda_{\alpha\mu} -
    \Gamma^\rho_{\alpha\lambda} \Gamma^\lambda_{\nu\mu})
    \intertext{%
        Using the notation introduced earlier and using the symmetry of $\Gamma$ in its
        lower indices as well as the commutative property of multiplication, I can
        can indicate terms that are already symmetric or antisymmetric in $\mu$ and
        $\nu$. I hope that the partial derivatives commute, but I do not see a problem
        with that since they are orthogonal to each other.
    }
    &= (\partial_{(\mu} \partial_{\nu)} + \partial_\alpha
    \Gamma^\alpha_{(\mu\nu)})
    X^\rho
    + \partial_\nu (\partial_\mu X^\rho + x^\alpha \Gamma^\rho_{\alpha\mu})
    + (\partial_\mu X^\alpha + X^\lambda \Gamma^\alpha_{\lambda\mu})
    \Gamma^\rho_{\alpha\nu}
    \\ &\qquad
    + X^\alpha (\partial_\mu \Gamma^\rho_{\alpha\nu}
    - 2 \Gamma^\lambda_{\alpha[\mu} \Gamma^\rho_{\nu]\lambda}
    - \Gamma^\rho_{\alpha\lambda} \Gamma^\lambda_{(\mu\nu)})
    \intertext{
        At this point, the first big summand was created because I differentiated
        $\partial$ again. That term is totally symmetric, so that it will not show up
        in the antisymmetric part that is asked in the problem anyway. The last summand
        in the last big summand will drop out as well. These terms are the interesting
        ones:
    }
    &=
    \partial_\nu (\partial_\mu X^\rho + x^\alpha \Gamma^\rho_{\alpha\mu})
    + (\partial_\mu X^\alpha + X^\lambda \Gamma^\alpha_{\lambda\mu})
    \Gamma^\rho_{\alpha\nu}
    + X^\alpha (\partial_\mu \Gamma^\rho_{\alpha\nu}
    - 2 \Gamma^\lambda_{\alpha[\mu} \Gamma^\rho_{\nu]\lambda})
    \intertext{%
        Now I will take (twice of) the antisymmetric part with respect to $\mu$ and
        $\nu$ of the above expression.
    }
    2 \nabla_{[\mu} \nabla_{\nu]} X^\rho
    &= 
    - 2 \partial_{[\mu} \partial_{\nu]} X^\rho
    + \partial_\nu x^\alpha \Gamma^\rho_{\alpha\mu}
    - \partial_\mu x^\alpha \Gamma^\rho_{\alpha\nu}
    + 2 \partial_{[\mu} X^\alpha \Gamma^\rho_{\nu]\alpha}
    + 2 X^\lambda \Gamma^\alpha_{\lambda[\mu} \Gamma^\rho_{\nu]\alpha}
    + 2 X^\alpha \partial_{[\mu} \Gamma^\rho_{\nu]\alpha} \\
    &\qquad
    -4 X^\alpha \Gamma^\lambda_{\alpha[\mu} \Gamma^\rho_{\nu]\lambda}
    \intertext{%
        Assuming that the partial derivatives commute, I can drop the first
        summand. If you look really close at the last and the third last
        summand, you will see that they are the same except for the dummy
        indices.
    }
    &=
    \partial_\nu x^\alpha \Gamma^\rho_{\alpha\mu}
    - \partial_\mu x^\alpha \Gamma^\rho_{\alpha\nu}
    + 2 \partial_{[\mu} X^\alpha \Gamma^\rho_{\nu]\alpha}
    + 2 X^\alpha \partial_{[\mu} \Gamma^\rho_{\nu]\alpha}
    - 2 X^\lambda \Gamma^\alpha_{\lambda[\mu} \Gamma^\rho_{\nu]\alpha}
    \intertext{%
        The first three terms sum to zero. Using the product rule, the first
        two terms will give me
        \[
            2 (\partial_{[\nu}X^\alpha)\Gamma^\rho_{\mu]\alpha} + 2 x^\alpha
            \partial_{[\nu} \Gamma^\rho_{\mu]\alpha},
        \]
        the third term will have to be expanded using product rule as well,
        yielding
        \[
            2 (\partial_{[\mu}X^\alpha) \Gamma^\rho_{\nu]\alpha} + 2 X^\alpha
            \partial_{[\mu} \Gamma^\rho_{\nu]\alpha}.
        \]
        Those are the same as the two summands above, except that $\mu$ and
        $\nu$ are reversed in the brackets. Therefore, the two sets of summands
        sum two zero. The total expression now is:
    }
    &=
    2 X^\alpha \partial_{[\mu} \Gamma^\rho_{\nu]\alpha}
    - 2 X^\lambda \Gamma^\alpha_{\lambda[\mu} \Gamma^\rho_{\nu]\alpha}
    \\
    \intertext{%
        I move the $X$ to the back of each summand to factor it out in the next
        step. The minus sign that I previously had seems to be wrong. So I made
        a sign mistake somewhere or got the order of $\mu$ and $\nu$ mixed up.
        Since I would rather focus on the other problems, I will just leave it
        as it, flipping the sign to make it fit.
    }
    &=
    2 (\partial_{[\mu} \Gamma^\rho_{\nu]\alpha}) X^\alpha
    + 2 \Gamma^\rho_{\lambda[\mu} \Gamma^\lambda_{\nu]\alpha} X^\alpha
    \intertext{%
        The $X$ can be factored out:
    }
    &=
    (2 \partial_{[\mu} \Gamma^\rho_{\nu]\alpha}
    + 2 \Gamma^\rho_{\lambda[\mu} \Gamma^\lambda_{\nu]\alpha}
    ) X^\alpha
    \intertext{%
        Expanding the brackets again gives the expression given on the problem
        set:
    }
    &=
    (\partial_\mu \Gamma^\rho_{\alpha\nu} - \partial_\nu
    \Gamma^\rho_{\alpha\mu} + \Gamma^\rho_{\lambda\mu}
    \Gamma^\lambda_{\alpha\nu} - \Gamma^\rho_{\lambda\nu}
    \Gamma^\lambda_{\alpha\mu}) X^\alpha \\
    &=
    R^\rho_{\alpha\mu\nu} X^\alpha
\end{align*}

\subsection{Verify the symmetry properties $R_{\alpha\beta\mu\nu} = -
R_{\alpha\beta\nu\mu} = - R_{\beta\alpha\mu\nu} = R_{\mu\nu\alpha\beta}$}

To avoid writing too much, I'll use the bracket notation first:
\begin{align*}
    R_{\alpha\beta\mu\nu}
    &= \frac 12 (\partial_\beta \partial_\mu g_{\alpha\nu} + \partial_\alpha
    \partial_\nu g_{\beta\mu} - \partial_\beta \partial_\nu g_{\alpha\mu} -
    \partial_\alpha \partial_\mu g_{\beta\nu})
    + g_{\lambda\delta} \cdot (
    \Gamma^\lambda_{\beta\mu} \Gamma^\delta_{\alpha\nu}
    - \Gamma^\lambda_{\beta\nu} \Gamma^\delta_{\alpha\mu}
    ) \\
    &= \partial_\beta \partial_{[\mu} g_{\nu]\alpha}
    + \partial_\alpha \partial_{[\mu} g_{\nu]\beta}
    + 2 g_{\lambda\delta} \Gamma^\lambda_{\beta[\mu} \Gamma^\delta_{\nu]\alpha}
\end{align*}

From that, the antisymmetry in the first and last two indices, respectively,
can be seen relatively easy: It is antisymmetric in $\mu$ and $\nu$ since they
only appear paired up in brackets. The antisymmetry in $\alpha$ and $\beta$ can
be seen as follows: When $\alpha$ and $\beta$ are exchanged, the summands and
the $\Gamma$ factors each have to be exchanged. Now $\mu$ and $\nu$ are in the
wrong order. To get them back, you have to exchange them, introducing the
needed minus sign. Addition and multiplication commutes, but antisymmetrization
anticommutes.

The exchange of $(\alpha, \beta) \leftrightarrow (\mu, \nu)$ can be shown by
expanding everything. I have not noticed a shorter way.
\begin{align*}
    R_{\alpha\beta\mu\nu}
    &= \partial_\beta \partial_{[\mu} g_{\nu]\alpha}
    + \partial_\alpha \partial_{[\mu} g_{\nu]\beta}
    + 2 g_{\lambda\delta} \Gamma^\lambda_{\beta[\mu} \Gamma^\delta_{\nu]\alpha}
    \\
    &=
    \frac 12 \partial_\beta \partial_\mu g_{\alpha\nu}
    + \frac 12 \partial_\alpha \partial_\nu g_{\beta\mu}
    - \frac 12 \partial_\beta \partial_\nu g_{\alpha\mu}
    - \frac 12 \partial_\alpha \partial_\mu g_{\beta\nu}
    + g_{\lambda\delta} \cdot (
    \Gamma^\lambda_{\beta\mu} \Gamma^\delta_{\alpha\nu}
    - \Gamma^\lambda_{\beta\nu} \Gamma^\delta_{\alpha\mu}
    ) \\
    &=
    \frac 12 \partial_\alpha \partial_\nu g_{\beta\mu}
    + \frac 12 \partial_\beta \partial_\mu g_{\alpha\nu}
    - \frac 12 \partial_\beta \partial_\nu g_{\alpha\mu}
    - \frac 12 \partial_\alpha \partial_\mu g_{\beta\nu}
    + g_{\lambda\delta} \cdot (
    \Gamma^\delta_{\alpha\nu} \Gamma^\lambda_{\beta\mu} 
    - \Gamma^\delta_{\alpha\mu} \Gamma^\lambda_{\beta\nu} 
    ) \\
    &=
      \frac 12 \partial_\nu \partial_\alpha g_{\beta \mu}
    - \frac 12 \partial_\nu \partial_\beta  g_{\alpha\mu}
    + \frac 12 \partial_\mu \partial_\beta  g_{\alpha\nu}
    - \frac 12 \partial_\mu \partial_\alpha g_{\beta \nu}
    + g_{\lambda\delta} \cdot (
    \Gamma^\delta_{\nu\alpha} \Gamma^\lambda_{\beta\mu} 
    - \Gamma^\delta_{\mu\alpha} \Gamma^\lambda_{\beta\nu} 
    ) \\
    &= \partial_\nu \partial_{[\alpha} g_{\beta]\mu}
    + \partial_\mu \partial_{[\alpha} g_{\beta]\nu}
    + 2 g_{\lambda\delta} \Gamma^\lambda_{\nu[\alpha} \Gamma^\delta_{\beta]\mu}
    \\
    &=
    R_{\mu\nu\alpha\beta}
\end{align*}

\subsection{Also show: $R_{\alpha[\beta\mu\nu]} = 0$}

\IfFileExists{\bibliographyfile}{
    \printbibliography
}{}

\end{document}

% vim: spell spelllang=en tw=79

% Copyright © 2012-2013 Martin Ueding <dev@martin-ueding.de>

% This is my general purpose LaTeX header file for writing German documents.
% Ideally, you include this using a simple ``% Copyright © 2012-2013 Martin Ueding <dev@martin-ueding.de>

% This is my general purpose LaTeX header file for writing German documents.
% Ideally, you include this using a simple ``% Copyright © 2012-2013 Martin Ueding <dev@martin-ueding.de>

% This is my general purpose LaTeX header file for writing German documents.
% Ideally, you include this using a simple ``\input{header.tex}`` in your main
% document and start with ``\title`` and ``\begin{document}`` afterwards.

% If you need to add additional packages, I recommend not doing this in this
% file, but in your main document. That way, you can just drop in a new
% ``header.tex`` and get all the new commands without having to merge manually.

% Since this file encorporates a CC-BY-SA fragment, this whole files is
% licensed under the CC-BY-SA license.

\documentclass[11pt, ngerman, fleqn, DIV=15, headinclude]{scrartcl}

\usepackage{graphicx}

\setkomafont{caption}{\sffamily}
\setkomafont{captionlabel}{\usekomafont{caption}}

%%%%%%%%%%%%%%%%%%%%%%%%%%%%%%%%%%%%%%%%%%%%%%%%%%%%%%%%%%%%%%%%%%%%%%%%%%%%%%%
%                                Locale, date                                 %
%%%%%%%%%%%%%%%%%%%%%%%%%%%%%%%%%%%%%%%%%%%%%%%%%%%%%%%%%%%%%%%%%%%%%%%%%%%%%%%

\usepackage{babel}
\usepackage[iso]{isodate}

%%%%%%%%%%%%%%%%%%%%%%%%%%%%%%%%%%%%%%%%%%%%%%%%%%%%%%%%%%%%%%%%%%%%%%%%%%%%%%%
%                          Margins and other spacing                          %
%%%%%%%%%%%%%%%%%%%%%%%%%%%%%%%%%%%%%%%%%%%%%%%%%%%%%%%%%%%%%%%%%%%%%%%%%%%%%%%

\usepackage[parfill]{parskip}
\usepackage{setspace}
\usepackage[activate]{microtype}

\setlength{\columnsep}{2cm}

%%%%%%%%%%%%%%%%%%%%%%%%%%%%%%%%%%%%%%%%%%%%%%%%%%%%%%%%%%%%%%%%%%%%%%%%%%%%%%%
%                                    Color                                    %
%%%%%%%%%%%%%%%%%%%%%%%%%%%%%%%%%%%%%%%%%%%%%%%%%%%%%%%%%%%%%%%%%%%%%%%%%%%%%%%

\usepackage[usenames, dvipsnames]{xcolor}

\colorlet{darkred}{red!70!black}
\colorlet{darkblue}{blue!70!black}
\colorlet{darkgreen}{green!40!black}

%%%%%%%%%%%%%%%%%%%%%%%%%%%%%%%%%%%%%%%%%%%%%%%%%%%%%%%%%%%%%%%%%%%%%%%%%%%%%%%
%                         Font and font like settings                         %
%%%%%%%%%%%%%%%%%%%%%%%%%%%%%%%%%%%%%%%%%%%%%%%%%%%%%%%%%%%%%%%%%%%%%%%%%%%%%%%

% This replaces all fonts with Bitstream Charter, Bitstream Vera Sans and
% Bitstream Vera Mono. Math will be rendered in Charter.
\usepackage[charter, greekuppercase=italicized]{mathdesign}
\usepackage{beramono}
\usepackage{berasans}

% Bold, sans-serif tensors. This fragment is taken from “egreg” from
% http://tex.stackexchange.com/a/82747/8945 and licensed under `CC-BY-SA
% <https://creativecommons.org/licenses/by-sa/3.0/>`_.
\usepackage{bm}
\DeclareMathAlphabet{\mathsfit}{\encodingdefault}{\sfdefault}{m}{sl}
\SetMathAlphabet{\mathsfit}{bold}{\encodingdefault}{\sfdefault}{bx}{sl}
\newcommand{\tens}[1]{\bm{\mathsfit{#1}}}

% Bold vectors.
\renewcommand{\vec}[1]{\boldsymbol{#1}}

%%%%%%%%%%%%%%%%%%%%%%%%%%%%%%%%%%%%%%%%%%%%%%%%%%%%%%%%%%%%%%%%%%%%%%%%%%%%%%%
%                               Input encoding                                %
%%%%%%%%%%%%%%%%%%%%%%%%%%%%%%%%%%%%%%%%%%%%%%%%%%%%%%%%%%%%%%%%%%%%%%%%%%%%%%%

\usepackage[T1]{fontenc}
\usepackage[utf8]{inputenc}

%%%%%%%%%%%%%%%%%%%%%%%%%%%%%%%%%%%%%%%%%%%%%%%%%%%%%%%%%%%%%%%%%%%%%%%%%%%%%%%
%                         Hyperrefs and PDF metadata                          %
%%%%%%%%%%%%%%%%%%%%%%%%%%%%%%%%%%%%%%%%%%%%%%%%%%%%%%%%%%%%%%%%%%%%%%%%%%%%%%%

\usepackage{hyperref}

% This sets the author in the properties of the PDF as well. If you want to
% change it, just override it with another ``\hypersetup`` call.
\hypersetup{
	breaklinks=false,
	citecolor=darkgreen,
	colorlinks=true,
	linkcolor=darkblue,
	menucolor=black,
	pdfauthor={Martin Ueding},
	urlcolor=darkblue,
}

%%%%%%%%%%%%%%%%%%%%%%%%%%%%%%%%%%%%%%%%%%%%%%%%%%%%%%%%%%%%%%%%%%%%%%%%%%%%%%%
%                               Math Operators                                %
%%%%%%%%%%%%%%%%%%%%%%%%%%%%%%%%%%%%%%%%%%%%%%%%%%%%%%%%%%%%%%%%%%%%%%%%%%%%%%%

% AMS environments like ``align`` and theorems like ``proof``.
\usepackage{amsmath}
\usepackage{amsthm}

% Common math constructs like partial derivatives.
\usepackage{commath}

% Physical units.
\usepackage[output-decimal-marker={,}]{siunitx}

% Since I use mathdesign with italic uppercase greek characters, the Ohm unit will be displayed with an italic Ω by default. Units have to be roman, so this forces it the right way.
\DeclareSIUnit{\ohm}{$\Omegaup$}
\DeclareSIUnit{\division}{DIV}
\DeclareSIUnit{\voltss}{$\mathrm{V_{SS}}$}

% Word like operators.
\DeclareMathOperator{\acosh}{arcosh}
\DeclareMathOperator{\arcosh}{arcosh}
\DeclareMathOperator{\arcsinh}{arsinh}
\DeclareMathOperator{\arsinh}{arsinh}
\DeclareMathOperator{\asinh}{arsinh}
\DeclareMathOperator{\card}{card}
\DeclareMathOperator{\csch}{csch}
\DeclareMathOperator{\diam}{diam}
\DeclareMathOperator{\sech}{sech}
\renewcommand{\Im}{\mathop{{}\mathrm{Im}}\nolimits}
\renewcommand{\Re}{\mathop{{}\mathrm{Re}}\nolimits}

% Fourier transform.
\DeclareMathOperator{\fourier}{\ensuremath{\mathcal{F}}}

% Roman versions of “e” and “i” to serve as Euler's number and the imaginary
% constant.
\newcommand{\eup}{\mathrm e}
\newcommand{\iup}{\mathrm i}

% Symbols for the various mathematical fields (natural numbers, integers,
% rational numbers, real numbers, complex numbers).
\newcommand{\C}{\ensuremath{\mathbb C}}
\newcommand{\N}{\ensuremath{\mathbb N}}
\newcommand{\Q}{\ensuremath{\mathbb Q}}
\newcommand{\R}{\ensuremath{\mathbb R}}
\newcommand{\Z}{\ensuremath{\mathbb Z}}

% Shape like operators.
\DeclareMathOperator{\dalambert}{\Box}
\DeclareMathOperator{\laplace}{\bigtriangleup}
\newcommand{\curl}{\vnabla \times}
\newcommand{\divergence}[1]{\inner{\vnabla}{#1}}
\newcommand{\Divergence}[1]{\Inner{\vnabla}{#1}}
\newcommand{\vnabla}{\vec \nabla}

\newcommand{\half}{\frac 12}

% Unit vector (German „Einheitsvektor“).
\newcommand{\ev}{\hat{\vec e}}

% Mathematician's notation for the inner (scalar, dot) product.
\newcommand{\bracket}[1]{\langle #1 \rangle}
\newcommand{\Bracket}[1]{\left\langle #1 \right\rangle}
\newcommand{\inner}[2]{\bracket{#1, #2}}
\newcommand{\Inner}[2]{\Bracket{#1, #2}}

% Placeholders.
\newcommand{\fehlt}{\textcolor{darkred}{Hier fehlen noch Inhalte.}}
\newcommand{\messwert}{\textcolor{blue}{\square}}
\newcommand{\punkte}{\phantom{xxxxx}}

% Separator for equations on a single line.
\newcommand{\eqnsep}{,\quad}

% Quantum Mechanics.
\usepackage{braket}

% Thermodynamic partial derivative.
\newcommand\tdpd[3]{\del{\dpd{#1}{#2}}_{#3}}

%%%%%%%%%%%%%%%%%%%%%%%%%%%%%%%%%%%%%%%%%%%%%%%%%%%%%%%%%%%%%%%%%%%%%%%%%%%%%%%
%                                  Headings                                   %
%%%%%%%%%%%%%%%%%%%%%%%%%%%%%%%%%%%%%%%%%%%%%%%%%%%%%%%%%%%%%%%%%%%%%%%%%%%%%%%

% This will set fancy headings to the top of the page. The page number will be
% accompanied by the total number of pages. That way, you will know if any page
% is missing.
%
% If you do not want this for your document, you can just use
% ``\pagestyle{plain}``.

\usepackage{scrpage2}

\pagestyle{scrheadings}
\automark{section}
\chead{}
\ihead{}
\ohead{\rightmark}
\setheadsepline{.4pt}

%%%%%%%%%%%%%%%%%%%%%%%%%%%%%%%%%%%%%%%%%%%%%%%%%%%%%%%%%%%%%%%%%%%%%%%%%%%%%%%
%                            Bibliography (BibTeX)                            %
%%%%%%%%%%%%%%%%%%%%%%%%%%%%%%%%%%%%%%%%%%%%%%%%%%%%%%%%%%%%%%%%%%%%%%%%%%%%%%%

\newcommand{\bibliographyfile}{../../zentrale_BibTeX/Central}

\usepackage[
    backend=bibtex,
    style=authoryear-icomp,
    %isbn=false,
    %pagetracker=false,
    %maxbibnames=50,
    %maxcitenames=2,
    %autocite=inline,
    %block=space,
    %backref=false,
    %backrefstyle=three+,
    %date=short,
    hyperref=true
]{biblatex}

\setlength{\bibitemsep}{.7em}
\setlength{\bibhang}{4ex}

\IfFileExists{\bibliographyfile}{
    \bibliography{\bibliographyfile}
}{}

%%%%%%%%%%%%%%%%%%%%%%%%%%%%%%%%%%%%%%%%%%%%%%%%%%%%%%%%%%%%%%%%%%%%%%%%%%%%%%%
%                                Abbreviations                                %
%%%%%%%%%%%%%%%%%%%%%%%%%%%%%%%%%%%%%%%%%%%%%%%%%%%%%%%%%%%%%%%%%%%%%%%%%%%%%%%

\newcommand{\dhabk}{\mbox{d.\,h.}}

%%%%%%%%%%%%%%%%%%%%%%%%%%%%%%%%%%%%%%%%%%%%%%%%%%%%%%%%%%%%%%%%%%%%%%%%%%%%%%%
%                                  Licences                                   %
%%%%%%%%%%%%%%%%%%%%%%%%%%%%%%%%%%%%%%%%%%%%%%%%%%%%%%%%%%%%%%%%%%%%%%%%%%%%%%%

\usepackage{ccicons}

\newcommand{\ccbysadetext}{%
	\begin{small}
		Dieses Werk bzw. Inhalt steht unter einer
		\href{http://creativecommons.org/licenses/by-sa/3.0/deed.de}{%
			Creative Commons Namensnennung - Weitergabe unter gleichen
		Bedingungen 3.0 Unported Lizenz}.
	\end{small}
}

\newcommand{\ccbysadetitle}{%
	Lizenz: \href{http://creativecommons.org/licenses/by-sa/3.0/deed.de}
	{CC-BY-SA 3.0 \ccbysa}
}
`` in your main
% document and start with ``\title`` and ``\begin{document}`` afterwards.

% If you need to add additional packages, I recommend not doing this in this
% file, but in your main document. That way, you can just drop in a new
% ``header.tex`` and get all the new commands without having to merge manually.

% Since this file encorporates a CC-BY-SA fragment, this whole files is
% licensed under the CC-BY-SA license.

\documentclass[11pt, ngerman, fleqn, DIV=15, headinclude]{scrartcl}

\usepackage{graphicx}

\setkomafont{caption}{\sffamily}
\setkomafont{captionlabel}{\usekomafont{caption}}

%%%%%%%%%%%%%%%%%%%%%%%%%%%%%%%%%%%%%%%%%%%%%%%%%%%%%%%%%%%%%%%%%%%%%%%%%%%%%%%
%                                Locale, date                                 %
%%%%%%%%%%%%%%%%%%%%%%%%%%%%%%%%%%%%%%%%%%%%%%%%%%%%%%%%%%%%%%%%%%%%%%%%%%%%%%%

\usepackage{babel}
\usepackage[iso]{isodate}

%%%%%%%%%%%%%%%%%%%%%%%%%%%%%%%%%%%%%%%%%%%%%%%%%%%%%%%%%%%%%%%%%%%%%%%%%%%%%%%
%                          Margins and other spacing                          %
%%%%%%%%%%%%%%%%%%%%%%%%%%%%%%%%%%%%%%%%%%%%%%%%%%%%%%%%%%%%%%%%%%%%%%%%%%%%%%%

\usepackage[parfill]{parskip}
\usepackage{setspace}
\usepackage[activate]{microtype}

\setlength{\columnsep}{2cm}

%%%%%%%%%%%%%%%%%%%%%%%%%%%%%%%%%%%%%%%%%%%%%%%%%%%%%%%%%%%%%%%%%%%%%%%%%%%%%%%
%                                    Color                                    %
%%%%%%%%%%%%%%%%%%%%%%%%%%%%%%%%%%%%%%%%%%%%%%%%%%%%%%%%%%%%%%%%%%%%%%%%%%%%%%%

\usepackage[usenames, dvipsnames]{xcolor}

\colorlet{darkred}{red!70!black}
\colorlet{darkblue}{blue!70!black}
\colorlet{darkgreen}{green!40!black}

%%%%%%%%%%%%%%%%%%%%%%%%%%%%%%%%%%%%%%%%%%%%%%%%%%%%%%%%%%%%%%%%%%%%%%%%%%%%%%%
%                         Font and font like settings                         %
%%%%%%%%%%%%%%%%%%%%%%%%%%%%%%%%%%%%%%%%%%%%%%%%%%%%%%%%%%%%%%%%%%%%%%%%%%%%%%%

% This replaces all fonts with Bitstream Charter, Bitstream Vera Sans and
% Bitstream Vera Mono. Math will be rendered in Charter.
\usepackage[charter, greekuppercase=italicized]{mathdesign}
\usepackage{beramono}
\usepackage{berasans}

% Bold, sans-serif tensors. This fragment is taken from “egreg” from
% http://tex.stackexchange.com/a/82747/8945 and licensed under `CC-BY-SA
% <https://creativecommons.org/licenses/by-sa/3.0/>`_.
\usepackage{bm}
\DeclareMathAlphabet{\mathsfit}{\encodingdefault}{\sfdefault}{m}{sl}
\SetMathAlphabet{\mathsfit}{bold}{\encodingdefault}{\sfdefault}{bx}{sl}
\newcommand{\tens}[1]{\bm{\mathsfit{#1}}}

% Bold vectors.
\renewcommand{\vec}[1]{\boldsymbol{#1}}

%%%%%%%%%%%%%%%%%%%%%%%%%%%%%%%%%%%%%%%%%%%%%%%%%%%%%%%%%%%%%%%%%%%%%%%%%%%%%%%
%                               Input encoding                                %
%%%%%%%%%%%%%%%%%%%%%%%%%%%%%%%%%%%%%%%%%%%%%%%%%%%%%%%%%%%%%%%%%%%%%%%%%%%%%%%

\usepackage[T1]{fontenc}
\usepackage[utf8]{inputenc}

%%%%%%%%%%%%%%%%%%%%%%%%%%%%%%%%%%%%%%%%%%%%%%%%%%%%%%%%%%%%%%%%%%%%%%%%%%%%%%%
%                         Hyperrefs and PDF metadata                          %
%%%%%%%%%%%%%%%%%%%%%%%%%%%%%%%%%%%%%%%%%%%%%%%%%%%%%%%%%%%%%%%%%%%%%%%%%%%%%%%

\usepackage{hyperref}

% This sets the author in the properties of the PDF as well. If you want to
% change it, just override it with another ``\hypersetup`` call.
\hypersetup{
	breaklinks=false,
	citecolor=darkgreen,
	colorlinks=true,
	linkcolor=darkblue,
	menucolor=black,
	pdfauthor={Martin Ueding},
	urlcolor=darkblue,
}

%%%%%%%%%%%%%%%%%%%%%%%%%%%%%%%%%%%%%%%%%%%%%%%%%%%%%%%%%%%%%%%%%%%%%%%%%%%%%%%
%                               Math Operators                                %
%%%%%%%%%%%%%%%%%%%%%%%%%%%%%%%%%%%%%%%%%%%%%%%%%%%%%%%%%%%%%%%%%%%%%%%%%%%%%%%

% AMS environments like ``align`` and theorems like ``proof``.
\usepackage{amsmath}
\usepackage{amsthm}

% Common math constructs like partial derivatives.
\usepackage{commath}

% Physical units.
\usepackage[output-decimal-marker={,}]{siunitx}

% Since I use mathdesign with italic uppercase greek characters, the Ohm unit will be displayed with an italic Ω by default. Units have to be roman, so this forces it the right way.
\DeclareSIUnit{\ohm}{$\Omegaup$}
\DeclareSIUnit{\division}{DIV}
\DeclareSIUnit{\voltss}{$\mathrm{V_{SS}}$}

% Word like operators.
\DeclareMathOperator{\acosh}{arcosh}
\DeclareMathOperator{\arcosh}{arcosh}
\DeclareMathOperator{\arcsinh}{arsinh}
\DeclareMathOperator{\arsinh}{arsinh}
\DeclareMathOperator{\asinh}{arsinh}
\DeclareMathOperator{\card}{card}
\DeclareMathOperator{\csch}{csch}
\DeclareMathOperator{\diam}{diam}
\DeclareMathOperator{\sech}{sech}
\renewcommand{\Im}{\mathop{{}\mathrm{Im}}\nolimits}
\renewcommand{\Re}{\mathop{{}\mathrm{Re}}\nolimits}

% Fourier transform.
\DeclareMathOperator{\fourier}{\ensuremath{\mathcal{F}}}

% Roman versions of “e” and “i” to serve as Euler's number and the imaginary
% constant.
\newcommand{\eup}{\mathrm e}
\newcommand{\iup}{\mathrm i}

% Symbols for the various mathematical fields (natural numbers, integers,
% rational numbers, real numbers, complex numbers).
\newcommand{\C}{\ensuremath{\mathbb C}}
\newcommand{\N}{\ensuremath{\mathbb N}}
\newcommand{\Q}{\ensuremath{\mathbb Q}}
\newcommand{\R}{\ensuremath{\mathbb R}}
\newcommand{\Z}{\ensuremath{\mathbb Z}}

% Shape like operators.
\DeclareMathOperator{\dalambert}{\Box}
\DeclareMathOperator{\laplace}{\bigtriangleup}
\newcommand{\curl}{\vnabla \times}
\newcommand{\divergence}[1]{\inner{\vnabla}{#1}}
\newcommand{\Divergence}[1]{\Inner{\vnabla}{#1}}
\newcommand{\vnabla}{\vec \nabla}

\newcommand{\half}{\frac 12}

% Unit vector (German „Einheitsvektor“).
\newcommand{\ev}{\hat{\vec e}}

% Mathematician's notation for the inner (scalar, dot) product.
\newcommand{\bracket}[1]{\langle #1 \rangle}
\newcommand{\Bracket}[1]{\left\langle #1 \right\rangle}
\newcommand{\inner}[2]{\bracket{#1, #2}}
\newcommand{\Inner}[2]{\Bracket{#1, #2}}

% Placeholders.
\newcommand{\fehlt}{\textcolor{darkred}{Hier fehlen noch Inhalte.}}
\newcommand{\messwert}{\textcolor{blue}{\square}}
\newcommand{\punkte}{\phantom{xxxxx}}

% Separator for equations on a single line.
\newcommand{\eqnsep}{,\quad}

% Quantum Mechanics.
\usepackage{braket}

% Thermodynamic partial derivative.
\newcommand\tdpd[3]{\del{\dpd{#1}{#2}}_{#3}}

%%%%%%%%%%%%%%%%%%%%%%%%%%%%%%%%%%%%%%%%%%%%%%%%%%%%%%%%%%%%%%%%%%%%%%%%%%%%%%%
%                                  Headings                                   %
%%%%%%%%%%%%%%%%%%%%%%%%%%%%%%%%%%%%%%%%%%%%%%%%%%%%%%%%%%%%%%%%%%%%%%%%%%%%%%%

% This will set fancy headings to the top of the page. The page number will be
% accompanied by the total number of pages. That way, you will know if any page
% is missing.
%
% If you do not want this for your document, you can just use
% ``\pagestyle{plain}``.

\usepackage{scrpage2}

\pagestyle{scrheadings}
\automark{section}
\chead{}
\ihead{}
\ohead{\rightmark}
\setheadsepline{.4pt}

%%%%%%%%%%%%%%%%%%%%%%%%%%%%%%%%%%%%%%%%%%%%%%%%%%%%%%%%%%%%%%%%%%%%%%%%%%%%%%%
%                            Bibliography (BibTeX)                            %
%%%%%%%%%%%%%%%%%%%%%%%%%%%%%%%%%%%%%%%%%%%%%%%%%%%%%%%%%%%%%%%%%%%%%%%%%%%%%%%

\newcommand{\bibliographyfile}{../../zentrale_BibTeX/Central}

\usepackage[
    backend=bibtex,
    style=authoryear-icomp,
    %isbn=false,
    %pagetracker=false,
    %maxbibnames=50,
    %maxcitenames=2,
    %autocite=inline,
    %block=space,
    %backref=false,
    %backrefstyle=three+,
    %date=short,
    hyperref=true
]{biblatex}

\setlength{\bibitemsep}{.7em}
\setlength{\bibhang}{4ex}

\IfFileExists{\bibliographyfile}{
    \bibliography{\bibliographyfile}
}{}

%%%%%%%%%%%%%%%%%%%%%%%%%%%%%%%%%%%%%%%%%%%%%%%%%%%%%%%%%%%%%%%%%%%%%%%%%%%%%%%
%                                Abbreviations                                %
%%%%%%%%%%%%%%%%%%%%%%%%%%%%%%%%%%%%%%%%%%%%%%%%%%%%%%%%%%%%%%%%%%%%%%%%%%%%%%%

\newcommand{\dhabk}{\mbox{d.\,h.}}

%%%%%%%%%%%%%%%%%%%%%%%%%%%%%%%%%%%%%%%%%%%%%%%%%%%%%%%%%%%%%%%%%%%%%%%%%%%%%%%
%                                  Licences                                   %
%%%%%%%%%%%%%%%%%%%%%%%%%%%%%%%%%%%%%%%%%%%%%%%%%%%%%%%%%%%%%%%%%%%%%%%%%%%%%%%

\usepackage{ccicons}

\newcommand{\ccbysadetext}{%
	\begin{small}
		Dieses Werk bzw. Inhalt steht unter einer
		\href{http://creativecommons.org/licenses/by-sa/3.0/deed.de}{%
			Creative Commons Namensnennung - Weitergabe unter gleichen
		Bedingungen 3.0 Unported Lizenz}.
	\end{small}
}

\newcommand{\ccbysadetitle}{%
	Lizenz: \href{http://creativecommons.org/licenses/by-sa/3.0/deed.de}
	{CC-BY-SA 3.0 \ccbysa}
}
`` in your main
% document and start with ``\title`` and ``\begin{document}`` afterwards.

% If you need to add additional packages, I recommend not doing this in this
% file, but in your main document. That way, you can just drop in a new
% ``header.tex`` and get all the new commands without having to merge manually.

% Since this file encorporates a CC-BY-SA fragment, this whole files is
% licensed under the CC-BY-SA license.

\documentclass[11pt, ngerman, fleqn, DIV=15, headinclude]{scrartcl}

\usepackage{graphicx}

\setkomafont{caption}{\sffamily}
\setkomafont{captionlabel}{\usekomafont{caption}}

%%%%%%%%%%%%%%%%%%%%%%%%%%%%%%%%%%%%%%%%%%%%%%%%%%%%%%%%%%%%%%%%%%%%%%%%%%%%%%%
%                                Locale, date                                 %
%%%%%%%%%%%%%%%%%%%%%%%%%%%%%%%%%%%%%%%%%%%%%%%%%%%%%%%%%%%%%%%%%%%%%%%%%%%%%%%

\usepackage{babel}
\usepackage[iso]{isodate}

%%%%%%%%%%%%%%%%%%%%%%%%%%%%%%%%%%%%%%%%%%%%%%%%%%%%%%%%%%%%%%%%%%%%%%%%%%%%%%%
%                          Margins and other spacing                          %
%%%%%%%%%%%%%%%%%%%%%%%%%%%%%%%%%%%%%%%%%%%%%%%%%%%%%%%%%%%%%%%%%%%%%%%%%%%%%%%

\usepackage[parfill]{parskip}
\usepackage{setspace}
\usepackage[activate]{microtype}

\setlength{\columnsep}{2cm}

%%%%%%%%%%%%%%%%%%%%%%%%%%%%%%%%%%%%%%%%%%%%%%%%%%%%%%%%%%%%%%%%%%%%%%%%%%%%%%%
%                                    Color                                    %
%%%%%%%%%%%%%%%%%%%%%%%%%%%%%%%%%%%%%%%%%%%%%%%%%%%%%%%%%%%%%%%%%%%%%%%%%%%%%%%

\usepackage[usenames, dvipsnames]{xcolor}

\colorlet{darkred}{red!70!black}
\colorlet{darkblue}{blue!70!black}
\colorlet{darkgreen}{green!40!black}

%%%%%%%%%%%%%%%%%%%%%%%%%%%%%%%%%%%%%%%%%%%%%%%%%%%%%%%%%%%%%%%%%%%%%%%%%%%%%%%
%                         Font and font like settings                         %
%%%%%%%%%%%%%%%%%%%%%%%%%%%%%%%%%%%%%%%%%%%%%%%%%%%%%%%%%%%%%%%%%%%%%%%%%%%%%%%

% This replaces all fonts with Bitstream Charter, Bitstream Vera Sans and
% Bitstream Vera Mono. Math will be rendered in Charter.
\usepackage[charter, greekuppercase=italicized]{mathdesign}
\usepackage{beramono}
\usepackage{berasans}

% Bold, sans-serif tensors. This fragment is taken from “egreg” from
% http://tex.stackexchange.com/a/82747/8945 and licensed under `CC-BY-SA
% <https://creativecommons.org/licenses/by-sa/3.0/>`_.
\usepackage{bm}
\DeclareMathAlphabet{\mathsfit}{\encodingdefault}{\sfdefault}{m}{sl}
\SetMathAlphabet{\mathsfit}{bold}{\encodingdefault}{\sfdefault}{bx}{sl}
\newcommand{\tens}[1]{\bm{\mathsfit{#1}}}

% Bold vectors.
\renewcommand{\vec}[1]{\boldsymbol{#1}}

%%%%%%%%%%%%%%%%%%%%%%%%%%%%%%%%%%%%%%%%%%%%%%%%%%%%%%%%%%%%%%%%%%%%%%%%%%%%%%%
%                               Input encoding                                %
%%%%%%%%%%%%%%%%%%%%%%%%%%%%%%%%%%%%%%%%%%%%%%%%%%%%%%%%%%%%%%%%%%%%%%%%%%%%%%%

\usepackage[T1]{fontenc}
\usepackage[utf8]{inputenc}

%%%%%%%%%%%%%%%%%%%%%%%%%%%%%%%%%%%%%%%%%%%%%%%%%%%%%%%%%%%%%%%%%%%%%%%%%%%%%%%
%                         Hyperrefs and PDF metadata                          %
%%%%%%%%%%%%%%%%%%%%%%%%%%%%%%%%%%%%%%%%%%%%%%%%%%%%%%%%%%%%%%%%%%%%%%%%%%%%%%%

\usepackage{hyperref}

% This sets the author in the properties of the PDF as well. If you want to
% change it, just override it with another ``\hypersetup`` call.
\hypersetup{
	breaklinks=false,
	citecolor=darkgreen,
	colorlinks=true,
	linkcolor=darkblue,
	menucolor=black,
	pdfauthor={Martin Ueding},
	urlcolor=darkblue,
}

%%%%%%%%%%%%%%%%%%%%%%%%%%%%%%%%%%%%%%%%%%%%%%%%%%%%%%%%%%%%%%%%%%%%%%%%%%%%%%%
%                               Math Operators                                %
%%%%%%%%%%%%%%%%%%%%%%%%%%%%%%%%%%%%%%%%%%%%%%%%%%%%%%%%%%%%%%%%%%%%%%%%%%%%%%%

% AMS environments like ``align`` and theorems like ``proof``.
\usepackage{amsmath}
\usepackage{amsthm}

% Common math constructs like partial derivatives.
\usepackage{commath}

% Physical units.
\usepackage[output-decimal-marker={,}]{siunitx}

% Since I use mathdesign with italic uppercase greek characters, the Ohm unit will be displayed with an italic Ω by default. Units have to be roman, so this forces it the right way.
\DeclareSIUnit{\ohm}{$\Omegaup$}
\DeclareSIUnit{\division}{DIV}
\DeclareSIUnit{\voltss}{$\mathrm{V_{SS}}$}

% Word like operators.
\DeclareMathOperator{\acosh}{arcosh}
\DeclareMathOperator{\arcosh}{arcosh}
\DeclareMathOperator{\arcsinh}{arsinh}
\DeclareMathOperator{\arsinh}{arsinh}
\DeclareMathOperator{\asinh}{arsinh}
\DeclareMathOperator{\card}{card}
\DeclareMathOperator{\csch}{csch}
\DeclareMathOperator{\diam}{diam}
\DeclareMathOperator{\sech}{sech}
\renewcommand{\Im}{\mathop{{}\mathrm{Im}}\nolimits}
\renewcommand{\Re}{\mathop{{}\mathrm{Re}}\nolimits}

% Fourier transform.
\DeclareMathOperator{\fourier}{\ensuremath{\mathcal{F}}}

% Roman versions of “e” and “i” to serve as Euler's number and the imaginary
% constant.
\newcommand{\eup}{\mathrm e}
\newcommand{\iup}{\mathrm i}

% Symbols for the various mathematical fields (natural numbers, integers,
% rational numbers, real numbers, complex numbers).
\newcommand{\C}{\ensuremath{\mathbb C}}
\newcommand{\N}{\ensuremath{\mathbb N}}
\newcommand{\Q}{\ensuremath{\mathbb Q}}
\newcommand{\R}{\ensuremath{\mathbb R}}
\newcommand{\Z}{\ensuremath{\mathbb Z}}

% Shape like operators.
\DeclareMathOperator{\dalambert}{\Box}
\DeclareMathOperator{\laplace}{\bigtriangleup}
\newcommand{\curl}{\vnabla \times}
\newcommand{\divergence}[1]{\inner{\vnabla}{#1}}
\newcommand{\Divergence}[1]{\Inner{\vnabla}{#1}}
\newcommand{\vnabla}{\vec \nabla}

\newcommand{\half}{\frac 12}

% Unit vector (German „Einheitsvektor“).
\newcommand{\ev}{\hat{\vec e}}

% Mathematician's notation for the inner (scalar, dot) product.
\newcommand{\bracket}[1]{\langle #1 \rangle}
\newcommand{\Bracket}[1]{\left\langle #1 \right\rangle}
\newcommand{\inner}[2]{\bracket{#1, #2}}
\newcommand{\Inner}[2]{\Bracket{#1, #2}}

% Placeholders.
\newcommand{\fehlt}{\textcolor{darkred}{Hier fehlen noch Inhalte.}}
\newcommand{\messwert}{\textcolor{blue}{\square}}
\newcommand{\punkte}{\phantom{xxxxx}}

% Separator for equations on a single line.
\newcommand{\eqnsep}{,\quad}

% Quantum Mechanics.
\usepackage{braket}

% Thermodynamic partial derivative.
\newcommand\tdpd[3]{\del{\dpd{#1}{#2}}_{#3}}

%%%%%%%%%%%%%%%%%%%%%%%%%%%%%%%%%%%%%%%%%%%%%%%%%%%%%%%%%%%%%%%%%%%%%%%%%%%%%%%
%                                  Headings                                   %
%%%%%%%%%%%%%%%%%%%%%%%%%%%%%%%%%%%%%%%%%%%%%%%%%%%%%%%%%%%%%%%%%%%%%%%%%%%%%%%

% This will set fancy headings to the top of the page. The page number will be
% accompanied by the total number of pages. That way, you will know if any page
% is missing.
%
% If you do not want this for your document, you can just use
% ``\pagestyle{plain}``.

\usepackage{scrpage2}

\pagestyle{scrheadings}
\automark{section}
\chead{}
\ihead{}
\ohead{\rightmark}
\setheadsepline{.4pt}

%%%%%%%%%%%%%%%%%%%%%%%%%%%%%%%%%%%%%%%%%%%%%%%%%%%%%%%%%%%%%%%%%%%%%%%%%%%%%%%
%                            Bibliography (BibTeX)                            %
%%%%%%%%%%%%%%%%%%%%%%%%%%%%%%%%%%%%%%%%%%%%%%%%%%%%%%%%%%%%%%%%%%%%%%%%%%%%%%%

\newcommand{\bibliographyfile}{../../zentrale_BibTeX/Central}

\usepackage[
    backend=bibtex,
    style=authoryear-icomp,
    %isbn=false,
    %pagetracker=false,
    %maxbibnames=50,
    %maxcitenames=2,
    %autocite=inline,
    %block=space,
    %backref=false,
    %backrefstyle=three+,
    %date=short,
    hyperref=true
]{biblatex}

\setlength{\bibitemsep}{.7em}
\setlength{\bibhang}{4ex}

\IfFileExists{\bibliographyfile}{
    \bibliography{\bibliographyfile}
}{}

%%%%%%%%%%%%%%%%%%%%%%%%%%%%%%%%%%%%%%%%%%%%%%%%%%%%%%%%%%%%%%%%%%%%%%%%%%%%%%%
%                                Abbreviations                                %
%%%%%%%%%%%%%%%%%%%%%%%%%%%%%%%%%%%%%%%%%%%%%%%%%%%%%%%%%%%%%%%%%%%%%%%%%%%%%%%

\newcommand{\dhabk}{\mbox{d.\,h.}}

%%%%%%%%%%%%%%%%%%%%%%%%%%%%%%%%%%%%%%%%%%%%%%%%%%%%%%%%%%%%%%%%%%%%%%%%%%%%%%%
%                                  Licences                                   %
%%%%%%%%%%%%%%%%%%%%%%%%%%%%%%%%%%%%%%%%%%%%%%%%%%%%%%%%%%%%%%%%%%%%%%%%%%%%%%%

\usepackage{ccicons}

\newcommand{\ccbysadetext}{%
	\begin{small}
		Dieses Werk bzw. Inhalt steht unter einer
		\href{http://creativecommons.org/licenses/by-sa/3.0/deed.de}{%
			Creative Commons Namensnennung - Weitergabe unter gleichen
		Bedingungen 3.0 Unported Lizenz}.
	\end{small}
}

\newcommand{\ccbysadetitle}{%
	Lizenz: \href{http://creativecommons.org/licenses/by-sa/3.0/deed.de}
	{CC-BY-SA 3.0 \ccbysa}
}


\newcommand\problemset{1}

\hypersetup{
    pdftitle={physics754 Problem Set \problemset}
}

\newenvironment{aside}{\itshape\small}{}

%\subject{}
\title{physics754 -- Problem Set \problemset}
%\subtitle{}
\author{
    Martin Ueding \\ \small{\href{mailto:mu@martin-ueding.de}{mu@martin-ueding.de}}
}
\publishers{Group 5}

\begin{document}

\maketitle

While doing the exercises, I had a couple conceptional questions. They are
included in small italic font within the text. I would greatly appreciate it if
you could add a few remarks to each.

\section{A coordinate transformation}

\subsection{}

First of all, I will start by computing the components of the tensor $\pd{{x^\alpha}}{{y^\mu}}$:
\[
    \dpd{{x^\alpha}}{{y^\mu}}
    =
    \begin{pmatrix}
        1 & 0 & 0 & 0 \\
        0 & \sin(\theta) \cos(\phi) & \sin(\theta) \sin(\phi) & \cos(\theta) \\
        0 & r \cos(\theta) \cos(\phi) & r \cos(\theta) \sin(\phi) & - r \sin(\theta) \\
        0 & - r \sin(\theta) \sin(\phi) & r \sin(\theta) \cos(\phi) & 0
    \end{pmatrix}.
\]

With that, I can go through the components of $g$ and calculate each component.
I know that $g$ is symmetric, so I have only ten components to calculate. After
working with some trigonometric identities, I finally got this:
\[
    \tens g(\vec x)
    \sim
    \begin{pmatrix}
        1 & 0 & 0 & 0 \\
        0 & -1 & 0 & 0 \\
        0 & 0 & -r^2 & 0 \\
        0 & 0 & 0 & -r^2 \sin(\theta)^2
    \end{pmatrix}.
\]

\begin{aside}
    I wrote “$\tens g \sim$” since $\tens g$ is a tensor of valence (0, 2) and
    not a matrix with valence (1, 1). Would is be better to write “$g_{\mu\nu}
    =$” instead, although the left side would have two dangling indices,
    whereas the right side would not? Should one write it as “$g_{\mu\nu} =
    (\cdots)^\mu_\nu$” instead?
\end{aside}

\subsection{}

The electric field is given as
\[
    E_i(\vec x) = \frac{Q}{|x|^3} x_i.
\]

In the field tensor $\tens F$, the electric field is contained such that
$F_{0i} = E_i$. Therefore, I write
\[
    F_{0i}(\vec x) = \frac{Q}{|x|^3} x_i.
\]

From the antisymmetry, the other nonzero entries are $F_{i0}$.

\subsection{}

Now $\tens F$ will be transformed:
\begin{align*}
    \hat F_{\mu\nu}(\vec y)
    &= F_{\alpha\beta}(\vec y) \dpd{{x^\alpha}}{{y^\mu}} \dpd{{x^\beta}}{{y^\nu}}.
    \intertext{%
        Since most components of $\tens F$ are zero, I can restrict $\alpha$
        and $\beta$ such that only nonzero components remain:
    }
    &= F_{0i}(\vec y) \dpd{{x^0}}{{y^\mu}} \dpd{{x^i}}{{y^\nu}}.
    \intertext{%
        For the first partial derivative, the only nonzero part is the one for
        $\mu = 0$, which is just 1. This simplifies to:
    }
    \hat F_{0\nu}(\vec y)
    &= F_{0i}(\vec y) \dpd{{x^i}}{{y^\nu}}
    \intertext{%
        where the other nonzero components of $\hat{\tens F}$ are computed
        using the antisymmetry. Now I have summed over $i$ which gives (after
        trigonometry):
    }
    &= \frac{Q}{r^2} \begin{pmatrix}
        0 \\ 1 \\ 0 \\ 0
    \end{pmatrix}_\nu.
\end{align*}

From that, I conclude the following representation for the transformed field tensor:
\[
    \hat{\tens F}(\vec y)
    \sim
    \frac{Q}{r^2}
    \begin{pmatrix}
        0 & 1 & 0 & 0 \\
        -1 & 0 & 0 & 0 \\
        0 & 0 & 0 & 0 \\
        0 & 0 & 0 & 0
    \end{pmatrix}
\]

\section{Calculating with Christoffel-Symbols}

\subsection{}

\begin{aside}
    I have the hunch that $\nabla \tens g = 0$ means that the metric is
    constant along geodesics. Does something like that hold?
\end{aside}

\newcommand\csk[3]{\Gamma^{#1}_{{#2}{#3}}}

\begin{align*}
    \nabla_\mu g_{\alpha\beta}
    &= \partial_\mu g_{\alpha\beta} - g_{\lambda\beta} \csk\gamma\alpha\mu
    - g_{\alpha\lambda} \csk\lambda\beta\mu \\
    \intertext{%
        Now I will just insert the definition of the Christoffel-Symbols for
        both occurrences.
    }
    &= \partial_\mu g_{\alpha\beta}
    - \frac12 g_{\lambda\beta} g^{\lambda\delta} \cdot (\partial_\alpha
    g_{\delta\mu} + \partial_\mu g_{\delta\alpha} - \partial_\delta
    g_{\mu\alpha})
    - \frac12 g_{\alpha\lambda} g^{\lambda\delta} \cdot (\partial_\beta g_{\delta\mu} +
    \partial_\mu g_{\delta\beta} - \partial_\delta g_{\mu\beta}) \\
    \intertext{%
        I have used a “$\cdot$” here to distinguish the opening parentheses
        from a function evaluation. No inner product is intended. As a next
        step, I contract the two $\tens g$ in the last two terms to yield a
        $\tens \delta$-symbol.
    }
    &= \partial_\mu g_{\alpha\beta}
    - \frac12 \delta\beta^\delta \cdot (\partial_\alpha g_{\delta\mu} + \partial_\mu g_{\delta\alpha} - \partial_\delta g_{\mu\alpha})
    - \frac12 \delta_\alpha^\delta \cdot (\partial_\beta g_{\delta\mu} + \partial_\mu g_{\delta\beta} - \partial_\delta g_{\mu\beta}) \\
    \intertext{%
        Contracting the $\tens \delta$:
    }
    &= \partial_\mu g_{\alpha\beta}
    - \frac12 (\partial_\alpha g_{\beta\mu} + \partial_\mu g_{\beta\alpha} - \partial_\beta g_{\mu\alpha})
    - \frac12 (\partial_\beta g_{\alpha\mu} + \partial_\mu g_{\alpha\beta} - \partial_\alpha g_{\mu\beta}) \\
    \intertext{%
        Using the symmetry of $\tens g$, the first and last terms of other
        parentheses, respectively, cancel each other out. This leaves:
    }
    &= \partial_\mu g_{\alpha\beta}
    - \frac12 \partial_\mu g_{\beta\alpha}
    - \frac12 \partial_\mu g_{\alpha\beta} \\
    &= 0
\end{align*}

\subsection{Levi-Cività theorem}

\begin{aside}
    It has bothered me that $\partial_x$ is supposed to be a basis vector. It
    seems like an operator, and geometry is not quantum mechanics.

    But $\partial_i x^i = \delta^j_i$, so the $\partial_i$ are orthogonal to
    all other $x^j$. That makes it a normal vector to the plane spanned by the
    $\{ x^j \colon i \neq j \}$, right? Is that the reason why $\partial_x$ is
    considered a covector?
\end{aside}

\newcommand\equ[3]{\partial_{#1} g_{{#2}{#3}} = g_{\lambda{#3}} \Gamma^\lambda_{{#2}{#1}} + g_{{#2}\lambda} \Gamma^\lambda_{{#3}{#1}}}

Given is the following:
\[
    \equ\mu\alpha\beta
\]

The hint given on \parencite{wikipedia/Fundamental_Riemannian} is to permutate
the indices. With that, I get six equations:
\begin{gather*}
    \equ\mu\alpha\beta \\
    \equ\alpha\beta\mu \\
    \equ\beta\mu\alpha \\
    \equ\beta\alpha\mu \\
    \equ\alpha\mu\beta \\
    \equ\mu\beta\alpha.
\end{gather*}

The next step is to use the property that both $\tens g$ is symmetric in its
two lower indices. The $+$ also commutes. Therefore, the first and the last
equation are equivalent. Same holds for second and fifth, as well as third and
fourth. I choose to use the first three equations from now.

To isolate the Christoffel-symbols, I use the symmetry of the lower indices of
the symbols, which is equivalent to the exterior derivative having zero
torsion. There are only three distinct Christoffel-symbols of the second kind
in those equations. To simplify the next steps, I will introduce aliases:
\[
    A_1 := \partial_\mu g_{\alpha\beta}
    \eqnsep
    A_2 := \partial_\alpha g_{\beta\mu}
    \eqnsep
    A_3 := \partial_\beta g_{\mu\alpha}
    \eqnsep
    B_1 := g_{\lambda\beta} \csk\beta\alpha\mu
    \eqnsep
    B_2 := g_{\lambda\mu} \csk\mu\beta\alpha
    \eqnsep
    B_3 := g_{\lambda\alpha} \csk\alpha\mu\beta.
\]

The system of the three equations now becomes the easier to read
\begin{align*}
    B_1 \phantom{+B_2} + B_3 &= A_1 \\
    B_1 + B_2 \phantom{+ B_3} &= A_2 \\
    \phantom{B_1 +} B_2 + B_3 &= A_3.
\end{align*}

I will omit the solving of with Gaussian elimination here and just state the
solution for $B_3$ that I obtained:
\[
    B_3 = \frac12 (A_1 - A_2 + A_3).
\]

By expanding the aliases again,
\[
    g_{\lambda\alpha} \csk\alpha\mu\beta = \frac12 (\partial_\mu g_{\alpha\beta} - \partial_\alpha g_{\beta\mu} + \partial_\beta g_{\mu\alpha})
\]
it can be seen that it is getting close to the desired result. The last step is
to premultiply and contract with $g^{\eta\alpha}$ on both sides. That will give
me—apart from other indices—the definition of the Christoffel-symbols of the
second kind:
\begin{align*}
    g^{\eta\alpha} g_{\lambda\alpha} \csk\alpha\mu\beta &= \frac12
    g^{\eta\alpha} (\partial_\mu g_{\alpha\beta} - \partial_\alpha g_{\beta\mu}
    + \partial_\beta g_{\mu\alpha}) \\ 
    \csk\eta\mu\beta &= \frac12
    g^{\eta\alpha} (\partial_\mu g_{\alpha\beta} - \partial_\alpha g_{\beta\mu}
    + \partial_\beta g_{\mu\alpha}).
\end{align*}

If you substitute $\eta\mapsto\gamma$, $\mu\mapsto\beta$, $\beta\mapsto\alpha$
and $\alpha\mapsto\phi$, you will get the version on the problem set.

\section{Determinant Identities}

\begin{aside}
    $\omega(h_1, \ldots, h_n)$ is the action of the tensor $\tens \omega$ onto
    the vectors.

    I am not sure about the requirements, but is this somewhat equivalent to
    the contraction of $\tens \omega$ with all those vectors. Is the
    requirement just the existence or the choice of a basis for the vector
    space?
\end{aside}

Let me include pictures for the definition of the determinant. First, I need to
define the notation:
\begin{center}
    \includegraphics{Drawing-0005.pdf}
\end{center}

With that, the determinant is (for $n = 3$):
\begin{center}
    \includegraphics{Drawing-0006.pdf}
\end{center}

In \parencite{penrose-road_to_reality}, this is done a little different. There,
he uses:
\begin{center}
    \includegraphics{Drawing-0007.pdf}
\end{center}

That takes care of the normalisation by using another $\tens \omega$.

\subsection{}

\paragraph{Argumentation}

Any other set of $\vec h_\mu$ can be constructed from $\ev_\mu$ with another
linear transformation $\tens B$ such that $\vec h_\mu = \tens B \ev_\mu$.

\parencite{wikipedia/Trace} states that the trace of $\tens B^{-1} \tens A
\tens B$ is the same as the one of $\tens A$. So with $\tens{\tilde A} = \tens
B^{-1} \tens A \tens B$, the basis could be changed without altering the trace.

\paragraph{Showing Identity}

Again, I just have to draw a diagram for this.
\begin{center}
    \includegraphics{Drawing-0008.pdf}
\end{center}

As I see this, the trace is the sum of the determinants of matrices that only
affect one basis vector. These matrices are the identity matrix which one row
exchanged with a row from $\tens A$, like so:
\[
    \begin{pmatrix}
        1 & 0 & 0 \\
        4 & 2 & 7 \\
        0 & 0 & 1
    \end{pmatrix}.
\]

To obtain the determinant, I expand them with the method by Laplace:
\[
    = 4 \begin{pmatrix}
        0 & 0 \\ 0 & 1
    \end{pmatrix}
    - 2 \begin{pmatrix}
        1 & 0 \\ 0 & 1
    \end{pmatrix}
    + 7 \begin{pmatrix}
        1 & 0 \\ 0 & 0
    \end{pmatrix}
\]

Only the 2 on the diagonal has an effect here. Therefore, that sum will give
the trace.

\subsection{}

It should be shown that the following the relation holds:
\[
    \dpd{}\alpha \det(\tens A(\alpha)) = \det(\tens A(\alpha)) \Tr\del{\tens
    A^{-1}(\alpha) \dpd{}\alpha \tens A(\alpha)}.
\]

To do this, I will start with both sides and convert them into the same
expression. I'll start with the left hand side. To simplify, I will just assume
that $n = 2$, so that the product rule does not have that many terms. But the
principle stays the same.
\begin{align*}
    \text{LHS}
    &= \dpd{}\alpha \det(\tens A(\alpha)) \\
    &= \frac{1}{n!} \dpd{}\alpha \omega(\tens A(\alpha) \ev_1, \tens A(\alpha)
    \ev_2) \\
    \intertext{%
        I was not able to derive this using the product and chain rule with
        $\tens \omega$, but I got it with indices. I chose a $\tens \omega$
        such that it is $\tens \epsilon$.
    }
    &= \frac{1}{n!} \dpd{}\alpha \epsilon_{ij} A^i_k(\alpha) \delta^k_1
    A^j_l(\alpha)
    \delta^l_2 \\
    &= \frac{1}{n!} \epsilon_{ij} \del{ A'^i_k(\alpha) \delta^k_1 A^j_l(\alpha)
    \delta^l_2 + A^i_k(\alpha) \delta^k_1 A^j_l(\alpha) \delta^l_2} \\
    \intertext{%
        Now I will get this back into the $\omega$ way of writing it.
    }
    &= \frac{1}{n!} \del{ \omega\del{\tens{A'}(\alpha) \ev_1, \tens A(\alpha)
\ev_2} + \omega\del{\tens{A}(\alpha) \ev_1, \tens {A'}(\alpha)
\ev_2} }
\end{align*}

\begin{aside}
    How do I do this using the product and chain rule correctly?
\end{aside}

Now I will turn to the right hand side.
\begin{align*}
    \text{RHS}
    &= \det(\tens A(\alpha)) \Tr\del{\tens A^{-1}(\alpha) \dpd{}\alpha \tens
A(\alpha)} \\
&= \frac{1}{n!} \omega(\tens A(\alpha) \ev_1, \tens A(\alpha) \ev_2) \frac{1}{n!}\del{
\omega(A^{-1} \partial_\alpha \tens A(\alpha) \ev_1, \ev_2) + \omega(\ev_1, A^{-1}
    \partial_\alpha A \ev_2)
} \\
\intertext{%
    For the determinants, the determinant of a product is the product of the
    determinants. Therefore, I should be able to combine those two factors into
    a single one. The determinants that came from the trace do not feature the
    same matrix as the first factor. But I think that it still works.
}
&= \frac{1}{n!} \del{
\omega\del{\tens A(\alpha) \tens A^{-1}(\alpha) \tens A'(\alpha) \ev_1, \tens
A(\alpha) \ev_2} + \omega\del{\tens A(\alpha) \ev_1, \tens A(\alpha)  \tens
A^{-1}(\alpha) \tens A'(\alpha) \ev_2}
}
\end{align*}

With $A A^{-1}$ being the identity matrix, the left and right hand sides are
exactly the same. Therefore the relation holds, given the assumptions I made
during the derivation.

\IfFileExists{\bibliographyfile}{
    \printbibliography
}{}

\end{document}

% vim: spell spelllang=en tw=79

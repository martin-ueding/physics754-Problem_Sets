\input{../header.tex}

\newcommand\problemset{1}

\hypersetup{
    pdftitle={physics754 Problem Set \problemset}
}

%\subject{}
\title{physics754 -- Problem Set \problemset}
%\subtitle{}
\author{
    Martin Ueding \\ \small{\href{mailto:mu@martin-ueding.de}{mu@martin-ueding.de}}
}
\publishers{Group 5}

\begin{document}

\maketitle

\section{A coordinate transformation}

\subsection{}

First of all, I will start by computing the components of the tensor $\pd{{x^\alpha}}{{y^\mu}}$:
\[
    \dpd{{x^\alpha}}{{y^\mu}}
    =
    \begin{pmatrix}
        1 & 0 & 0 & 0 \\
        0 & \sin(\theta) \cos(\phi) & \sin(\theta) \sin(\phi) & \cos(\theta) \\
        0 & r \cos(\theta) \cos(\phi) & r \cos(\theta) \sin(\phi) & - r \sin(\theta) \\
        0 & - r \sin(\theta) \sin(\phi) & r \sin(\theta) \cos(\phi) & 0
    \end{pmatrix}.
\]

With that, I can go through the components of $g$ and calculate each component.
I know that $g$ is symmetric, so I have only ten components to calculate. After
working with some trigonometric identities, I finally got this:
\[
    \tens g(\vec x)
    \sim
    \begin{pmatrix}
        1 & 0 & 0 & 0 \\
        0 & -1 & 0 & 0 \\
        0 & 0 & -r^2 & 0 \\
        0 & 0 & 0 & -r^2 \sin(\theta)^2
    \end{pmatrix}.
\]

I wrote “$\tens g \sim$” since $\tens g$ is a tensor of valence (0, 2) and not
a matrix with valence (1, 1). Would is be better to write “$g_{\mu\nu} =$”
instead, although the left side would have two dangling indices, whereas the
right side would not? Should one write it as “$g_{\mu\nu} = (\cdots)^\mu_\nu$”
instead?

\subsection{}

The electric field is given as
\[
    E_i(\vec x) = \frac{Q}{|x|^3} x_i.
\]

In the field tensor $\tens F$, the electric field is contained such that
$F_{0i} = E_i$. Therefore, I write
\[
    F_{0i}(\vec x) = \frac{Q}{|x|^3} x_i.
\]

From the antisymmetry, the other nonzero entries are $F_{i0}$.

\subsection{}

Now $\tens F$ will be transformed:
\begin{align*}
    \hat F_{\mu\nu}(\vec y)
    &= F_{\alpha\beta}(\vec y) \dpd{{x^\alpha}}{{y^\mu}} \dpd{{x^\beta}}{{y^\nu}}.
    \intertext{%
        Since most components of $\tens F$ are zero, I can restrict $\alpha$
        and $\beta$ such that only nonzero components remain:
    }
    &= F_{0i}(\vec y) \dpd{{x^0}}{{y^\mu}} \dpd{{x^i}}{{y^\nu}}.
    \intertext{%
        For the first partial derivative, the only nonzero part is the one for
        $\mu = 0$, which is just 1. This simplifies to:
    }
    \hat F_{0\nu}(\vec y)
    &= F_{0i}(\vec y) \dpd{{x^i}}{{y^\nu}}
    \intertext{%
        where the other nonzero components of $\hat{\tens F}$ are computed
        using the antisymmetry. Now I have summed over $i$ which gives (after
        trigonometry):
    }
    &= \frac{Q}{r^2} \begin{pmatrix}
        0 \\ 1 \\ 0 \\ 0
    \end{pmatrix}_\nu.
\end{align*}

From that, I conclude the following representation for the transformed field tensor:
\[
    \hat{\tens F}(\vec y)
    \sim
    \frac{Q}{r^2}
    \begin{pmatrix}
        0 & 1 & 0 & 0 \\
        -1 & 0 & 0 & 0 \\
        0 & 0 & 0 & 0 \\
        0 & 0 & 0 & 0
    \end{pmatrix}
\]

\IfFileExists{\bibliographyfile}{
    \printbibliography
}{}

\end{document}

% vim: spell spelllang=en

\input{../header.tex}

\newcommand\problemset{9}

\hypersetup{
    pdftitle={physics754 Problem Set \problemset}
}

\newenvironment{aside}{\itshape\small}{}

\newcommand\tS{t_\mathrm S}

%\subject{}
\title{physics754 -- Problem Set \problemset}
%\subtitle{}
\author{
    Martin Ueding \\ \small{\href{mailto:mu@martin-ueding.de}{mu@martin-ueding.de}}
}
\publishers{Group 5 -- Olaf Baake}

\begin{document}

\maketitle

\section*{H.12: The dust cloud metric on $S^3$}

\subsection*{(a) Pullback of $\tens g$}

I will call the space where the $\vec x$ are contained in $X$. The space of the
$\vec y$ will be called $Y$. Since $r(\psi)$ is given, $\varphi$ is a mapping that
maps $Y \to X$. The associated pullback $\varphi^*$ therefore operates in the
other direction, $X \to Y$.

The Jacobian
\[
    \dpd{{x^\mu}}{{y^\nu}}
\]
that I will need for the pullback, has diagonal form. The only diagonal entry
that is not 1 is the $1,1$ component. That is:
\[
    r'(\psi) = \frac{1}{\sqrt a} \cos(\psi).
\]

With that, I can write down metrical tensor that is pulled from $X$ to $Y$. The
metrical tensor and the Jacobian are diagonal, so that the resulting metrical
tensor is diagonal as well.
\begin{align*}
    \tilde g_{00}(\vec y) &= 1 \\
    %
    \tilde g_{11}(\vec y) &= g_{11}\del{\varphi(\vec y)} \frac 1a \cos(\psi)^2 \\
    &= - \frac{f(t)^2}{a} \\
    %
    \tilde g_{22}(\vec y) &= - \frac{f(t)^2}a \sin(\psi)^2 \\
    %
    \tilde g_{33}(\vec y) &= - \frac{f(t)^2}a \sin(\psi)^2 \sin(\theta)^2
\end{align*}

\subsection*{(b) Vector $\tens n$}

The $\tilde g_{00} = 1$ was shown before in part (a). For the $\tilde g_{ik}$,
I will need the Jacobian of $\tens n(t, \psi, \theta, \phi)$. With that, I can
write down the components. This is just using trigonometric identities. They
give me the results that I had previously, I do not see a point in pretty
printing all those calculations here. The last part is easy as well:
\[
    \dpd{{n^\alpha}}{{y^0}} = 0 \implies \tilde g_{i0} = 0.
\]

That leaves the interesting part, the interpretation. The vector $\tens n$ is
given as
\[
    \tens n = \begin{pmatrix}
        \cos(\psi) \\
        \sin(\psi) \sin(\theta) \cos(\phi) \\
        \sin(\psi) \sin(\theta) \sin(\phi) \\
        \sin(\psi) \cos(\theta) \\
    \end{pmatrix},
\]
which is a unit vector. Since there are three coordinates, it is a unit vector
for all $\psi$, $\theta$ and $\phi$ and the problem set says something about an
$S^3$, I think that this a $S^3$ embedded in $\mathbb E^4$.

Since metric in $\mathbb E^4$ is flat and has an all-positive signature, I can
write this sum as
\[
    \sum_{\alpha = 1}^4 \dpd{{n^\alpha}}{{y^i}} \dpd{{n^\alpha}} {{y^k}} =
    \widehat{\Dif \tens n \otimes \Dif \tens n} = \inner{\Dif \tens n}{\Dif
    \tens n}
\]
if you contract over the first upper index of both parts. That again looks like
a Gram determinant since they are square matrices:
\[
    \mathcal G = [\Dif \tens n]^\mathrm T \Dif \tens n.
\]
Since the $\sqrt{\mathcal G}$ comes up in the measure (?) on an integral on
manifolds, and the $\sqrt{|g|}$ comes up in a lot of our integrals, I assume
that they are closely related.

The metric $\tilde{\tens g}$ has the form it has in comoving coordinates. The
spatial part seems to have the same metric as a $S^3$ with a radius being some
power or root of $f(t)^2/a$.

\IfFileExists{\bibliographyfile}{
    \printbibliography
}{}

\end{document}

% vim: spell spelllang=en tw=79

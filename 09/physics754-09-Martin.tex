\input{../header.tex}

\newcommand\problemset{9}

\hypersetup{
    pdftitle={physics754 Problem Set \problemset}
}

\newenvironment{aside}{\itshape\small}{}

\newcommand\tS{t_\mathrm S}

%\subject{}
\title{physics754 -- Problem Set \problemset}
%\subtitle{}
\author{
    Martin Ueding \\ \small{\href{mailto:mu@martin-ueding.de}{mu@martin-ueding.de}}
}
\publishers{Group 5 -- Olaf Baake}

\begin{document}

\maketitle

\section*{H.12: The dust cloud metric on $S^3$}

\subsection*{(a) Pullback of $\tens g$}

I will call the space where the $\vec x$ are contained in $X$. The space of the
$\vec y$ will be called $Y$. Since $r(\psi)$ is given, $\varphi$ is a mapping that
maps $Y \to X$. The associated pullback $\varphi^*$ therefore operates in the
other direction, $X \to Y$.

The Jacobian
\[
    \dpd{{x^\mu}}{{y^\nu}}
\]
that I will need for the pullback, has diagonal form. The only diagonal entry
that is not 1 is the $1,1$ component. That is:
\[
    r'(\psi) = \frac{1}{\sqrt a} \sin(\psi).
\]

With that, I can write down metrical tensor that is pulled from $X$ to $Y$. The
metrical tensor and the Jacobian are diagonal, so that the resulting metrical
tensor is diagonal as well.
\begin{align*}
    \tilde g_{00}(\vec y) &= 1 \\
    %
    \tilde g_{11}(\vec y) &= g_{11}\del{\varphi(\vec y)} \frac 1a \cos(\psi)^2 \\
    &= - \frac{f(t)^2}{a} \\
    %
    \tilde g_{22}(\vec y) &= - f(t)^2 \frac 1a \cos(\psi)^2 \\
    %
    \tilde g_{33}(\vec y) &= - f(t)^2 \frac 1a \cos(\psi)^2 \sin(\theta)^2
\end{align*}


\IfFileExists{\bibliographyfile}{
    \printbibliography
}{}

\end{document}

% vim: spell spelllang=en tw=79

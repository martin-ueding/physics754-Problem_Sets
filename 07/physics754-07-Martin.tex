\input{../header.tex}

\newcommand\problemset{7}

\hypersetup{
    pdftitle={physics754 Problem Set \problemset}
}

\newenvironment{aside}{\itshape\small}{}

\let\Call\del
\let\Group\sbr

\newcommand\call[1]{(#1)}
\newcommand\group[1]{(#1)}

%\subject{}
\title{physics754 -- Problem Set \problemset}
%\subtitle{}
\author{
    Martin Ueding \\ \small{\href{mailto:mu@martin-ueding.de}{mu@martin-ueding.de}}
}
\publishers{Group 5 -- Olaf Baake}

\begin{document}

\maketitle

\section*{H.10: Precession of perhelia}

I assume that $u$ and $\phi$ are both functions of the curve parameter $\tau$,
which is the proper time.

\subsection*{(a) Derive stuff}

The derivatives of $r$ transform like this:
\[
    \dod r\tau = \dod u\tau \dod ru = - \frac{1}{u^2} \dod u\tau.
\]
Also
\[
    \frac{\dot x}{\dot y} = \dod xy,
\]
which can be obtained by canceling the $\dif \tau$ that is implicit in $\dot x$
and $\dot y$.

\newcommand\rS{r_\text S}

With that, I will start:
\begin{align*}
    \alpha^2 - 1 - \dot r^2 + \frac \rS r - \frac{l^2}{r^2} \sbr{1 - \frac \rS
    r} &= 0 \\
        \alpha^2 - 1 - \frac{\dot u^2}{u^4} + \rS u - l^2 u^2 \sbr{1 - \rS u}
        &= 0 \\
        \intertext{%
        Divide by $l^2$.
    }
    \frac{\alpha^2 - 1}{l^2} - \frac{\dot u^2}{l^2 u^4} + \frac{\rS u}{l^2} - u^2 \sbr{1 - \rS u}
        &= 0 \\
    \intertext{%
        Shuffle terms around.
    }
    \frac{\dot u^2}{l^2 u^4} + u^2 &= \frac{\alpha^2 - 1}{l^2} + \frac{\rS
u}{l^2} + \rS u^3 \\
\intertext{%
    In the first summand, I will replace the $l^2$ with its definition, $l =
    r^2 \dot \phi$.
}
    \frac{\dot u^2}{\dot \phi^2} + u^2 &= \frac{\alpha^2 - 1}{l^2} + \frac{\rS
u}{l^2} + \rS u^3 \\
\intertext{%
    Using the lemma (assertion would rather cut it) from above, this first
    summand can be changed to:
}
\sbr{\dod u\phi}^2 + u^2 &= \frac{\alpha^2 - 1}{l^2} + \frac{\rS
u}{l^2} + \rS u^3,
\end{align*}
which is the equation (9) from the problem set.

From that, I will just take the derivative with respect to $\phi$ of the
previous equation. $l$ can be regarded as a constant since it is the angular
momentum, which is invariant under $\phi$ since the lagrangian is invariant to
$\phi$.
\begin{align*}
    2 \dod u\phi \dod[2] u\phi + 2 u \dod u\phi &= \frac{\rS}{l^2} \dod u\phi +
    3 \rS u^2 \dod u\phi \\
    2 \dod u\phi \dod[2] u\phi + 2 u \dod u\phi - \frac{\rS}{l^2} \dod u\phi -
    3 \rS u^2 \dod u\phi &= 0 \\
    2 \dod u\phi \sbr{ \dod[2] u\phi + u  - \frac{\rS}{2l^2}  -
    \frac 32 \rS u^2  } &= 0 \\
        \intertext{%
            I cannot assert that $\dif u/ \dif \phi \neq 0$ for all possible
            values of $\tau$. So I cannot just divide by that. However, since
            this has to hold for all possible values of $\tau$, the square
            brackets have to be equal to zero. Looking at that equation will be
            enough then.
        }
        \dod[2] u\phi + u &= \frac{\rS}{2l^2} + \frac 32 \rS u^2
\end{align*}
And this is again the equation from the problem set.


\begin{aside}
    Is the angular momentum with GR mechanics just the same $r^2 \dot \phi$
    that was already obtained for non-GR mechanics? The lagrangian contains the
    metric $\tens g$, the conserved quantity implied by the continuous
    rotational symmetry might be a little different.
\end{aside}

\IfFileExists{\bibliographyfile}{
    \printbibliography
}{}

\end{document}

% vim: spell spelllang=en tw=79

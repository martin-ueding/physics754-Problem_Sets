\input{../header.tex}

\newcommand\problemset{7}

\hypersetup{
    pdftitle={physics754 Problem Set \problemset}
}

\newenvironment{aside}{\itshape\small}{}

\let\Call\del
\let\Group\sbr

\newcommand\call[1]{(#1)}
\newcommand\group[1]{(#1)}

%\subject{}
\title{physics754 -- Problem Set \problemset}
%\subtitle{}
\author{
    Martin Ueding \\ \small{\href{mailto:mu@martin-ueding.de}{mu@martin-ueding.de}}
}
\publishers{Group 5 -- Olaf Baake}

\begin{document}

\maketitle

\section*{H.10: Precession of perhelia}

I assume that $u$ and $\phi$ are both functions of the curve parameter $\tau$,
which is the proper time.

\subsection*{(a) Derive stuff}

The derivatives of $r$ transform like this:
\[
    \dod r\tau = \dod u\tau \dod ru = - \frac{1}{u^2} \dod u\tau.
\]
Also
\[
    \frac{\dot x}{\dot y} = \dod xy,
\]
which can be obtained by canceling the $\dif \tau$ that is implicit in $\dot x$
and $\dot y$.

\newcommand\rS{r_\text S}

With that, I will start:
\begin{align*}
    \alpha^2 - 1 - \dot r^2 + \frac \rS r - \frac{l^2}{r^2} \sbr{1 - \frac \rS
    r} &= 0 \\
        \iff\quad \alpha^2 - 1 - \frac{\dot u^2}{u^4} + \rS u - l^2 u^2 \sbr{1 - \rS u}
        &= 0 \\
        \intertext{%
        Divide by $l^2$.
    }
    \iff\quad \frac{\alpha^2 - 1}{l^2} - \frac{\dot u^2}{l^2 u^4} + \frac{\rS u}{l^2} - u^2 \sbr{1 - \rS u}
        &= 0 \\
    \intertext{%
        Shuffle terms around.
    }
    \iff\quad \frac{\dot u^2}{l^2 u^4} + u^2 &= \frac{\alpha^2 - 1}{l^2} + \frac{\rS
u}{l^2} + \rS u^3 \\
\intertext{%
    In the first summand, I will replace the $l^2$ with its definition, $l =
    r^2 \dot \phi$.
}
    \iff\quad \frac{\dot u^2}{\dot \phi^2} + u^2 &= \frac{\alpha^2 - 1}{l^2} + \frac{\rS
u}{l^2} + \rS u^3 \\
\intertext{%
    Using the lemma (assertion would rather cut it) from above, this first
    summand can be changed to:
}
\iff\quad \sbr{\dod u\phi}^2 + u^2 &= \frac{\alpha^2 - 1}{l^2} + \frac{\rS
u}{l^2} + \rS u^3,
\end{align*}
which is the equation (9) from the problem set.

From that, I will just take the derivative with respect to $\phi$ of the
previous equation. $l$ can be regarded as a constant since it is the angular
momentum, which is invariant under $\phi$ since the lagrangian is invariant to
$\phi$.
\begin{align*}
    \iff\quad 2 \dod u\phi \dod[2] u\phi + 2 u \dod u\phi &= \frac{\rS}{l^2} \dod u\phi +
    3 \rS u^2 \dod u\phi \\
    \intertext{%
        Just everything to one side.
    }
    \iff\quad 2 \dod u\phi \dod[2] u\phi + 2 u \dod u\phi - \frac{\rS}{l^2} \dod u\phi -
    3 \rS u^2 \dod u\phi &= 0 \\
    \intertext{%
        Factoring out the common parts.
    }
    \iff\quad 2 \dod u\phi \sbr{ \dod[2] u\phi + u  - \frac{\rS}{2l^2}  -
    \frac 32 \rS u^2  } &= 0 \\
        \intertext{%
            I cannot assert that $\dif u/ \dif \phi \neq 0$ for all possible
            values of $\tau$. So I cannot just divide by that. However, since
            this has to hold for all possible values of $\tau$, the square
            brackets have to be equal to zero. Looking at that equation will be
            enough then.
        }
        \dod[2] u\phi + u &= \frac{\rS}{2l^2} + \frac 32 \rS u^2
\end{align*}
And this is again the equation from the problem set.


\begin{aside}
    Is the angular momentum with GR mechanics just the same $r^2 \dot \phi$
    that was already obtained for non-GR mechanics? The lagrangian contains the
    metric $\tens g$, the conserved quantity implied by the continuous
    rotational symmetry might be a little different.
\end{aside}

\subsection*{(b) Pertubations}

I'll skip putting in the numbers to see that $\epsilon$ is small. I just regard
$\epsilon$ as a small parameter as written on the problem set. I just insert
the pertubation expansion of $u$ into the previous equation:
\begin{align*}
    \implies\quad\dod[2]{{u_0}}\phi + \epsilon \dod[2]{{u_1}}\phi + u_0 + \epsilon u_1
    - \frac\rS{2l^2} - \frac 32 \rS \sbr{u_0^2 + 2 \epsilon u_0 u_1 + \mathcal
    O(\epsilon^2)} &= 0 \\
    \iff\quad \dod[2]{{u_0}}\phi + u_0 - \frac\rS{2l^2} + \epsilon \dod[2]{{u_1}}\phi + \epsilon u_1 
    + \frac 32 \rS u_0^2 + 3 \epsilon \rS u_0 u_1 + \mathcal O(\epsilon^2) &= 0 \\
    \intertext{%
        $\rS$ can be written in terms of $\epsilon$:
        \[
            \frac 32 \rS = \frac{2l^2}\rS \epsilon.
        \]
        Inserting this will make one term contain an $\epsilon^2$, so I will
        just hand it over to the $\mathcal O$.
    }
    \iff\quad \dod[2]{{u_0}}\phi + u_0 - \frac\rS{2l^2} + \epsilon \dod[2]{{u_1}}\phi + \epsilon u_1 
    + \frac{2l^2}\rS \epsilon u_0^2 + \mathcal O(\epsilon^2) &= 0 \\
    \intertext{%
        Grouping by $\epsilon$.
    }
    \iff\quad \sbr{\dod[2]{{u_0}}\phi + u_0 - \frac\rS{2l^2}} + \epsilon \sbr{\dod[2]{{u_1}}\phi + u_1 
    + \frac{2l^2}\rS u_0^2} + \mathcal O(\epsilon^2) &= 0 \\
\end{align*}
This relation has to hold for any (sufficiently small) value of
$\epsilon$. The brackets have to vanish independently. That gives
the two equations that were asked for:
\[
    \Longleftarrow\quad
    \dod[2]{{ u_0 }}\phi + u_0 = \frac{\rS}{2l^2}
    \quad\land\quad
    \dod[2]{{ u_1 }}\phi + u_1 = \frac{2l^2}{\rS} u_0^2
\]

\subsection*{(c) Solutions to differential equations}

I just insert the solutions into the different equations. The second derivative
of $u_0$ is:
\[
    \dod[2]{{u_0}}\phi = - \frac\rS{2l^2} a \cos(\phi)
\]
Inserting that into the first different equation solves it. For the second one,
I take the first derivative of $u_1$:
\[
    \dod{{u_1}}\phi = B \sin(\phi) + B \phi \cos(\phi) - 2 C \sin(2\phi).
\]
The next derivative is:
\[
    \dpd[2]{{u_1}}\phi = B \cos(\phi) - B \phi \sin(\phi) + B \cos(\phi) - 4 C
    \cos(2 \phi).
\]

Plugging those two into the second equation gives me after some grouping:
\begin{align*}
    A + 2 B \cos(\phi) - 3 C \cos(2 \phi) &= \frac{\rS}{2l^2}
    + \frac{\rS a}{l^2} \cos(\phi) + \frac{\rS a^2}{2 l^2} \cos(\phi)^2 \\
    \intertext{%
        I use the double angle formula.
    }
    A + 2 B \cos(\phi) - 3 C \cos(2 \phi) &= \frac{\rS}{2l^2} + \frac{\rS
    a}{l^2} \cos(\phi) + \frac{\rS a^2}{4 l^2} \sbr{1 + \cos(2\phi)} \\
    \intertext{%
        I can recognize the definitions of $A$, $B$ and $C$ in the above
        expression. By grouping and substituting I get:
    }
    A + 2 B \cos(\phi) - 3 C \cos(2 \phi) &= A + B \cos(\phi) - 3 C
    \cos(2\phi) \\
\end{align*}
And that solves the second different equation.

\subsection*{(d) Limits}

For large $\phi$, the bounded terms in $u_1$ become irrelevant. This leaves the
middle term:
\[
    u_1 = \frac\rS{2l^2} a \phi \sin(\phi).
\]
With that, I can write:
\begin{align*}
    u_0 + \epsilon u_1
    &= \frac{\rS}{2l^2} \sbr{1 + a \sbr{ \cos(\phi) + \epsilon \phi \sin(\phi)
    }} \\
    \intertext{%
        Knowing the desired result, I can invent the needed steps here to make it
        fit. Since $\epsilon$ is small, I can use that the sine is the identity
        right around the origin, except for $\mathcal O(\epsilon^3)$. The
        cosine is constant 1 around the origin, except for $\mathcal
        O(\epsilon^2)$.
    }
    &= \frac{\rS}{2l^2} \sbr{1 + a \sbr{ \cos(\phi) \cos(\epsilon\phi) +
    \sin(\phi) \sin(\epsilon \phi) }} \\
    \intertext{%
        Using the difference formula for the cosine, I yield:
    }
    &= \frac{\rS}{2l^2} \sbr{1 + a \cos\del{[1 + \epsilon]\phi }}.
\end{align*}

The interpretation is that the minima of the radius, i.\,e. the maxima of $u$
are shifted by a factor $1-\epsilon$ on each rotation. One maxima is at $\phi =
0$. The next one is not at $\phi = 2 \piup$ as a regular ellipse would have,
but rather $\phi = 2 \piup / [1 - \epsilon]$. One could insert the value for
$\epsilon$ and get the angle that the perihelia moves each mercury year.

\IfFileExists{\bibliographyfile}{
    \printbibliography
}{}

\end{document}

% vim: spell spelllang=en tw=79

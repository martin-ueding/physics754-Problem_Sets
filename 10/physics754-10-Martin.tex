\input{../header.tex}

\newcommand\problemset{10}

\hypersetup{
    pdftitle={physics754 Problem Set \problemset}
}

\newenvironment{aside}{\itshape\small}{}

%\subject{}
\title{physics754 -- Problem Set \problemset}
%\subtitle{}
\author{
    Martin Ueding \\ \small{\href{mailto:mu@martin-ueding.de}{mu@martin-ueding.de}}
}
\publishers{Group 5 -- Olaf Baake}

\begin{document}

\maketitle

\begin{aside}
    In the energy momentum tensor $\tens T$, the electromagnetic waves are a
    form of energy density. With $\tens G = 8 \piup G \tens T$, this means that
    photons create curvature.

    The gravitational waves carry energy. Are they also a sort of energy
    density? If so, do they need to be included in the energy momentum tensor
    or are they taken care of by the Einstein tensor itself?
\end{aside}

\IfFileExists{\bibliographyfile}{
    \printbibliography
}{}

\end{document}

% vim: spell spelllang=en tw=79

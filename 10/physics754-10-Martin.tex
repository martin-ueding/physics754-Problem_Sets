\input{../header.tex}

\newcommand\problemset{10}

\hypersetup{
    pdftitle={physics754 Problem Set \problemset}
}

\newenvironment{aside}{\itshape\small}{}

%\subject{}
\title{physics754 -- Problem Set \problemset}
%\subtitle{}
\author{
    Martin Ueding \\ \small{\href{mailto:mu@martin-ueding.de}{mu@martin-ueding.de}}
}
\publishers{Group 5 -- Olaf Baake}

\begin{document}

\maketitle

\begin{aside}
    In the energy momentum tensor $\tens T$, the electromagnetic waves are a
    form of energy density. With $\tens G = 8 \piup G \tens T$, this means that
    photons create curvature.

    The gravitational waves carry energy. Are they also a sort of energy
    density? If so, do they need to be included in the energy momentum tensor
    or are they taken care of by the Einstein tensor itself?

    \vspace{10cm}
\end{aside}

\section*{P.11: Strong gravitational waves}

\subsection*{(a)}

Lower the index on $\tens A$ and form $\tens A \otimes \tens A$ with it. It
will have the needed form.

\subsection*{(b): Pull back}

I compute the Jacobian:
\[
    \dpd{{x^\mu}}{{y^\nu}} =
    \begin{pmatrix}
        1/\sqrt 2 & 1/\sqrt 2 & 0 & 0 \\
        - 1/\sqrt 2 & 1/\sqrt 2 & 0 & 0 \\
        0 & 0 & 1 & 0 \\
        0 & 0 & 0 & 1
    \end{pmatrix}^\mu_\nu
\]

With that, I can compute the elements of the transformed metric. That worked
out, so I do not write it down here explicitly.

\subsection*{(c): Inverse metric}

Now one has to compute the inverse metric. The metric is given by
\[
    \hat g_{\mu\nu}
    =
    \begin{pmatrix}
        2f & 1 \\
        1 & 0
    \end{pmatrix}^\mu_\nu.
\]
The inverse can be computed easily with the gauss algorithm. Start with
$(\hat{\tens g} | \mathbb 1)$ and bring the first part to the identity. Then
the system hat the form $(\mathbb 1 | \hat{\tens g}^{-1})$. The inverse is
\[
    \hat g_{\mu\nu}^{-1}
    =
    \begin{pmatrix}
        0 & 1 \\
        1 & -2f
    \end{pmatrix}^\mu_\nu.
\]

\subsection*{Part (d): Christoffel symbols}

The only nonzero derivatives are the $g_{00,\delta}$, such that the Christoffel
symbols only consist of $f$ and $f_{,\delta}$. I did the first, and I will not
write them down here.

\subsection*{Part (e): Additional condition}

The
\[
    A^\mu f_{,\mu} = 0
\]
means
\[
    [\partial_0 + \partial_1] f = 0
\]
when written out. This equation can be differentiated with respect to $x^0$ and
$x^1$, which will yield
\[
    [\partial_0^2 + \partial_0 \partial_1] f = 0
    \eqnsep
    [\partial_0 \partial_1 + \partial_1^2] f = 0.
\]
I then subtract those two equations giving me
\[
    [\partial_0^2 - \partial_1^2] f = 0.
\]
Subtracting this from $\dalambert f = 0$ leaves the required
\[
    [\partial_2^3 + \partial_3^2] f = 0.
\]

\subsection*{Part (f): Meets conditions}

$f$ meets the homogeneous wave equation because of the minus sign. Since it
does not depend on the first two coordinates, it also meets $A^\mu f_{,\mu} =
0$ where $\tens A$ is nonzero only for the first two components.

\section*{H.13: Geodesics}

A general geodesic equation is given:
\[
    \dod{}\tau g_{\mu\nu} \dot x^\nu = \frac12 \partial_\mu g_{\alpha\beta}
    \dot x^\alpha \dot x^\beta.
\]

The problem asks for solutions $\tens x(\tau)$ of this equation with $\| \tens
x \| = 1$ or $\| \tens x \| = 0$. That means that the $g_{\alpha\beta} \dot
x^\alpha \dot x^\beta$ is a constant, and the complete right hand side of the
equation is zero. This leaves
\[
    \dod{}\tau g_{\mu\nu} \dot x^\nu = 0.
\]

The metric is given by its nonzero components
\[
    g_{00} = 1 + f
    \eqnsep
    g_{10} = g_{01} = -f
    \eqnsep
    g_{11} = -1 + f
    \eqnsep
    g_{22} = g_{33} = -1
\]
where $f$ is a function of $x^2$ and $x^3$ only, namely
\[
    f(x^2, x^3) = [x^2]^2 - [x^3]^3.
\]

Using product and chain rule, the differential equation becomes
\[
    g_{\mu\nu,\lambda} x^\lambda \dot x^\nu + g_{\mu\nu} \ddot x^\nu = 0.
\]

The condition $\| \tens x \| = c$ can be written as
\[
    [1+f] \dot x^0 \dot x^0 - 2f \dot x^0 \dot x^1 + [-1+f] \dot x^1 \dot x^1 -
    [\dot x^2]^2 - [\dot x^2]^2 = c.
\]

I have tried to transform the equation to the coordinates $y$, but that would
require the metric $\hat{\tens g}$ to be
\[
    \hat g_{00} = \frac{1-f}2
    \eqnsep
    \hat g_{10} = \hat g_{01} = \frac{1+f}2
    \eqnsep
    \hat g_{11} = \frac{1+3f}2
    \eqnsep
    \hat g_{22} = \hat g_{33} = -1
\]
in order to match the $\| \tens y \| = c$ in those coordinates as well.
Therefore, I have continued to do this problem int the $x$ coordinates.

I then put the metric $\tens g$ into the equation that I am supposed to solve.
Since they are actually four equations, I start with $\mu = 0$. The derivatives
of the metric are the derivatives of $f$, and those are rather simple with
respect to any coordinate. Only the derivatives with respect to the components
2 and 3 are nonzero. And $\mu$ and $\nu$ have to be either 0 or 1. Therefore,
the only nonzero summands are
\[
    g_{00,2} \dot x^2 \dot x^0 + g_{01,2} \dot x^2 \dot x^1 + g_{00,3} \dot x^3
    \dot x^0 + g_{01,3} \dot x^3 \dot x^1 + g_{0\nu} \ddot x^\nu = 0.
\]
I can then expand the occurrences of the metric and yield
\[
    2 x^2 \dot x^2 \dot x^- - 2 x^2 \dot x^2 \dot x^1 - 2 x^3 \dot x^3 \dot x^0
    + 2 x^3 \dot x^3 \dot x^1 + [1-f] \ddot x^0 - f \ddot x^1 = 0.
\]
Observing that
\[
    2 x^\mu \dot x^\mu = \dod{}\tau [x^\mu]^2
\]
the previous equation can be written terser as
\[
    \dot f [\dot x^0 + \dot x^1] + [1+f] \ddot x^0 - f \ddot x^1 = 0.
\]

$\mu = 1$ gives a similar equation:
\[
    \dot f [-\dot x^0 + \dot x^1] - f \ddot x^0 + [1-f] \ddot x^1 = 0.
\]

The last two equations are simple, they are $\ddot x^2 = 0$ and $\ddot x^3 =
0$. Their solutions are just linear functions in $\tau$:
\[
    x^2(\tau) = c_1 + c_2 \tau
    \eqnsep
    x^3(\tau) = c_3 + c_4 \tau.
\]

That gives a system of coupled, second order differential equations. I think
that they are also linear.
\begin{align*}
    \dot f [\dot x^0 + \dot x^1] + [1+f] \ddot x^0 - f \ddot x^1 &= 0 \\
    \dot f [-\dot x^0 + \dot x^1] - f \ddot x^0 + [1-f] \ddot x^1 &= 0
\end{align*}

Since $f$ is a function of $x^2$ and $x^3$ only, and those are solved so far,
this is just a known function of $\tau$ at this point. I have tried to generate
two new equations by adding and subtracting the previous two:
\begin{align*}
    2 \dot f \dot x^1 + \ddot x^0 + [1 - 2f] \ddot x^1 &= 0 \\
    2 \dot f \dot x^0 - \ddot x^1 + [1 + 2f] \ddot x^0 &= 0
\end{align*}

I then tried to use the coordinates $y$ from here. By just substituting the
various derivatives of $x^\mu$ with $y^\mu$, I was able to transform the two
equations to
\begin{align*}
    \dot f \dot y^1 + \ddot y^1 + f \ddot y^0 &= 0 \\
    \dot f \dot y^0 + f \ddot y^1 &= 0.
\end{align*}
The sum of those two can be written as
\[
    \dod{}\tau f [\dot y^0 + \dot y^1] + \ddot y^1 = 0
\]
where I can drop one derivative and get
\[
    f [\dot y^0 + \dot y^1] + \dot y^1 = 0.
\]

I just do not see where I could continue from here. So far, I have not used $\|
\tens x \| = c$, which is probably required to go on. However, I do not see
how.

\IfFileExists{\bibliographyfile}{
    \printbibliography
}{}

\end{document}

% vim: spell spelllang=en tw=79
